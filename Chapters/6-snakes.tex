\chapter{Case Study 2: Snakes}
\section{Literature Review}
Snakes are long legless reptiles that display many pigmentation patterns such as rings, spots, and stripes or more complex patterns like blotches and diamonds, as shown in Fig \ref{fig:realSnakePatterns}. The snake is entirely covered in translucent scales made of keratin, under which the patterns are found. Although the scales are translucent, they may contribute to the pattern's appearance by acting as acting as prisms through which light refracts; producing an iridescent sheen. They may also limit diffusion of pigments across the scale boundary as seen in the ocellated lizard \cite{manukyan2017}.

Pigmentation patterns perform many useful functions, such as hiding the snake from prey and helping the snake itself avoid becoming prey. These functions are fulfilled by camouflaging the snake from potential predators or acting as a brightly coloured warning signal that the snake is may be venomous. An example warning pattern is seen on the north American coral snake, \textit{Micrurus fulvius}, with its distinctive red, yellow, and black coloured bands \hl{cite}. Some non-venomous snakes employ mimicry as a defence mechanism by displaying a pattern similar to a venomous snake. A well-known example of this is the scarlet king snake, \hl{Lampropeltis elapsoides}, who also displays red, black and yellow bands like \textit{M. fulvius}, but in a different order, shown in Fig \ref{fig:realSnakePatterns}.

The regions of a snake that display pigmentation patterns can be roughly partitioned as the head, body, tail, and underbelly \hl{cite?}. The main patterns of interest in this work occur on the body and tail, which are often indistinguishable. The similarity between patterns found on these regions and those found on the head or underbelly vary between species. It is not uncommon for the head or underbelly to display a different pattern than each other or what is seen on the body. This may be due to the difference in scale shape between regions as well as the visibility and importance of these regions \hl{ref}. A notable example is the ring-necked Snake which presents a plain brown color its back which serves as camouflage. When the ring-necked snake is provoked, it will display its bright red and orange underside.

%snake initial conditions (neural tube)?
%Snake growth and the effect on patterns?

\begin{figure}[hb]
	\centering
	\includegraphics[width=\columnwidth]{snakes.png}
	\caption{A collection of snakes with interesting patterns. \textbf{top-left:} The tail of a ring-necked snake. Photograph by Peter Paplanus licensed under CC BY-NC 2.0. \textbf{top-middle:} A Scarlet King Snake. Photograph by Andrew DuBois licensed under CC BY 2.0. \textbf{top-right:} A Sonoran Coral Snake. Photograph by David Jahn licensed under CC BY-SA 2.0. \textbf{bottom-left:} A python. \textbf{bottom-right:} A cornsnake. Photographs from Pixabay under CC0.}
	\label{fig:realSnakePatterns}
\end{figure}

Patterns on the ocellated lizard have been simulated by \cite{manukyan2017}. These lizards are covered in a quasi-hexagonal lattice of pigmented scales. As a juvenile, the lizard displays white spots on a brown background. This changes in adults, whose scales are coloured individually either a solid green or black. By studying pattern formation on ocellated lizards over three to four years, it was found that the colouration of each scale is driven by a reaction-diffusion system where the diffusion rate is mediated by the scale boundaries. It is hypothesized that the scale thickness hinders pigment cell interaction. Consequently, individual scales are the discrete locations the comprise the domain. Remarkably, this results in a cellular automata \hl{cite} whose state changes throughout the lifetime of the lizard.

The study of pigmentation patterns on lizards has been performed in \hl{ref}. Here they look at varying stages of pattern development in four lined snake and found that pigmentation is first seen along the spine \hl{ref}? Growth / chemotaxis 


patterns have been simulated is marcelo paper. This work does not use a purely reaction-diffusion model, but a reaction diffusion pattern is used as the starting point to pattern formation. \hl{cite marcelo snakes}

The effect of the environment and snake behaviour has been correlated with the observed patterns. Anisotropic and RD are an important component to this model. edge detection...[Camouflage by edge enhancement in animal coloration patterns and its implications for visual mechanisms.]. Participants change RD values to closely approximate perceived snake patterns. users rated contrast between pattern elements.

Snake patterns have been simulated through reaction-diffusion in Murray et al. This work uses chemotaxis, which is the effect of cell movement to areas of high chemical concentration. is likely a important effect in snake pattern development as shown in \hl{[4 line snake paper]}. These patterns have been implemented on gecko meshes as in \hl{[gecko ppr]}. 

\section{Model Description}
I have reproduced some models from behaviour ppr. 

I have also modelled various visually similar patterns.

\hl{need models with growth, to highlight and justify the program feature}

One model forms stripes and then grows to produce double stripes as seen in the copperhead.

\section{Results}

\section{Future work}
Ultimately, chemotaxis plays a large role in pattern formation on sauropad reptiles.

role of scales in pattern formation and appearance? 
