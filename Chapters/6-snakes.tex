\chapter{Case Study 2: Snakes}
\section{Literature Review}
Snakes are long legless reptiles that display many pigmentation patterns such as simple rings, spots, and stripes or more complex patterns such as blotches and diamonds, as shown in Fig \ref{fig:realSnakePatterns}. The snake is entirely covered in translucent scales under which the pigmentation patterns occur. Although the scales are translucent, they can contribute to the pattern's appearance by acting as prisms through which light refracts; producing an iridescent sheen. 

The function of these patterns is often used to help the snake avoid being preyed upon. This is achieved by camouflaging the snake from potential predators or acting as a warning signal that the snake is may be venomous. This is seen in coral snakes found in North America, with their distinctive red, yellow, and black coloured bands. Some non-venomous snakes employ mimicry as a defence mechanism by displaying a pattern similar to a venomous snake. A well-known example of this is the Louisiana milk snake, who has displays red, black and yellow bands but in a different order than the coral snake. Another use for pigmentation patterns is to allow the snake to hunt prey without being detected.

The regions of snake that display pigmentation patterns can be partitioned as the head, body, tail, and underbelly. The main patterns of interest in this work are the ones that occur on the body and tail. And in many cases the character of the pattern does not change between these regions. The underbelly scales are ventral scales wide known as and are used for movement \hl{ref}. This region might contain patterns and if so, they can have a different appearance than those found elsewhere on the snake. The patterns on the head of the snake may be the same as seen on the body or a less regular appearance. 

snake initial conditions (neural tube)?

Snake growth and the effect on patterns?
\begin{figure}[p]
	\centering
	\includegraphics[width=\columnwidth]{snake.png}
	\caption{A collection of real snakes displaying various patterns.}
	\label{fig:realSnakePatterns}
\end{figure}

The study of pigmentation patterns on lizards has been performed in \hl{ref}. Here they look at varying stages of pattern development in four lined snake and found that pigmentation is first seen along the spine \hl{ref}? Growth / chemotaxis 

Cellular automata on lizard scales has been observed in nature and modelled in \cite{}. In this work pigment patterns on ocellated lizards are determined to be a reaction-diffusion system, that drives a cellular automata. The individual scales act is discrete spatial cells as their boundaries limit diffusion between adjacent scales. This causes each cell to display a solid color.

Another work where snake patterns have been simulated is marcelo paper. This work does not use a purely reaction-diffusion model, but a reaction diffusion pattern is used as the starting point to pattern formation. \hl{cite marcelo snakes}

The effect of the environment and snake behaviour has been correlated with the observed patterns. Anisotropic and RD are an important component to this model. edge detection...[Camouflage by edge enhancement in animal coloration patterns and its implications for visual mechanisms.]. Participants change RD values to closely approximate perceived snake patterns. users rated contrast between pattern elements.

Snake patterns have been simulated through reaction-diffusion in Murray et al. This work uses chemotaxis, which is the effect of cell movement to areas of high chemical concentration. is likely a important effect in snake pattern development as shown in \hl{[4 line snake paper]}. These patterns have been implemented on gecko meshes as in \hl{[gecko ppr]}. 

\section{Model Description}
I have reproduced some models from behaviour ppr. 

I have also modelled various visually similar patterns.

\hl{need models with growth, to highlight and justify the program feature}

One model forms stripes and then grows to produce double stripes as seen in the copperhead.

\section{Results}

\section{Future work}
Ultimately, chemotaxis plays a large role in pattern formation on sauropad reptiles.

role of scales in pattern formation and appearance? 
