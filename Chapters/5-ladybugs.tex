\chapter{Case Study 1: Ladybug patterns}

\section{Literature review}
The ladybug (also known as ladybird beetle or lady beetle) is an insect in the family Coccinellidae. They are a small round beetle that ranges in length from 0.8 to 18mm \citep{King1996} and are often found in leaf piles and gardens. Ladybugs exist throughout the world and display a myriad of spot and stripe pigmentation patterns on their elytra: the two symmetric hard shells on the dorsal side of the insect. The elytra's main purpose is to protect the fragile wings located underneath and the pattern is thought to deter predators by indicating that the ladybug is bitter tasting \citep{King1996}. The pattern on one elytron is a mirror image of the pattern on the other. 

\begin{figure}[ht]
	\centering
	\includegraphics[width=\columnwidth]{realLadybeetles.png}
	\caption{A selection of \textit{H. axyridis} ladybugs displaying various spot patterns. \textit{Photographs: courtesy of \citet{entomart2019}}}
	\label{fig:realLadyBugPatterns}
\end{figure}

Understanding the ladybug life cycle can give insight into how and when their patterns form. The cycle starts with eggs laid on the underside of leaves. These eggs hatch into larvae, which then eat aphids and other food sources until they can pupate and metamorphose into adults. Immediately after pupation, the elytra appear patternless and are a pale-yellow colour. In a timespan of hours to days, dark spots emerge and become black. The yellow transitions to red, giving the ladybug its characteristic appearance. Although certain species of ladybugs are described by the number of spots on their elytra, there can be a variable number and shape of spots found on ladybugs of the same species.

\citet{Ando2018} explored the genetic mechanisms governing formation of ladybug patterns. They found a gene, \textit{pannier}, which is responsible for much of the observed pigmentation patterns in \textit{Harmonia axyridis} and \textit{Coccinella septempunctata}. Before any pigment is visible, \textit{pannier} is found in a pre-pattern on the elytra. It then promotes melanin (black) and inhibits carotenoids (red) pigmentation accumulation creating a visible pattern. Future work is needed to identify if other specific genes are involved in pigmentation expression.

Although the specific genetic mechanisms behind the pattern formation are not fully known, \citet{liaw2001} used reaction-diffusion to simulate visually similar ladybug patterns. The equations used are activator-depleted substrate (Eqn. \ref{eq:activatorSubstrateRD}) with saturation. Their results were obtained numerically by using forward Euler integration on a grid located on the surface of a partial sphere. Diffusion on the partial sphere is computed by assuming a radius of 1 and using a Laplacian defined in spherical coordinates. It was found that the domain boundary and the curvature change the final position of spots and stripes. Five models were proposed. Three species considered display black spots on a red/orange background. In particular, \textit{Platynaspidius quinquepunctatus}, \textit{Coccinella septempunctata}, and \textit{Epilachna crassimala} display 5, 7, and 10 spots respectively. \textit{Macroilleis hauseri} has brown stripes aligned with the long axis of the shell on a yellow background. \textit{Bothrocalvia albolineata} displays elongated orange loops on a brown background.

\section{Model description} 
I used \ProgramName{} to improve these models by simulating on a triangular mesh. The equations remain the activator-depleted substrate formula \citep{meinhardt1982}:
\begin{equation}
	\begin{aligned} \label{eq:ladybugRD}
   \frac{\partial u}{\partial t} &= \rho_u \frac{u^2v}{1+\kappa u^2} + \sigma_u - \mu_u u + D_u \Lap u, \\
   \frac{\partial v}{\partial t} &= -\rho_v \frac{u^2v}{1+\kappa u^2} + \sigma_v + D_v \Lap v.
	\end{aligned}
\end{equation}

The activator is represented by $u$ and is displayed as the pigmentation on the elytra. The substrate is denoted by $v$. The rate of conversion of $v$ into $u$ is determined by the reaction rate $\rho_u$. Similarly, $\rho_v$ represents how much $v$ is used in the conversion to $u$. $\kappa$ controls saturation, $\sigma_u$ and $\sigma_v$ are the base production rates, and $\mu_u$ is the decay of $u$. $\Lap$ is the discrete Laplacian with the diffusion rates being $D_u$ and $D_v$ for $u$ and $v$.

The simulation domain is a triangular mesh mimicking the real shape of the elytra. The boundary conditions are no-flux and the shell is considered one piece. One exception is the model of  \textit{B. albolineata} which contains a sink along the middle of the domain where the two shell halves would meet. 

\section{Results}
I produced and improved on ladybug patterns as they appear in \citep{liaw2001}. The patterns are simulated on a triangular mesh, which, due to its flexible representation of 2D surfaces, provides a more faithful representation of ladybug elytra compared to a partial sphere. These results are shown in Fig. \ref{fig:ladyBugPatterns}. Images of real ladybug specimens were used to determine the pattern colours, increasing the pattern's realism. The development of the simulated patterns was also visualized and this is shown in Fig. \ref{fig:ladyBugDev}. The original parameters provided by \citet{liaw2001} have been altered to be more mathematically sound and so that pattern's character on a mesh match nature. A complete list of model parameters is shown in Table \ref{tab:ladyBugParameters}. 

%This approach can produce similar patterns with a different number of spots by only varying the diffusion rate of $u$.

\begin{figure}[H]
	\centering
	\includegraphics[width=\columnwidth]{ladybugs.png}
	\caption{Simulation of ladybug patterns. \textbf{Row 1:} initial simulation state. \textbf{Row 2:} final simulation state. \textbf{Row 3:} collection of the real ladybug species. \textbf{A-E:} \textit{P. quinquepunctatus}, \textit{C. septempunctata}, \textit{E. crassimala}, \textit{M. hauseri}, and \textit{B. albolineata} respectively. 
	 \textit{Photographs: \citep{chen2008}}}
	\label{fig:ladyBugPatterns}
\end{figure}

\begin{figure}[H]
	\centering
	\includegraphics[width=\columnwidth]{ladybug-dev.png}
	\caption{Progression of ladybug patterns overtime.}
	\label{fig:ladyBugDev}
\end{figure}

\newcommand{\orig}[1]{\textcolor{gray}{(#1)}}

\begin{table}[p]
	\centering
	\begin{tabular}{cllll}
	\hline
	\textbf{Model} &\bm{$D_u$} &\bm{$D_v$} &\bm{$\kappa$} & \bm{$\sigma_u$} \\ \hline 
	A & 0.0005 & 0.035 & 0 & 0 \\ 
	B & 0.0005 & 0.025 & 0 & 0 \\ 
	C & 0.0003 & 0.024 \orig{0.04} & 0.5 \orig{0}  & 0.01 \orig{0} \\ 
	D & 0.000028 & 0.00168 \orig{0.000168} & 0.5 \orig{0.35} & 0 \\ 
	E  & 0.000026 & 0.00182 \orig{0.000182} & 0.45 & 0.0019 \\
	\hline
	\end{tabular}
	\caption {Parameter values used for ladybug models on a mesh. The following parameters remain constant for all models $dt = 0.001$, $\sigma_v= 0.1$, $\rho_u = 0.18$, $\rho_v = 0.36$, and $\mu_u = 0.08$. The total number of steps is $1,500,000$ for all models except E where it has been decreased to $500,000$. Parameters in parenthesis are the original values used by \citep{liaw2001}. This change in parameter values are due to small differences between the patterns formed on a partial sphere and a mesh. In my model of D and E, the parameter $D_v$ has been increased by an order of magnitude to obey the rule that $D_v / D_u \geq 7.8$ \citep{liaw2001}. The number of spots on A (\textit{E. crassimala}) was less than ten. To rectify this, I lowered $D_v$ from $0.04$ to $0.024$ and changed the initial morphogen distribution to a stripe at the top as seen in A. $\sigma_u$ was then increased from $0$ to $0.01$, allowing for horizontal lines to form which turn into spots over time. Another discrepancy observed with the initial parameters was that the stripes in D and E turned into spots and irregular lines near the boundary. In D, the initial morphogens propagate as a wave, leaving stripes in its wake. On a mesh, the wave was observed to outpace itself in places, causing it to self-interact and destroy the vertical line pattern. I have decreased $\kappa$ from $0.35$ to $0.05$, strengthening the tendency to form lines. I also changed the initial distribution from a vertical stripe of $u$ down the centre to include stripes along the boundary (excluding the top). This has the effect of aligning the pattern by reducing the distance the middle wave must travel and avoiding the self-interaction. The total simulation steps is reduced to 500,000 from the original model's 1,500,000 steps to account for lines becoming spots in E.}
	\label{tab:ladyBugParameters}
\end{table}

\section{Discussion and future work}
These models produce patterns on a triangular mesh and are visually similar to real ladybugs. Previously, the domain ladybug patterns were simulated on was restricted to spherical surfaces. This work allows for arbitrary surfaces which can more closely model real ladybug elytra. The changes to initial conditions and parameters provide an alternative to those found in \citep{liaw2001}. Improvements to these models can be made once the biological chemical interactions are fully understood. The most relevant insights would be, the real initial distribution of morphogens, and the actual reaction behaviour between morphogens. Further investigation should be made to determine if morphogens really diffuse across the gap in the elytra or if the elytra are at one point joined during pupation. Also, how much do patterns found on the head of the ladybug effect the elytra pattern? Another important question is if the patterns displayed are at a biological steady state or does pattern formation stop prematurely.

%Both my model and \citep{liaw2001} use a morphogen sink along the centre of the elytra to create the loop pattern. Patterns observed in D seem to be a good fit for modelling E as well.

\begin{figure}[p]
	\centering
	\includegraphics[width=\columnwidth]{ladybug.png}
	\caption{A rendering of two ladybugs on a leaf.}
	\label{fig:ladybugRender}
\end{figure}