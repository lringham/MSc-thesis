\chapter{Introduction}
From the stripes on a zebra to the spots on a leopard, nature provides a wide variety of beautiful patterns. A selection of which are shown in Fig \ref{fig:naturalPatterns1}. In 1952 Alan Turing proposed a system of partial differential equations (PDEs) aimed at explaining the formation of patterns varying spatially and temporally \citep{Turing1952}. These patterns are in terms of chemicals diffusing and reacting together through a spatial medium. Named reaction-diffusion, this idea has since become the standard in mathematical and computational modelling of natural pattern formation. Hans Meinhardt and Alfred Grier independently discovered and advanced reaction-diffusion by focusing on the roles of long-range activation and short-range inhibition as its basis \citep{Gierer1972}. Since then, advanced reaction-diffusion models have been created and used to explain many different biological patterns \citep{GarzonAlvarado2011, fowler1992modeling, lefevre2010reaction}, and further generalizations have been created to account for a wider variety of patterns \citep{KONDO2017120}.

\begin{figure}[H]
  \centering
  \includegraphics[width=\columnwidth]{fig1}
  \caption{Examples of beautiful patterns found in nature. \textcolor{citation-gray}{Photographs: Pixabay}}
  \label{fig:naturalPatterns1}
\end{figure}

There has been a large body of work with the goal of simulating reaction-diffusion patterns. These simulations can be used to give insights into the inner workings of nature or provide textures for use in computer graphics. Most simulations use regular grids to represent the space in which the partial differential equations are computed. Each grid cell represents a discrete area where chemicals are located. This arrangement of space offers many advantages such as ease of evaluation and the use of specialized graphics hardware to accelerate computation. In nature, however patterns do not arise on grids. 

\citet{Witkin1991} use reaction-diffusion as a method of texture synthesis for computer graphics. They extend the range of possible patterns from traditional reaction-diffusion by introducing anisotropic diffusion and heterogeneous diffusion rates to their simulations. Their patterns are simulated on a grid which is then mapped as a texture onto geometry represented by a parametric surface.  To avoid seams in texturing, multiple grids can be joined together by altering their boundary conditions. 

Some pattern features like alignment and scale may occur from growth of the spatial medium supporting pattern formation. A fact that Turing identified but purposely ignored. Although grid domains can support global growth, an inherent problem is simulating local growth while maintaining an overall rectangular shape. 

\citet{fowler1992modeling} modelled the patterns found on seashells using reaction-diffusion. The patterns are simulated on a polygonal mesh in a series of layers. The first layer represents the initial conditions of the pattern and each subsequent layer is a progression through time of the simulation. This approach is flexible and allows for modelling arbitrary domain growth. But it does not allow for traditional 2D pattern formation, as once a layer is simulated it is frozen in time.

\citet{Turk1991} simulated reaction-diffusion on meshes by using a proxy Voronoi mesh to represent the original triangle mesh. This proxy mesh was used as the spatial domain and chemicals were stored in its faces. Diffusion between faces is mediated by adjacent edges and their lengths. This approach provides convincing results by allowing finer pattern details than might be afforded from the original base mesh resolution. Also, there is no need to correct for pattern distortion that occurs when mapping a grid to a surface. Unfortunately, there is no consideration for growth or interaction. 

Reaction-diffusion has also been solved directly on triangular meshes \citep{Descombes2016}, avoiding the need of a proxy mesh. This work leveraged the GPU allowing for much faster pattern formation compared to CPU based computation. This speed facilitates parameter space exploration and pattern creation. 

This thesis aims to provide an environment named \ProgramName{} that allows for convenient simulation and exploration of reaction-diffusion patterns, including those formed on grids and arbitrary triangular meshes. These meshes have support for growth, anisotropic diffusion, and heterogeneous parameters. I compute reaction-diffusion directly on triangular meshes by using a hybrid CPU and GPU approach. The CPU is used for algorithms such as adaptive subdivision and user interaction and the results are synchronized with the GPU. Such algorithms do not benefit from parallelism due to their recursive or sequential nature. This hybrid approach allows me to create fast and interactive simulations and I apply this environment to explore several cases of patterns that occur in nature.

The thesis is laid out as follows: I review the principles of reaction-diffusion patterning. Then I explain the computational aspects of simulating reaction-diffusion. Next, I describe the implementation details of the \ProgramName{} system. With that basis formalized I present four case studies. The first study models ladybug pigmentation patterns which occur on the ladybug shell. \citep{Liaw2001} simulated ladybug patterns on a grid and mapped them on a sphere. This is used as the basis for my simulations performed on a triangular mesh. The second study produces reaction-diffusion patterns that look like those found on snakes. Anisotropy is used to align these patterns with respect to the snake mesh and growth is used to create more advanced patterns like rows of dual spots. The next study produces visually similar pigmentation patterns on flower petal meshes using heterogeneous parameters and anisotropic diffusion. One of the most striking aspects of real flower petals are their pigmentation patterns, an area which remains largely unexplored despite the large amount of work put into modelling other aspects of flower petals \citep{Owens2016}. Next, I present a case study on modelling the auto-immune disease psoriasis based on contemporary research into cytokine interactions. This has the potential to be a powerful representation of psoriasis as it provides a mathematical representation of the disease and a fast, non-invasive way to test the disease's response to treatments. It is hoped this model can be extended to other similar auto-immune diseases. Last, I conclude the main contributes of this work and discuss future work.