\chapter{Case Study 4: Psoriasis}

\hl{This section is out of date and requires the current version of the paper, figures, citations, and revisions}

Most skin diseases manifest themselves with reproducible patterns of skin lesions, which are conventionally described in terms of lesion morphology (e.g. macules, papules, plaques, etc) and distribution on the skin surface (Nast et al. 2016). The biological basis of pattern formation is only understood in a few special cases. For instance, the segmental pattern of herpes zoster reflects dermatomal viral reactivation through sensory nerves, and the linear pattern in Blaschko lines represents genetic mosaicism. In most cases, however, the mechanisms by which pathological processes in the skin produce reproducible patterns remain virtually unknown. 

The majority of skin diseases are inflammatory, which explains why the lesions are often red, elevated and scaly (resulting from, respectively:  vasodilation and hyperemia, inflammatory infiltrate and edema, and pathologically increased epidermal keratinization secondary to inflammation). The skin has a large surface (average 1.5 m2 - 2.0 m2) compared to its thickness (0.5 mm-4 mm; the surface-to-volume ratio of approximately 650m2/m3) (Leider 1949), and is therefore ideally suited to study the mechanisms of spatial propagation of inflammatory processes in a tissue. Psoriasis, a chronic, autoimmune inflammatory skin disease affecting 2\%-3\% of the population in Western countries (Nestle et al. 2009) provides a particularly useful model. The lesions are sharply demarcated, scaly, and distributed symmetrically on the body (Nestle et al. 2009; Griffiths and Barker 2007; Christophers 2001). The plaques evolve from pinpoint papules by centrifugal growth (Farber et al. 1985; Soltani and Van Scott 1972), and may coalesce into larger structures (Farber et al. 1985; Christophers 2001). In some instances the plaques have the appearance of rings (referred to as annular, arciform or circinate patterns) (Nast et al. 2016; Christophers 2001). The mechanisms responsible for these patterns are not readily explainable in terms of the lateral propagation of inflammation, in which one would expect gradual attenuation of inflammation due to the dilution of proinflammatory agents that diffuse in the skin. In contrast, in psoriatic lesions the intensity of inflammation is preserved throughout the whole plaque and sharply suppressed at its margin over the distance of a few millimeters. Here we show that the phenotypic features of psoriasis can be explained in terms of interactions between key pathogenic cytokines consistent with a reaction-diffusion model. This model captures all cardinal phenotypic features of psoriasis and may provide a wider framework to understand the patterning and maintenance of inflammation in other skin diseases. 

\begin{figure}[!ht]
	\centering
	\includegraphics[width=1.0\columnwidth]{Realpso.png}
	\label{fig:pso1}
\end{figure}
		
\section{Model Description}
Patterns of skin lesions in psoriasis have been reviewed using the “(psoriasis AND clinical AND (pattern OR shapes)) OR (clinical spectrum AND psoriasis)” search phrases in PubMed. We retrieved 715 papers and after reviewing the titles and abstracts we selected 40 relevant papers (Saleh and Tanner 2018; Chau et al. 2017; Hernández-Vásquez et al. 2017; Picciani et al. 2017; Di Meglio et al. 2014; M G et al. 2013; Raychaudhuri et al. 2014; Seneschal et al. 2012; Fernandes et al. 2011; Wollenberg and Eames 2011; de Oliveira et al. 2010; Meier and Sheth 2009; Balato et al. 2009; Reich et al. 2009; Ziemer et al. 2009; Cordoro 2008; Ayala 2007; Naldi and Gambini 2007; Schön et al. 2005; Gropper 2001; Morris et al. 2001; Christophers 2001; Jablonska et al. 2000; Magro and Crowson 1997; Talwar et al. 1995; Kumar et al. 1995; Beylot et al. 1979; Hodge and Comaish 1977; Lal 1966; Mitchell 1962; Lebwohl 2003; Stern 1997; Menter and Barker 1991; Buxton 1987; Champion 1986; Stankier 1974; Baker 1971; Melski et al. 1983; Rasmussen 1986; Whyte and Baughman 1964). We also reviewed 482 clinical photographs of psoriasis in the database of the Department of Dermatology University of Copenhagen and the Division of Dermatology, University of Alberta. 
The progress and treatment of the disease was simulated by numerically solving the initial value problem for the reaction-diffusion equations using the forward Euler method.  Simulations have been implemented using a custom program written in C++ using the Windows Visual Studio programming environment. To facilitate interactive exploration of the models and their parameters, computation has been accelerated (carried out in a parallel fashion) on a Nvidia GeForce GTX 850M Graphical Processing Unit, with arrays representing the previous and current state of the simulation in each step implemented as textures (Harris et al. 2005; Dematté and Prandi 2010). The textures used in all simulations had a resolution of 500 x 500 texels, with each pixel representing a sample point of a discretized patch of the skin.  Cytokine concentrations were represented with 32-bit floating point accuracy. 

\begin{figure}[!ht]
	\centering
	\includegraphics[width=1.0\columnwidth]{graph.png}
	\label{fig:pso2}
\end{figure}
	
\section{Results}
Mature psoriatic plaques often have an oval contour bearing witness to their origin from a radially growing papule, or a polycyclic contour resulting from their subsequent merging. Plaques may also have a circinate (coiled) shape, which is the predominant morphological feature in approximately 5\% of patients (Morris et al. 2001). The patterns repetitively identified in the literature and in our clinical photograph repository are listed in Fig. 1, with further morphological details characteristic of different patterns shown in Fig. S1. We have excluded linear psoriasis, psoriatic erythroderma, and guttate psoriasis from our analyses. The linear pattern is the isomorphic (Köbner) response to trauma and does not emerge spontaneously (Melski et al. 1983). Erythroderma manifests itself as diffuse redness and inflammation of the entire skin, and thus lacks features of a pattern. Guttate psoriasis is considered to be a separate clinical type of the disease (Whyte and Baughman 1964; Rasmussen 1986; Raychaudhuri et al. 2014), which is not distinguishable from papular or nummular psoriasis based on lesion morphology. 

A feature not explicitly discussed in the literature is the patterning of the plaque itself, manifest either by the shape of the scales or irregularities of the plaque surface. Early studies documented that the intensity of the inflammatory process is not homogenous within the plaques. In the very early lesions of pinpoint papules and the inflammatory infiltrate is most dense at the center, which translates into higher proliferative activity of the keratinocytes and a thicker scale centrally in the papule (Fig S1A) (Soltani and Van Scott 1972). As the lesion grows the inflammatory infiltrate becomes more irregular, with a tendency towards higher activity at the periphery and occasional hotspots inside the plaque. A growing plaque, such as a nummular lesion, is thus often slightly thicker and scalier at the periphery than in the center (Fig S1B). Likewise, the central portion of the plaque clears more rapidly during treatment, whereas the regression of inflammatory hotspots and the marginal region is delayed (Griffin et al. 1988) (Fig S1C). Large, mature plaques demonstrate a complex pattern of polygonal faceting rather than thickening of the margins (Fig S1D). 
Model of cytokine interactions in psoriasis


Cytokines IL-23, IL-17 and TNFa are central mediators in the psoriatic plaque formation, as underscored by the fact that a pharmacological blockade of either cytokine by monoclonal antibodies causes clinical remission in a large proportion of patients (Jabbar-Lopez et al. 2017). Interactions between the cytokines inferred from the available data are shown schematically in Fig 2A. The most important pathogenic cytokines are those of the IL-17 family being produced primarily by the TH17 lymphocytes (interaction o) (Krueger et al. 2012). These cells require IL-23 for expansion and activation (Wilson et al. 2007; Zheng et al. 2006; Cosmi et al. 2008), and amplify the inflammatory process by inducing other proinflammatory cytokines, the most important of which is TNFa (Boehncke and Schön 2015).  Psoriatic plaques contain both dendritic cells producing IL-23 and TH17 cells expressing the IL-23 receptor (Cosmi et al. 2008; Lee et al. 2004; Wilson et al. 2007; Tillack et al. 2014). Treatment with guselkumab, a selective therapeutic monoclonal antibody inhibiting IL-23, attenuates IL-17s in psoriatic plaques and in serum in patients with psoriasis (interaction a) (Hawkes et al. 2018; Sofen et al. 2014; Tillack et al. 2014). This attenuation is correlated with the clinical clearing of psoriasis lesions (Sofen et al. 2014). IL-17 increases the expression of TNFa (Jovanovic et al. 1998) (interaction b) and both cytokines synergize with each other (Krueger et al. 2012; Xu et al. 2017; Alzabin et al. 2012) (interactions b and  c). Therapeutic TNFa inhibition blocks IL-17 in responding patients (interaction c) (Zaba et al. 2009; Zaba et al. 2007). The positive feedback of IL-17 cytokines on their own production (interactions b,c) is supported by the findings that, on one hand, IL-17A induces IL-17C (Xu et al. 2018), and on the other hand, the therapeutic inhibition of the IL-17 receptor with brodalumab reduces IL-17 cytokine (IL-17A, C, F) expression (Russell et al. 2014). Furthermore, TNFa downregulates IL-23 (interaction d) either directly (Notley et al. 2008; Zakharova and Ziegler 2005) or indirectly via inhibition of interferons (Tillack et al. 2014; Palucka et al. 2005). Disturbance of this negative interaction is probably responsible for paradoxical induction of psoriasis in patients with rheumatoid arthritis and inflammatory bowel disease treated with TNFa antibodies (Tillack et al. 2014; Palucka et al. 2005). This induction is readily reverted by therapeutic inhibition of the excess of IL-23 by ustekinumab, an antibody binding to the p40 chain of IL-23 (Tillack et al. 2014). 

\begin{figure}[!ht]
	\centering
	\includegraphics[width=1.0\columnwidth]{Map.png}
	\label{fig:pso3}
\end{figure}

\begin{figure}[!ht]
	\centering
	\includegraphics[width=1.0\columnwidth]{development.png}
	\label{fig:pso4}
\end{figure}
	
\section{Discussion}

Since the foundation of dermatology as a medical specialty in the beginning of the 19th century, morphological patterns of different skin diseases provided a useful and robust criterion for their diagnosis and classification. However, the mechanisms through which skin diseases produce diverse patterns remained unknown. We have shown that all major morphological types of the common skin disease psoriasis (papular, small plaque, large plaque, circinate) can be generated by a reaction-diffusion model with different parameter values. The model is based on the currently known activating and inhibiting interactions between three proinflammatory cytokines: TNFa, IL-23 and IL-17. These interactions are not direct chemical reactions, but are mediated by the immunologically active cells stimulating or inhibiting the release and proliferation of intermediary cytokines. The model has a spatio-temporal character, explaining the emergence of patterns during disease development and their disappearance during subsequent treatment. Reaction-diffusion thus provides a promising framework for studying mechanisms underlying the progress and treatment of psoriasis. 

Inflammatory patterns related to psoriasis are also found in other diseases. For example, annular lesions are seen in erythema multiforme, dermatophytosis and erythema annulare centrifugum; reniform patterns in erythema gyratum repens, urticaria and lupus erythematosus; and rosettes in granuloma annulare. We thus hypothesize that reaction-diffusion models can be applied to explain the patterns of other inflammatory skin diseases, and suggest their treatment by selective cytokine inhibition, as well. Eventually, reaction-diffusion models could provide a framework for understanding the pathogenesis and pharmacologic intervention of a broad spectrum of skin diseases.