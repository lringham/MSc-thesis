\chapter{Case Study 4: Psoriasis}

\hl{This chapter is a reformatted paper [cite]}

\section*{Skin patterning in psoriasis by spatial interactions between pathogenic cytokines}
\subsubsection*{Lee Ringham, Przemyslaw Prusinkiewicz, Robert Gniadecki}

\section{Summary} 
Disorders of human skin manifest themselves with patterns of lesions ranging from simple scattered spots to complex rings and spirals. These patterns are an essential characteristic of the disease, yet the mechanisms through which they arise remain unknown. Here we show that all known patterns of psoriasis, a common inflammatory skin disease, can be explained in terms of reaction-diffusion. We constructed a computational model based on the known interactions between the main pathogenic cytokines, IL-17, IL-23 and TNF$\alpha$. Simulations revealed that the parameter space of the model contained all classes of psoriatic lesion patterns. They also faithfully reproduced the growth and evolution of the plaques, and the response to treatment by cytokine targeting. Thus, pathogenesis of inflammatory diseases, such as psoriasis, may readily be understood in the framework of the stimulatory and inhibitory interactions between a few diffusing mediators.

\section{Keywords}
autoimmune inflammation, psoriasis, patterning, reaction-diffusion, computational modelling.

\section{Introduction} 
Most skin diseases manifest themselves with reproducible patterns of skin lesions, which are conventionally described in terms of lesion morphology (e.g. macules, papules, plaques, etc) and distribution on the skin surface \citep{nast2016}. The biological basis of pattern formation is only understood in a few special cases. For instance, the segmental pattern of herpes zoster reflects dermatomal viral reactivation through sensory nerves, and the linear pattern in Blaschko lines represents genetic mosaicism. In most cases, however, the mechanisms by which pathological processes in the skin generate reproducible patterns remain virtually unknown \citep{nast2016}.
The majority of skin diseases are inflammatory, which explains why the lesions are often red, elevated and scaly (resulting from, respectively: vasodilation and hyperemia, inflammatory infiltrate and edema, and pathologically increased epidermal keratinization secondary to inflammation). The skin has a large surface (average 1.5 m$^2$ - 2.0 m$^2$) compared to its thickness (0.5 mm-4 mm; the surface-to-volume ratio of approximately 650 m$^2$/m$^3$) \citep{leider1949}, and is therefore ideally suited to study the mechanisms of spatial propagation of inflammatory processes in a tissue. Psoriasis, a chronic, autoimmune inflammatory skin disease affecting $2\%-3\%$ of the population in Western countries \citep{parisi2013} provides a particularly useful model. The lesions are sharply demarcated, scaly, and distributed symmetrically on the body \citep{christophers2001, griffiths2007, nestle2009}. The plaques evolve from pinpoint papules by centrifugal growth, which explains an oval contour of mature lesions \citep{farber1985, soltani1972}. Individual plaques may merge producing polycyclic contours \citep{christophers2001, farber1985}. In some instances the plaques have the appearance of rings (referred to as annular, arciform or circinate patterns) \citep{christophers2001, nast2016}, which is the predominant morphological feature in approximately $5\%$ of patients \citep{morris2001}. The mechanisms responsible for these patterns are not readily explainable in terms of the lateral propagation of inflammation, in which one would expect gradual attenuation of inflammation due to the dilution of proinflammatory agents that diffuse in the skin. In contrast, in psoriatic lesions the intensity of inflammation is preserved throughout the whole plaque and sharply suppressed at its margin over the distance of a few millimeters. We show that the phenotypic features of psoriasis can be explained in terms of interactions between key pathogenic cytokines consistent with a reaction-diffusion model. This model captures all cardinal phenotypic features of psoriasis and may provide a wider framework to understand the patterning and maintenance of inflammation in other skin diseases. 

\section{Results}
\subsection{Classification of psoriasis plaque patterns}
The patterns repetitively identified in the literature and in our clinical photograph repository are listed in Fig. 1, with further morphological details characteristic of different patterns shown in Fig. S1. As detailed in Transparent Methods, we have excluded linear psoriasis, psoriatic erythroderma, and guttate psoriasis from our classification. 

A feature not explicitly discussed in the literature is the patterning of the plaque itself, manifest in the shape of the scales and/or irregularities of the plaque surface. The intensity of the inflammatory process is not homogenous within the plaques. In the very early pinpoint papules the inflammatory infiltrate is most dense at the center, which translates into higher proliferative activity of the keratinocytes and a thicker scale centrally in the papule (Fig. S1 A) \citep{soltani1972}. As the lesion grows the inflammatory infiltrate becomes more irregular, with a tendency towards higher activity at the periphery and occasional hotspots inside the plaque. A growing plaque, such as a nummular lesion, is thus often slightly thicker and scalier at the periphery than in the center. Likewise, the central portion of the plaque clears more rapidly during treatment, whereas the regression of inflammatory hotspots and the marginal region is delayed \citep{griffin1988}. Large, mature plaques demonstrate a complex pattern of polygonal faceting rather than thickening of the margins (Fig. S1D).

\subsection{Model of cytokine interactions in psoriasis}
Cytokines IL-23, IL-17 and TNF$\alpha$ are central mediators in the psoriatic plaque formation, as underscored by the fact that pharmacological blockade of either cytokine by monoclonal antibodies causes clinical remission in a large proportion of patients \citep{jabbar2017}. Interactions between the cytokines inferred from the available data are shown schematically in Fig. 2A. The most important pathogenic cytokines are those of the IL-17 family being produced primarily by the T$_{\text{H}}17$ lymphocytes (interaction \textbf{0}) \citep{krueger2012}. These cells require IL-23 for expansion and activation \citep{cosmi2008, wilson2007, zheng2007}, and amplify the inflammatory process by inducing other proinflammatory cytokines, the most important of which is TNF$\alpha$ \citep{boehncke2015}. Psoriatic plaques contain both dendritic cells producing IL-23 and T$_{\text{H}}17$ cells expressing the IL-23 receptor \citep{cosmi2008, lee2004, tillack2014, wilson2007}. Treatment with guselkumab, a selective therapeutic monoclonal antibody inhibiting IL-23, attenuates IL-17s in psoriatic plaques and in serum in patients with psoriasis (interaction \textbf{1}) \citep{hawkes2018,sofen2014,tillack2014}. This attenuation is correlated with the clinical clearing of psoriasis lesions \citep{sofen2014}. IL-17 and TNF$\alpha$ synergize with each other \citep{alzabin2012, krueger2012, xu2017}: IL-17 increases the expression of TNF$\alpha$ \citep{jovanovic1998} (interaction \textbf{2}), whereas therapeutic TNF$\alpha$ inhibition blocks IL-17 in responding patients (interaction \textbf{3}) \citep{zaba2007, zaba2009}. The positive feedback of IL-17 cytokines on their own production (interactions \textbf{2} and \textbf{3}) is further demonstrated by the findings that IL-17A induces IL-17C \citep{xu2018}, and that the therapeutic inhibition of the IL-17 receptor with brodalumab reduces the expression of the IL-17 cytokine (IL-17A, C, F) \citep{russell2014}. TNF$\alpha$ downregulates IL-23 (interaction \textbf{4}) either directly \citep{notley2008, zakharova2005} or indirectly via inhibition of interferons \citep{palucka2005, tillack2014}. Disturbance of this negative interaction is probably responsible for paradoxical induction of psoriasis in patients with rheumatoid arthritis and inflammatory bowel disease treated with TNF$\alpha$ antibodies \citep{palucka2005, tillack2014}. That induction is readily reverted by therapeutic inhibition of the excess of IL-23 by ustekinumab, an antibody binding to the p40 chain of IL-23 \citep{tillack2014}. 

\subsection{Computational model construction}
To analyze whether the molecular-level interactions depicted in Fig. 2A can account for the observed plaque patterns and the response of the disease to treatment, we constructed a mathematical model. We followed the standard method of simplifying the modeled system to focus on its essence and make the model more amenable to analysis \citep{bak1996, gaines1977, prusinkiewicz1998}. This simplification reduced the size of the parameter space and thus, to the extent possible, the use of parameters for which quantitative data are currently unavailable. It has also related the problem of plaque pattern formation to a known class of reaction-diffusion systems, which provided guidance for the exploration of the parameter space, and facilitated the analysis and interpretation of the results.
 
We have pursued the following train of thought. The mutual promotion of cytokines IL-17 and TNF$\alpha$, represented by interactions \textbf{2} and \textbf{3} in Fig. 2A, suggests that their concentrations may change in concert. Assuming this is the case, we reduced the three-substance graph in Fig. 2A by representing IL-17 and TNF$\alpha$ jointly. The resulting two-substance graph (Fig. 2B) has the structure of an activator-depleted substrate reaction-diffusion model \citep{gierer1972, marcon2016}(Fig. 2C). In this model, the substrate S with concentration s is locally converted into the activator A with concentration a according to the canonical equations \citep{gierer1972, meinhardt1982}:

\begin{equation}
	\begin{aligned} \label{eq:ahPsoriasis}
	\frac{\partial a}{\partial t} &= ka^2s+\rho_{a0}-\mu_a a + D_a \Lap a \\
	\frac{\partial s}{\partial t} &= -ka^2s+\rho_{s0}-\mu_s s + D_s \Lap s
	\end{aligned}
\end{equation}

The term $ka^2s$ indicates that the conversion is autocatalytically promoted by the activator, with the rate controlled by parameter $k$. Its concentration increases at the expense of the substrate, thus the activator down regulates the substrate. Parameters $\rho_{a0}$ and $\rho_{s0}$ are the rates of the base production of the activator and the substrate, and $\mu_a$ and $\mu_s$ control their turnover. The remaining terms, $D_a \Lap a \text{ and } D_s \Lap s$, represent diffusion of the activator and substrate at the rates controlled by parameters $D_a$ and $D_s$, respectively (for simplicity, diffusion is not explicitly represented in Figs. 2A-C). Consistent with figures 2B and C, we identify variable $a$ with the concentration of cytokines TNF$\alpha$ and IL-17, and $s$ with the concentration of IL-23:

\[a=[TNF\alpha, IL17], s=[IL23].\]

In the simulations, a patch of skin surface (Fig. 2D) is represented by an array of interconnected computational “cells”, each of which performs local computation according to Equations (1) (Fig. 2E). The initial state in all simulations is a uniform distribution of IL-23 in the whole array, except for randomly distributed small “seed” areas with a high concentration of IL-17 and TNF$\alpha$. These areas represent IL-17-secreting cells (such as the T$_{\text{H}}17$-cell) that either have been activated in situ (Fig. 2D, interaction \textbf{a}) or have migrated from the circulation to the skin or (Fig. 2D, interaction \textbf{b}) \citep{krueger2012}. 

\subsection{Exploration of the model parameter space}
Currently, it is not feasible to measure the diffusion of cytokines in human skin and consequently, there are no experimental data to provide suggestions for the parameter values of the model. Consequently, we adopted a reverse strategy where we explored the model parameter space by searching for values that would yield psoriasis patterns observed in patients (Fig. 1). To guide this search, we referred to the Gray-Scott reaction-diffusion system \citep{gray1984}, for which the parameter space has been thoroughly explored:
\begin{equation}
	\begin{aligned} \label{eq:gsPsoriasis}
	\frac{\partial a}{\partial t} &= a^2s-(f+c)a+ D_a \Lap a \\
	\frac{\partial s}{\partial t} &= -a^2s+(1-s)f + D_s \Lap s
	\end{aligned}
\end{equation}
We observe (see also \citep{yamamoto2010, yamamoto2011}) that Equations (2) are a special case of Equations (1), where
\[ k=1,\text{   }\rho_{a0}=0,\text{   }\mu_a=f+c,\text{   }\rho_{s0}=f,\text{ and }\mu_s=f.\]
The parameter space and details of six patterns obtained for specific parameter values are shown in Fig. 3. These patterns correspond visually to the six types of psoriasis identified in patients (Fig. 1). Note that, consistent with the common assumption of the Gray-Scott reaction-diffusion model, the ratio of the diffusion rates of substrate and activator was set to $D_s:D_a=2$ \citep{pearson1993}. This is a departure from the much larger ratios typically used in reaction-diffusion models \citep{diego2018, gierer1972, kondo2010, lengyel1991, marcon2016, vastano1987}. On biochemical grounds, this departure is justified by the commensurate, small size of the three cytokines, implying comparable diffusion rates (see Table S1). The small ratio of diffusion rates does not preclude Turing instability and spontaneous pattern emergence for carefully chosen values of the remaining parameters (see Fig. S2). Nevertheless, the  parameter values leading to the formation of plaque patterns are compatible with the ``filtering” operation mode, in which the patterns do not emerge spontaneously in a homogeneous medium and elaborate initial pre-patterns instead \citep{diego2018, lee1993, muratov2000, pearson1993}. This latter mode is more pertinent to the development of psoriasis plaques, which is initiated by an activated T$_{\text{H}}17$ cell in the skin (Fig. 2D). 

\subsection{The development of lesions and response to treatment}
The simulated development of psoriasis lesions and the response to treatment are shown in Fig. 4 and in Movies M1-M6. The development was simulated by using the forward Euler method to advance the state of the reaction-diffusion model over time, given an initial random distribution of small papules. The parameter values and the initial conditions for each of these simulations are listed in supplementary Table S2, with additional information characterizing the sensitivity of simulations to the variation of (individual) parameter values collected in Table S3. Minimum values of the activator A, representing cytokines IL-17 and TNF$\alpha$, needed to initiate pattern formation are collected in Table S4. The simulated patterns shown in Fig. 4 A-D3 have striking resemblance to the actual patterns of psoriatic skin lesions shown in Fig. 1. Next, we simulated the effect of therapy by increasing the decay rate of cytokines IL-17 and TNF$\alpha$ (activator A), which mimics real-life treatment with an anti-cytokine antibody. Interestingly, the simulated lesion clearing was not simply a time-reversal of the processes of plaque formation: the interior of the plaques cleared first, producing annular lesions (Fig. 4, row 5). The residual lesions dispersed slowly, eventually disappearing entirely or leaving residual spots (Fig. 4, row 6). These results closely resemble clinical situations, in which residual annular or papular lesions are often observed (Fig. S1C). 

Finally, to verify that the modeling results do not critically depend on the reduction of the three-substance system in Fig. 2A to the two-substance system in Fig. 2B, we have constructed a simulation model corresponding directly to Fig. 2A (see Supplementary Text). Guided in part by parameter values found for the two-substance model (Supplementary Tables S2 and S3), we found values for which the three-substance model produces qualitatively the same plaque patterns (Table S5). This result validates the simplification underlying the two-substance model.

\section{Discussion}
Since the foundation of dermatology as a medical specialty in the beginning of the 19th century, morphological patterns provided a useful and robust criterion for the diagnosis and classification of skin diseases. However, the mechanisms through which skin diseases produce diverse patterns remained unknown. We have shown that all major morphological types of the common skin disease psoriasis (papular, small plaque, large plaque, and different forms of circinate patterns) can be generated by a reaction-diffusion model with different parameter values. The model is based on the currently known up- and down-regulating interactions between three proinflammatory cytokines: TNF$\alpha$, IL-23 and IL-17. These interactions are not direct chemical reactions, but are mediated by the immunologically active cells stimulating or inhibiting the release and proliferation of intermediary cytokines. The model has a spatio-temporal character, explaining the emergence of patterns during disease development and their disappearance during subsequent treatment. Reaction-diffusion thus provides a promising framework for studying mechanisms underlying the progress and treatment of psoriasis. As detailed data regarding the interaction and diffusion of cytokines involved in psoriasis become available, more elaborate models may be constructed to recreate the actual biological processes in the skin with an increased accuracy. Recent advances in the theoretical understanding of reaction-diffusion \citep{diego2018} suggest that the resulting models may also become more robust to parameter changes, currently limited to narrow ranges. Inflammatory patterns related to psoriasis are found in other diseases as well. For example, annular lesions are seen in erythema multiforme, dermatophytosis and erythema annulare centrifugum; reniform patterns in erythema gyratum repens, urticaria and lupus erythematosus; and rosettes in granuloma annulare. We thus hypothesize that reaction-diffusion models can be applied further to explain the patterns of other inflammatory skin diseases, and suggest their treatment by selective cytokine inhibition. Eventually, reaction-diffusion models could provide a framework for understanding the pathogenesis and pharmacologic intervention of a broad spectrum of skin diseases. 

\section{Limitations of the Study}
The main limitation of this study is that the validity of the proposed model cannot be confirmed by direct measurements of cytokine concentration gradients in the skin. Although the diffusion rates of the cytokines are expected to be similar to each other (Table S1), which is consistent with the Gray-Scott-type model, the range of diffusion is likely to be much larger than the predicted millimeter scale due to the accrual of cytokine-secreting cells to the inflammatory infiltrate and centrifugal cell movement. Currently, the large-scale measurements of cytokine gradients in human skin are not technically feasible. 

\section{Acknowledgements}
We thank Mikolaj Cieslak for insightful discussions and comments, and Robert Munafo for advice on Gray-Scott reaction-diffusion models and their parameter space. This work was supported by the Natural Sciences and Engineering Research Council of Canada Discovery Grant 2014-05325 to PP and unrestricted research grants from the Department of Medicine, University of Alberta and Department of Dermatology, Bispebjerg Hospital, University of Copenhagen, to RG.

\section{Author contribution}
P.P. and R.G. designed research, L.R. and P.P. created mathematical model and performed computer simulations, L.R., P.P. and R.G. wrote the paper. 

\section{Declaration of interests}
L.R., P.P. and R.G. have no conflict to declare

\begin{figure}[h]
	\centering
	\includegraphics[width=1.0\columnwidth]{Realpso.png}
	\label{fig:pso1}
\end{figure}
		
\begin{figure}[h]
	\centering
	\includegraphics[width=1.0\columnwidth]{graph.png}
	\label{fig:pso2}
\end{figure}
	
\begin{figure}[h]
	\centering
	\includegraphics[width=1.0\columnwidth]{Map.png}
	\label{fig:pso3}
\end{figure}

\begin{figure}[h]
	\centering
	\includegraphics[width=1.0\columnwidth]{development.png}
	\label{fig:pso4}
\end{figure}