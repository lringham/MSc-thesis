\chapter{Inputs to program}

\section{Command line arguments}
\label{appendix:CLargs}
\begin{description}[itemsep=0cm]
    \item[ModelsPath=] Path to where OBJ models are located.
    \item[ShadersPath=] Path to where shaders are located.
    \item[ColorMapsPath=] Path to where colormaps are located.
    \item[SavePath=] Path to save output to.
    \item[ConfigFile=] SimConfig filename (eg. SimConfig.txt).
    \item[SimFile=] Filename containing the starting morphogen concentrations.
    \item[Steps=] Integer number of simulation steps until program exits.
    \item[SaveOnExit] Enable saving the model when program exits.
    \item[Run] Start the simulation running.
\end{description}

% List of reserved parameter file labels and their descriptions
\section{Reserved labels used in a parameter file}
\label{appendix:Reservedlabels}
\begin{description}[itemsep=0cm]
    \item[ModelsPath:] Filepath to folder containing OBJ models.
    \item[ColorMapsPath:] Filepath to folder containing colormaps.
    \item[ShadersPath:] Filepath to folder containing shaders.
    \item[camera:] Six comma separated floating point values representing the position and look at points used to orient the camera.
    \item[model:] Nine comma separated floating point values representing the X, Y, Z vectors used to orient the domain. 
    \item[domain:] The domain as either an OBJ filename (eg. model.obj) or grid. Only manifold OBJ meshes are supported. Grid domains look for two other parameters \textit{width} and \textit{height} which denote grid resolution.
    \item[xRes:] Integer representing width in squares of a grid domain.
    \item[yRes:] Integer representing height in squares of a grid domain.
    \item[cellSize:] A float representing spatial width of a single square in a grid domain.    
    \item[simFile:] A filename of a text file containing all per vertex values such as morphogen concentrations, vector directions and principle diffusivities. (eg. simfile.rd)
    \item[colorMap:] A filename of binary file containing a 256 RGB colormap. This is used for both inside and outside the mesh (eg. color.map).
    \item[colorMapOutside:] The colormap for the outside of the mesh.
    \item[colorMapInside:] The colormap for the inside of the mesh.
    \item[growthTickLimit:] An integer representing the number of simulation steps before the domain is grown.
    \item[growing:] \textit{true} or \textit{false} to turn growth on or off.
    \item[growthX:] A float percentage representing growth percentage on global X axis.
    \item[growthY:] A float percentage representing growth percentage on global Y axis.
    \item[growthZ:] A float percentage representing growth percentage on global Z axis.
    \item[maxFaceArea:] The face area threshold as a float for adaptive subdivision.
    \item[pauseAt:] Integer number of simulation steps until program pauses.
    \item[exitAt:] Integer number of simulation steps until program exits.
    \item[morphogens:] A comma separated list of morphogen names in uppercase (eg. A, S, U, V).
    \item[initialConditions:] The start of initial condition specification.
    \item[rdModel:] Either GPU or CPU depending on desired computation mode. Also denotes the
    \item[indices:] Specifies integer indices used to define initial conditions and parameters. Valid values are: all or 1,2,3 or 1-3.
\end{description}

\section{Three-substance model of psoriasis}
To show that the results obtained for the two substance system in Fig. \ref{fig:2}B also hold for the three-substance system in Fig. \ref{fig:2}A, we have constructed a simulation model corresponding directly to Fig. \ref{fig:2}A. The equations have the form: 
\begin{equation}
	\begin{aligned}
		\frac{\partial[TNF\alpha]}{\partial t}&=\rho_{[TNF\alpha]0}-\mu_{[TNF\alpha]}[TNF\alpha]+\eta[IL17]-k[TNF\alpha]^2[IL23] \\ & \quad{} + D_{[TNF\alpha]}\Lap[TNF\alpha]\\
		\frac{\partial[IL17]}{\partial t}     &=\rho_{[IL17]0}     -\mu_{[IL17]}[IL17]                     +k[TNF\alpha]^2[IL23]+D_{[IL17]}\Lap[IL17]\\
		\frac{\partial[IL23]}{\partial t}     &=\rho_{[IL23]0}     -\mu_{[IL23]}[IL23]                     -k[TNF\alpha]^2[IL23]+D_{[IL23]}\Lap[IL23]
	\end{aligned}
\end{equation}
Parameter values resulting in the different pattern classes shown in Fig. \ref{fig:4} are collected in Table \ref{tab:S3}.

\begin{table}[ht]
\centering
\begin{tabular}{|l|l|l|l|l|l|l|}
\hline
\textbf{Name}           & \textbf{Papular} & \textbf{\thead{Small \\ Plaque}} & \textbf{\thead{Large \\ Plaque}}  & \textbf{Annular} & \textbf{Rosette} & \textbf{Reniform} \\ \hline
{[}IL23{]}={[}IL23{]}0       & 0.04             & 0.045                 & 0.055                 & 0.001            & 0.009            & 0.011             \\ \hline
\thead{$\mu_{[TNF\alpha]}$ \\ (before treatment)} & 0.103            & 0.103                 & 0.115                 & 0.028            & 0.055            & 0.054             \\ \hline
\thead{$\mu_{[TNF\alpha]}$ \\ (during treatment)} & 0.107            & 0.1087                & 0.12                  & 0.04             & 0.0615           & 0.062             \\ \hline
maxSteps                     & 13,000           & 14,000                & 143,000               & 4,500            & 3,900            & 15,500            \\ \hline
treatSteps                   & 12,000           & 12,000                & 140,000               & 1,700            & 2,700            & 13,000            \\ \hline
\end{tabular}
\caption[Parameter values for generating the six classes of psoriasis plaque patterns]{Parameter values for generating the six classes of psoriasis plaque patterns shown in Fig. \ref{fig:4} using the three-substance model. In all simulations: $k = 1, \eta = 2, \rho_{[TNF\alpha]0} = \rho_{[IL17]0} = 0, \mu_{[IL17] = 1, D_{[IL23]} = 0.5, D_{[TNF\alpha]} = D_{[IL17]} = 0.25}$. Simulations were carried out using forward Euler methods with time-step $dt = 0.4$ for $maxSteps$ iterations, with the treatment starting after $treatSteps$ iterations. The textures used in all simulations had a resolution of 500 x 500 texels, with each texel representing a sample point of a discretized patch of the skin. Parameters of individual simulations are collected in Table \ref{tab:S1}. We assumed Neumann boundary conditions set to 0, i.e., no diffusion of activator A and substrate S across the boundary. The initial activator concentration a was set to 0 in each texel except for 50 seed spots, placed randomly across the domain.  Each spot was represented by a 3x3 array of texels with $a$ concentration of 0.5 (See Table \ref{tab:S3} for the minimum values). The initial concentration of the substrate $s$ was 1.0 everywhere.  All concentrations were represented with 32-bit floating point accuracy.}
\label{tab:S5}
\end{table}