\chapter{Principles of reaction-diffusion patterning} 
Alan Turing proposed a system of differential equations as a model for biological pattern formation \citep{turing1952}. In this system, an initially homogeneous distribution of chemicals is located in a spatial medium. This arrangement is an unstable equilibrium. The presence of small chemical perturbations instigates pattern formation by moving the system out of the equilibrium. The system responds in a few ways: over time, it can regain stability in a patterned state, oscillate between patterns, or settle in a homogeneous non-patterned state. These chemical patterns are thought to cause various phenomena such as specialization of tissue in a process known as morphogenesis. Therefore, these chemicals are referred to as morphogens\footnote{The term \Quotes{morphogen} should not be confused with Wolpert’s positional signals definition \citep{wolpert1996}.}.

\section{Reaction-diffusion in a continuous domain}
Reaction-diffusion is formalized as a set of partial differential equations that represent the change in concentration of morphogens over time. Considering morphogens $a$ and $b$, a two-substance reaction-diffusion system is defined by the system of equations:
	\begin{equation}
	\begin{aligned} \label{eq:basicRD}
		\frac{\partial a}{\partial t} &= F(a, b) + D_a \Lap a,\\
		\frac{\partial b}{\partial t} &= G(a, b) + D_b \Lap b.
	\end{aligned}
	\end{equation}
As the name suggests, reaction-diffusion is composed of two mechanisms, reaction and diffusion. Functions $F$ and $G$ describe the production and decay of $a$ and $b$, and together constitute the reactions of the system. $D_a$ and $D_b$ are coefficients that control how fast these morphogens diffuse through the domain. Physically, they depend on the morphogen particle size and permeability of the domain. $\Lap{}$ is the Laplacian operator and in conjunction with the diffusion coefficients describes the diffusion component of the system.

These equations apply at any given position in the domain in which reaction-diffusion is occurring. The size, shape, and growth of this domain may also play a role in pattern formation. Some patterns require a minimum amount of space to form. As domain size increases, the same parameters can produce different patterns. Also, the shape of the domain can affect pattern positioning and orientation.

\section{Reaction-diffusion models}
\subsection{The Turing model}
\citet{turing1952} considered reaction-diffusion on both a discrete and continuous one-dimensional ring. Initially, morphogen concentrations would be constant across the domain and the system would be in a stable state. Small noise would then instigate pattern formation. The equations he proposed had the form:
	\begin{equation}
	\begin{aligned} \label{eq:turingRD}
			\frac{\partial v}{\partial t} &= s(uv-v-\beta) + D_v \Lap v,\\
			\frac{\partial u}{\partial t} &= s(\alpha-uv) + D_u \Lap u.
	\end{aligned}
	\end{equation}
The morphogen $u$ has a base production $\alpha$. The $uv$ term describes the rate at which the morphogen $u$ is converted into the morphogen $v$. Changes to $s$, the reaction rate, scale the pattern features. For example, a spot pattern will appear larger with smaller values of parameter $s$. $\beta$ controls a constant removal of $v$.

\subsection{The activator-inhibitor model}
The concept of reaction-diffusion was reinvented by \citet{gierer1972} (see also \citep{meinhardt1982}). They considered two morphogens, an activator $a$ and an inhibitor $h$. The activator is autocatalytic; using itself to reproduce. It also spurs the production of the inhibitor, which slows autocatalysis. The interplay between these two actions is what allows for patterns to form. Meinhardt and Gierer advanced the idea that pattern formation depended on is the activator diffusing much slower than the inhibitor. Often this is the case but such drastic differences in diffusion are not always needed \citep{gray1984, marcon2016}. The activator-inhibitor model is defined as:
	\begin{equation}
	\begin{aligned} \label{eq:activatorInhibitorRD}
			\frac{\partial a}{\partial t} &= \rho \frac{a^2}{h} - \mu_a a + \rho_a + D_a \Lap a,\\
			\frac{\partial h}{\partial t} &= \rho a^2 - \mu_h h  + \rho_h + D_h \Lap h.
	\end{aligned}
	\end{equation}
$\rho$ is the reaction rate, $\mu_a$ and $\mu_h$ are the decay rates of $a$ and $h$. $\rho_a$ and $\rho_h$ represent the base production of $a$ and $h$. 

\subsection{The activator-depleted substrate model}
Another model proposed by \citet{gierer1972} is named activator-depleted substrate. In this model, the inhibitor is replaced by a substrate that the activator uses to perform autocatalysis. The inhibition mechanism results from the depletion through the consumption of the substrate by the activator. This is represented by the equations:
	\begin{equation}
	\begin{aligned} \label{eq:activatorSubstrateRD}
		\frac{\partial a}{\partial t} &= \rho sa^2 - \mu_a a + \rho_a + D_a \Lap a,\\
		\frac{\partial s}{\partial t} &= -\rho sa^2 - \mu_s s + \rho_s + D_s \Lap s.
	\end{aligned}
	\end{equation}
Here $\rho$ is the reaction rate, $\mu_a$ and $\mu_h$ are the decay rates, and $\rho_a$ and $\rho_s$ are the base production rates of $a$ and $s$.

\subsection{The Gray-Scott model}
\citet{gray1984} investigated the behaviour of a simple irreversible set of reactions and discovered it produces interesting patterns. These reactions occur in an isothermal continuous stirred tank reactor into which chemicals $U$ is continuously fed. $V$ reacts with $U$ in the tank and decays into an inert product $P$. This system is known as the Gray-Scott model and the reactions are formalized as:
	\begin{equation}
	\begin{aligned}
		U + 2V &\to 3V, \\
		V &\to P.
	\end{aligned}
	\end{equation}
When represented as partial differential equations, this system has the form:
	\begin{equation}
	\begin{aligned} \label{eq:grayscottRD}
		\frac{\partial v}{\partial t} &= uv^2 - (F+k)v + D_v \Lap v,\\
		\frac{\partial u}{\partial t} &= -uv^2 + F(1-u) + D_u \Lap u.
	\end{aligned}
	\end{equation}
$uv^2$ represents $U + 2V \to 3V$ and the constant $k$ controls the rate at which $V \to P$ occurs. $F$ is a scalar parameter that controls how much of $u$ is fed into the system and the proportion of $u \text{ and } v$ that is removed. $F$ being used for feeding and removal of $u$ has the effect of trying to keep the concentration of $u$ near $1$. Remarkably, the diffusivity coefficients use a ratio of 1:2 for the activator and substrate, which is a smaller ratio than considered by Gierer and Meinhardt. Except for a narrow range of parameters, pattern formation in this system is not instigated by noise. A pre-pattern is required to start pattern formation. 

The Gray-Scott model is a subset of activator-depleted substrate model.	Using the substitutions: $a = v$, $s = u$, $\rho = 1$, $\rho_a = 0$, $\mu_a = F+k$, $\rho_s = F$, and $\mu_s = F$, we obtain Eqn. \ref{eq:activatorSubstrateRD}. \citet{pearson1993} extensively explored and visualized the Gray-Scott model in 2D and produced many diverse patterns (Fig. \ref{fig:grayscottParameterMap}). 

\begin{figure}[H]
	\centering
	\includegraphics[scale=0.35]{fig2_1.pdf}
	\caption{Gray-Scott parameter space for parameters $k$ on the x-axis ranging from $[0.019 - 0.078]$ and $F$ on the y-axis ranging from $[0 - 0.11]$. Visualized is the concentration of morphogen $a$. Low concentration values are light blue and high concentrations are dark blue.}
	\label{fig:grayscottParameterMap}
\end{figure}

\section{Extensions to basic reaction-diffusion}
The basic reaction-diffusion concept proposed by Turing and contextualized by Gierer and Meinhardt has provided the tools for reasoning about natural pattern formation. From that idealized system, we consider extensions to more accurately represent nature and produce realistic patterns.

% statements are too unconnected

% Non-homogeneities 
One such extension is changing parameters based on their position in the domain. A commonly changed parameter is the diffusion rate. Increasing the diffusion rate along the domain causes patterns such as meandering stripes to exhibit a preferred orientation \citep{zheng2009}. \citet{witkin1991} also varied diffusion rates to correct for pattern distortion on curved surfaces. Another use of changing parameters spatially is for creating complex patterns for artistic renderings. In a grid, a texture is used to determine the parameter values of the simulation. 

% Anisotropic diffusion
Not only does diffusion vary, it can be anisotropic due to the heterogeneous structure of tissue. \citet{zhou2007} modelled pigment patterns on flower petals by considering the influence of veins on diffusion. The diffusion between adjacent cells is modified based on the presence of vascular cells. This is expanded further by considering vein width and results in darker pigment patterns along veins as seen in real flower petals. Anisotropic diffusion is used in diffusion tensor imaging to produce 3D visualizations for medical purposes \citep{bihan2001}. Anisotropic diffusion has been used with reaction-diffusion to visualize vector fields \citep{sanderson2004}. A uniform spot pattern has its diffusion driven by an underlying vector field that produces distorted ovals. In a study of snake pattern diversity, \citet{allen2013} used anisotropic diffusion to model various snakeskin pigment patterns and associated them with snake behaviour.

% Growth controlled by pattern or pattern affected by growth
Morphogen patterns may form when a plant is young and stop developing before the plant has finished growing. Subsequent plant growth will cause patterns to stretch and deform. Alternatively, patterns can develop in tandem with the growth of a plant, adjusting and migrating with available space. Consequently, growth is a common addition to reaction-diffusion simulations. Pattern driven growth on arbitrary domains has been studied to model brain development \citep{lefevre2010} as well as structure formation in plants \citep{harrison2002, holloway2007}. \citet{fowler1992} modelled pigment patterns formed on a growing seashell margin using growth. The stripe patterns found on the marine angelfish Pomacanthus have been modelled with reaction-diffusion \citep{kondo1995}. The pattern is not affixed to the underlying skin allowing for pattern development alongside animal growth. The pattern gains new stripes that insert between existing ones as an angelfish grows. Snakeskin pigment patterns have been modelled taking growth into account as well \citep{murray1991}.

% multi chemical or multi-stage
Multi-stage models where parameter values change over time have been used to simulate the pigment patterns on leopards and jaguars \citep{liu2006}. \citet{malheiros2017} simulated moray eel spots by varying diffusion rates and changing morphogen saturation limits on different areas of the body. This had the effect of creating irregular spots that changed into a thicker labyrinth-like pattern. Having more than two morphogens allows for different patterns. \citet{schenk2000} created a three-morphogen model that produces a pattern of spot clusters that move together as a group. These clusters of spots are known as quasi-particles. \citet{meinhardt1982} created a five-morphogen model that produced zebra-like stripe patterns.

It is hard to justify reaction-diffusion models as a representation of biology. This is due to a high sensitivity to small changes in parameters and the dependence on a difference in diffusion rates not seen in nature. Immobile cell-autonomous factors like stationary cells, represented as non-diffusing morphogens, are abstracted away when creating a model. Omitting non-diffusing morphogens is done to simplify the model and focus on the obvious morphogens. \citet{marcon2016} found that 70\% of three or four-morphogen systems including non-diffusing morphogens do not require differing diffusion rates. The patterns formed by these models are much less sensitive to parameter changes and the models themselves are a closer representation of reality. This implies that some basic assumptions such as long-range inhibition and short-range activation should be used as a special case instead of the standard when modelling patterns found in nature.

\citet{kondo2017} proposed a method known as the kernel-based Turing model (KT model) to generalize the process of diffusion. In the KT model, a kernel is specified that corresponds to the activation and inhibition of morphogens based on their distance from a source. A normal Gaussian curve recreates the effect of isotropic diffusion. Allowing for arbitrary kernels supports a wider range of biological functions and patterns.
