\chapter{Conclusions}
\section{Summary}

In this thesis I have introduced \ProgramName{}, an environment for quick and efficient authoring of reaction-diffusion models. To create a model, users specify their equations and parameters in a text file. These models are efficiently simulated by using parallelized computation on the GPU. \ProgramName{} also allows for expression of more complex phenomena through support of anisotropic diffusion, non-homogeneous parameters, and domain growth with adaptive subdivision. This is useful in a scientific setting, allowing a user to express and test their hypotheses easily. Also, \ProgramName{} can be used to create textures or animations of pattern development for use in other computer graphics applications such as video games.

Using \ProgramName{}, I have produced models of natural patterns on grids and triangular meshes in a series of case studies. The first study is on patterns found on the elytra of ladybugs. Spot, stripe, and loop patterns are simulated on a triangular mesh. These are created to resemble visually similar patterns found on real ladybug species. The next study simulates pigmentation patterns seen on snakes. In modelling these patterns I have used growth and anisotropic diffusion to allow for creation of more complicated patterns. Flowers are the subject of my third study. I have used non-homogeneous parameters and anisotropic diffusion on flower meshes. Lastly, a biologically motivated model of the autoimmune disease psoriasis was modelled. By modelling the interactions and diffusion of transcription factors $TNF\alpha$, $IL17$, and $IL23$ we produced many psoriasis patterns that patients exhibit.

\section{Future Work}
Most patterns in biology do not only occur in a two dimensional layer. Adding support for tetrahedral or voxel domains could allow for representation and exploration of more realistic models with thickness. Active-transport is another biological phenomenon that could be a useful tool for modelling. There have also been some extensions to reaction-diffusion such as Kondo's kernels which are a generalization of morphogen movement. Support for this would allow for a wider range of patterns. Allowing for more integration methods may result in higher performance when modelling. Support for CPU multi-threading or distributed computing can also increase the software run-time and productivity of users. 

An interesting research area that could be explored more thoroughly is the coupling of domain growth and pattern formation. The only supported operating system is Windows, adding cross-platform support would make \ProgramName{} more usable.