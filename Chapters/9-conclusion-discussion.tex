\chapter{Conclusions}
\section{Summary}
In this thesis we have detailed how the \ProgramName{} program works and through case studies we have shown the utility of this program. \ProgramName{} provides a quick, expressive and efficient way to author reaction-diffusion patterns on triangular meshes. 

\section{Future Work}
Most patterns of interest do not only occur in 1 cell thick layers. Adding support for tetrahedral models could allow for representation and exploration of more realistic models with thickness. Support for user selection of different integration methods will allow for more flexibility when modelling. Coupling bidirectional feedback between growth and morphogen concentrations would also allow for more realistic modelling. Finer control of boundary conditions, would increase the expressibility of models. Support for CPU multi-threading or distributed computing can also increase the software run-time and productivity of users. Right now the only tested and support operating system is Windows. Adding cross-platform support would make \ProgramName{} more usable. Some basic features as in model loading during runtime would increase the quality of life for users as well as leveraging an established design language found in existing software.

Conclusion / Discussion
	Read through
	Better cotan?
	Parallelize growth
	Full feature support for grids
	A standardized model format... YAML?
	Input of a mathematical function to define intial vector field?
	Better support for parameter gradients.. possibly defined by a vector and geodesic distance?
	Voroni to better support
	Support for Kondo's kernels
	Use of cell complexes for mesh representation