\chapter{Conclusions}
\section{Summary}

In this thesis I have introduced \ProgramName{}, an environment for quick and efficient authoring of reaction-diffusion models. To create a model, users specify their equations and parameters in a text file. These models are then simulated by using parallelized computation on the GPU. \ProgramName{} also allows for expression of more complex phenomena through support of anisotropic diffusion, non-homogeneous parameters, and domain growth with adaptive subdivision. This can be used to create textures or animations of pattern development for use in other computer graphics applications such as video games, movies, and rendering. Another application of this software is in a scientific setting, allowing a user to express and test their hypotheses easily. Observations of morphogen interaction seen in nature can be modelled and experimented with.

Using \ProgramName{}, I have produced models of natural patterns on grids and triangular meshes in a series of case studies. The first study models patterns found on the elytra of ladybugs. Spot, stripe, and loop patterns that resemble those found on real ladybug species are simulated on triangular meshes. The next study simulates pigmentation patterns seen on snakes. I have used growth and anisotropic diffusion to orient patterns on snake meshes. Flowers are the subject of my third study. Non-homogeneous parameters and anisotropic diffusion facilitate creation of flower petal patterns. This includes a model of the monkeyflower based on existing research into protein iterations. Lastly, a biologically motivated model of the autoimmune disease psoriasis was made. By considering the interactions and diffusion of cytokines TNF$\alpha$, IL17, and IL23 I produced many psoriasis patterns that patients exhibit.

\section{Future Work}
Most patterns in biology do not only occur in a two dimensional layer. Adding tetrahedral or voxel domains could allow for representation and exploration of domains with thickness. Active-transport is another biological phenomenon that could be a useful feature in \ProgramName{}. There have also been some extensions to reaction-diffusion such as Kondo's kernels which are a generalization of morphogen movement. Kernels would allow for a wider range of patterns. Support for CPU multi-threading or distributed computing would increase the software performance and productivity of users. 

An interesting research area that could be explored more thoroughly is the coupling of domain growth and pattern formation. Currently the only supported operating system is Windows, adding cross-platform support would allow \ProgramName{} to reach a wider user base.