\chapter{Conclusions}
\section{Contributions}
In this thesis, I have introduced \ProgramName{}, an environment for quick and efficient authoring and simulation of reaction-diffusion models on grids and triangulated manifolds. To create a model, users specify their equations and parameters in a text file. These models are then simulated on the CPU or by using parallelized computation on the GPU. \ProgramName{} provides advanced features like anisotropic diffusion, non-homogeneous parameters, and domain growth with adaptive subdivision to give users flexibility when creating patterns. Furthermore, these patterns can then be used to create textures or animations of pattern development for use in computer graphics applications such as video games and films. \ProgramName{} also has applications in a scientific setting, allowing users to model their observations and test their biological hypotheses quickly. These models can then be used to gain a better understanding of nature.

Using \ProgramName{}, I have produced models of natural patterns on grids and triangular meshes in a series of case studies. The first study concerns simulating ladybug elytra patterns. These patterns are composed of spots, stripes, and loops and are coloured in vivid red, black, brown, and yellow. To represent the elytra, I used a triangular mesh. Using a mesh provides a flexible and natural representation of the elytra's curvature and shape compared to grid-based simulations. Pattern formation occurs directly on the mesh, avoiding any mapping distortion. The colour of the simulated patterns was chosen by referencing real ladybug images.

The next study concerns pigmentation patterns seen on snakes. These patterns can be spots and stripes or more complex patterns composed of zigzags and blotches. Growth and anisotropic diffusion augment simple patterns to create more complex ones. Pattern simulation occurs on a snake-shaped mesh.

The third study deals with simulating flower petal patterns. Although the colourful patterns seen on flower petals are a defining aspect of their appearance, few studies research simulating them. I created novel models of orchids by referencing parameters from other works as well as searching the parameter space myself. The models were expanded by varying parameters spatially and by using anisotropic diffusion. I also modelled the spotted purple flower \textit{Digitalis}, among others. Using existing protein interactions that determine pigment formation, I modelled the monkeyflower species. The monkeyflower is a rare case where the real world morphogens responsible for pattern formation are known.

%TODO encorporate this into conclusions?
%Disorders of human skin manifest themselves with patterns of lesions ranging from simple scattered spots to complex rings and spirals. These patterns are an essential characteristic of the disease, yet the mechanisms through which they arise remain unknown. Here we show that all known patterns of psoriasis, a common inflammatory skin disease, can be explained in terms of reaction-diffusion. We constructed a computational model based on the known interactions between the main pathogenic cytokines, IL-17, IL-23 and TNF$\alpha$. Simulations revealed that the parameter space of the model contained all classes of psoriatic lesion patterns. They also faithfully reproduced the growth and evolution of the plaques, and the response to treatment by cytokine targeting. Thus, pathogenesis of inflammatory diseases, such as psoriasis, may readily be understood in the framework of the stimulatory and inhibitory interactions between a few diffusing mediators.

In the final case study, I presented a biologically-motivated model of the autoimmune disease psoriasis. This study took a departure from modelling pigmentation patterns and ventured into the domain of medicine. A hallmark of psoriasis is the red lesions that appear on the skin with a variety of geometric patterns. The interactions and diffusion of cell-signalling proteins called cytokines drive the disease. Precisely, we modelled the interactions of TNF$\alpha$, IL17, and IL23 to produce patterns like those exhibited in real photographs. This computational model offers an exciting approach to understanding psoriasis better because of the rapid rate in which psoriasis can be simulated compared to the actual disease. Modelling also provides an avenue for testing treatments and possibly a future cure.

\section{Future work}
From this thesis, two main avenues can be explored as future work. Firstly, \ProgramName{} can be extended with more features. Coupling growth and patterning has been shown to create compelling structures in 3D \cite{harrison2002, holloway2007}. This feature would be easy to add as the essential components are already implemented. Support for volumetric domains would allow for the exploration of 3D patterning. An area of research that would benefit this feature is the simulation of vein formation. Active-transport is an alternative to diffusion that could also be a useful feature in \ProgramName{} for modelling biological systems. There have also been some extensions to reaction-diffusion, such as Kondo's kernels, which are a generalization of morphogen activation and inhibition. The addition of kernels would allow for a broader range of patterns. 

The second area of future work concerns the technical aspects of \ProgramName{}. Support for CPU multi-threading, SIMD, or distributed computing would increase the software's performance and productivity of users. Currently, the only supported operating system is Windows, adding cross-platform support would allow \ProgramName{} to reach a more extensive user base. Finally, I intend to open-source \ProgramName{} to allow users to explore and build off my results.