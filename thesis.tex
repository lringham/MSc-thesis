%%%%%%%%%%%%%%%%%%%%%%%%%%%%%%%%%%%%%%%%%%%
%%                                                                                                              %%
%% This is file `ucalgarythesis.tex'  -- a document template for                           %%
%% graduate theses at the University of Calgary.                                               %%
%%                                                                                                              %%
%% This template document is to be used in conjunction with the                        %%
%% thesis class file `ucalgarythesis.cls'.                                                           %%
%%                                                                                                              %%
%% Created by M.W. Girard, last updated 10 April 2016.                                      %%
%%                                                                                                              %%
%% This template was created to be in compliance with the University                  %%
%% of Calgary thesis guidelines (version 14 April 2014)                                       %%
%%       https://grad.ucalgary.ca/current/thesis/guidelines.                                   %%
%%                                                                                                               %%
%%%%%%%%%%%%%%%%%%%%%%%%%%%%%%%%%%%%%%%%%%%
%%                                                                    %%
%% By default, the text of the thesis is doublespaced and should be   %%
%% printed single-sided. All margins must be exactly one inch on all  %%
%% sides, and should not be bound or have binding edges. This is      %%
%% required by the University of Calgary thesis guidelines.           %%
%% All theses are now required to be submitted electronically.        %%
%%                                                                    %%
%%%%%%%%%%%%%%%%%%%%%%%%%%%%%%%%%%%%%%%%%%%


%%%%%%%%%%%%%%%%%%%%%%%%%%%%%%%%%%%%%%%%%%%
%%                                                                                                              %%
%% Load the `thesis.cls' document class                                                           %%
%%                                                                                                              %%
%%%%%%%%%%%%%%%%%%%%%%%%%%%%%%%%%%%%%%%%%%%

  \documentclass{thesis}
  
  %%%%%%%%%%%%%%%%%%%%%%%%%%%%%%%%%%%%%%%%%%
  %%                                                                                                            %%
  %% If you would like to print a personal copy of your thesis for                          %%
  %% your own record, you may print it double-sided and with extra                      %%
  %% margins for binding. Use the following line instead of the above                   %%
  %% \documentclass{ucalgarythesis} for compiling personal copies of                  %%
  %% the thesis to be printed double-sided and bound.                                       %%
  %%                                                                                                            %%
  %%%%%%%%%%%%%%%%%%%%%%%%%%%%%%%%%%%%%%%%%%
  %\documentclass[twoside,binding]{thesis}

%%%%%%%%%%%%%%%%%%%%%%%%%%%%%%%%%%%%%%%%%%%
%%                                                                                                              %%
%% Imported packages & custom user commands                                               %%
%%                                                                                                              %%
%%%%%%%%%%%%%%%%%%%%%%%%%%%%%%%%%%%%%%%%%%%

%% Include extra packages here

  \usepackage[utf8x]{inputenc} %sometimes utf8x can cause problems?
  \usepackage{amssymb,amsthm,amsmath}
  \usepackage[hidelinks]{hyperref} 
  \usepackage{graphicx}
  \graphicspath{{./Images/}}
  \usepackage{tikz}
  \usepackage[margin=10pt,font=small,labelfont=bf, labelsep=endash]{caption}
  \usepackage[ruled, linesnumbered, noresetcount]{algorithm2e}
  \usepackage{commath}
  \usepackage{float}
  \usepackage{import}
  \usepackage{soul}
  \usepackage{xcolor}  
  \usepackage{natbib}  
  \usepackage{xifthen}
  \usepackage{pdfpages}
  \usepackage{transparent}
  \usepackage{enumitem}
  \usepackage{makecell}
  \usepackage{bm}
  \usepackage[section]{placeins}
  %\usepackage{refcheck}

  
% used for multi-line content in a single cell
\renewcommand\theadfont{\normalsize}
\renewcommand\theadalign{bl}
%%%  
  
\newcommand{\incfig}[1]{%
    \def\svgwidth{0.7\columnwidth}
    \import{./Images/}{#1.pdf_tex}
}

\newcommand{\CpyMsg}[2]{\textit{\textcopyright{} {#2} by {#1}, used with permission}}
\newcommand{\CCMsg}[3]{\textit{{#1} {#2}, licensed under {#3}}}

%%  Include other user-defined commands here

  \newcommand{\ProgramName}{LRDS}
  \newcommand{\Div}{\nabla \cdot}
  \newcommand{\Lap}{\nabla^2}
  \newcommand{\Grad}{\nabla}
  \newcommand{\LapDisc}{L}
  \newcommand{\Quotes}[1]{``{#1}''}
  \theoremstyle{plain}
  \newtheorem{theorem}{Theorem}[chapter]
  \newtheorem{lemma}[theorem]{Lemma}
  \newtheorem{corollary}[theorem]{Corollary}
  
  \theoremstyle{definition}
  \newtheorem{definition}[theorem]{Definition}
  \definecolor{citation-gray}{gray}{0}
	
\makeatletter
\def\blfootnote{\xdef\@thefnmark{}\@footnotetext}
\makeatother
    
%%%%%%%%%%%%%%%%%%%%%%%%%%%%%%%%%%%%%%%%%%%
%%                                                                                                              %%
%% Begin document                                                                                       %%
%%                                                                                                              %%
%%%%%%%%%%%%%%%%%%%%%%%%%%%%%%%%%%%%%%%%%%%
\begin{document}

%%%%%%%%%%%%%%%%%%%%%%%%%%%%%%%%%%%%%%%%%%%
%%                                                                                                              %%
%% Title page                                                                                               %%
%%                                                                                                              %%
%%%%%%%%%%%%%%%%%%%%%%%%%%%%%%%%%%%%%%%%%%%

%%%%%%%%%%%%%%%%%%%%%%%%%%%%%%%%%%%%%%%%%%%
%% Instructions for title page information:                           %%
%%                                                                    %%
%%  Fill in the following fields with the required information:       %%
%%   - \title{...}        Title of the thesis                         %%
%%   - \author{...}       Your full name                              %%
%%   - \thesis{Thesis}    Type of document (may change to `Thesis' to %%
%%                           `Dissertation' depending on type of work)%%
%%   - \dept{...}         Full name of the graduate department or     %%
%%                           degree program                           %%
%%   - \degree{...}       Full name of the degree obtained            %%
%%                          (i.e. Doctor of Philosophy,               %%
%%                                 Master of Science, etc)            %%
%%   - \gradyear{...}     Year of submission                          %%
%%   - \monthname{...}    Month of submission                         %%
%%%%%%%%%%%%%%%%%%%%%%%%%%%%%%%%%%%%%%%%%%%

  \title{Modelling natural phenomenon with reaction-diffusion}
  \author{Lee Ringham}
  \thesis{Thesis}
  \dept{GRADUATE PROGRAM IN COMPUTER SCIENCE}
  \degree{MASTER OF SCIENCE}
  \gradyear{2019}
  \monthname{APRIL}
  

%%%%%%%%%%%%%%%%%%%%%%%%%%%%%%%%%%%%%%%%%%%  
%% Make the thesis title page.
%%%%%%%%%%%%%%%%%%%%%%%%%%%%%%%%%%%%%%%%%%%
  \frontmatter           %% Don't remove this line.
  \makethesistitle       %% Don't remove this line.


%%%%%%%%%%%%%%%%%%%%%%%%%%%%%%%%%%%%%%%%%%%
%%                                                                    %%
%% Prefatory pages                                                    %%
%%                                                                    %%
%%%%%%%%%%%%%%%%%%%%%%%%%%%%%%%%%%%%%%%%%%%
%% The following sections are in the correct order as specified by    %%
%% the April 2014 thesis guidelines set by the University of Calgary. %%
%%                                                                    %%
%% You may remove optional sections, but do not change the order.     %%
%%%%%%%%%%%%%%%%%%%%%%%%%%%%%%%%%%%%%%%%%%%

%%%%%%%%%%%%%%%%%%%%%%%%%%%%%%%%%%%%%%%%%%%
%%
%% Abstract page (REQUIRED)
%%%%%%%%%%%%%%%%%%%%%%%%%%%%%%%%%%%%%%%%%%%

  \begin{thesisabstract}  
Procedural methods provide an algorithmic way to produce textures for use in computer graphics. One such method, reaction-diffusion, is a powerful mathematical approach that describes natural pattern formation in terms of chemicals known as morphogens. This thesis describes \ProgramName{}, an environment for authoring reaction-diffusion models directly on arbitrary surfaces. Morphogens, their behaviours, and the domain in which they reside can be quickly and easily defined. By performing computation on the GPU, the pattern forming simulation can be interacted with in real-time, facilitating productivity and experimentation. Four case studies are presented. The first is a simulation of ladybug pigmentation patterns. The second is a simulation of pigmentation patterns seen on the body of snakes. The third study looks at flower petal pattern modelling. Lastly, a biologically-motivated model of the autoimmune disease psoriasis is presented.
  \end{thesisabstract}

%%%%%%%%%%%%%%%%%%%%%%%%%%%%%%%%%%%%%%%%%%%
%%
%% Acknowledgements page (REQUIRED)
%%%%%%%%%%%%%%%%%%%%%%%%%%%%%%%%%%%%%%%%%%%

  \chapter{Acknowledgements}  
I want to acknowledge my supervisor Dr. Prusinkiewicz for all his guidance and support. I would also like to thank the members of the Algorithmic Botany lab, including Andrew Owens, Jeremy Hart, Cory Bloor, Mik Cieslak, Philmo Gu, Desmond Larsen-Rosner, and Pascal Ferraro. Your advice and thoughtful conversations greatly helped me to learn, develop, and enjoy this work. For their work on psoriasis, I also thank Robert Gniadecki, and finally Robert Munafo for advice on Gray-Scott reaction-diffusion models and their parameter space.

    
%%%%%%%%%%%%%%%%%%%%%%%%%%%%%%%%%%%%%%%%%%%
%% Dedication page (this section is OPTIONAL)
%%%%%%%%%%%%%%%%%%%%%%%%%%%%%%%%%%%%%%%%%%%
  \chapter[Dedication]{}
  \begin{dedication}
    \emph{To Carol, Mark, Andrew, and my loving wife Natalie.} 
  \end{dedication}

%%%%%%%%%%%%%%%%%%%%%%%%%%%%%%%%%%%%%%%%%%%
%% Various lists
%%%%%%%%%%%%%%%%%%%%%%%%%%%%%%%%%%%%%%%%%%%
%% The Table of Contents and all Lists should be single spaced.

  \begin{singlespace}   %% Do not remove this line

%%%%%%%%%%%%%%%%%%%%%%%%%%%%%%%%%%%%%%%%%%%
%%
%% Table of Contents (REQUIRED)
%%%%%%%%%%%%%%%%%%%%%%%%%%%%%%%%%%%%%%%%%%%
  \renewcommand\contentsname{Table of Contents}
  \cleardoublepage\phantomsection
  \addcontentsline{toc}{chapter}{\contentsname}
  \tableofcontents

%%%%%%%%%%%%%%%%%%%%%%%%%%%%%%%%%%%%%%%%%%%
%% List of figures (required, if any)
  \renewcommand{\listfigurename}{List of Figures and Illustrations}
  \cleardoublepage\phantomsection
  \addcontentsline{toc}{chapter}{\listfigurename}
  \listoffigures

%%%%%%%%%%%%%%%%%%%%%%%%%%%%%%%%%%%%%%%%%%%
%% List of tables (required, if any)
  \renewcommand{\listtablename}{List of Tables}
  \cleardoublepage\phantomsection
  \addcontentsline{toc}{chapter}{\listtablename}
  \listoftables
  
%%%%%%%%%%%%%%%%%%%%%%%%%%%%%%%%%%%%%%%%%%%
%% List of Symbols, abbreviations, and nomenclature (required, if any)   
%%
%% (Note: You may use your own format for your list of symbols, 
%%  abbreviations, etc. This format is just a guideline.)
%%
%  \chapter{List of Symbols, Abbreviations and Nomenclature}      
%  \begin{tabbing}
%    Symbol or abbreviation~~~~~\= \ \ \ \ \ \ \ \ \ \ \ \ \ \ \ \ \ \ \ \ \ \ \ \ \ \ \ \ \ \ \ \ \ \ \ \  \parbox{5in}{Definition}\\

%    \addsymbol \mbox{$\Lap$}: {The Laplacian operator}
%    \addsymbol \mbox{$\Grad$}: {The gradient operator}
%    \addsymbol \mbox{RD}: {reaction-diffusion}
%    \addsymbol \mbox{KT model}: {kernel-based Turing model}
%    \addsymbol \mbox{PDEs}: {Partial differential equations}
                
    % Always keep the following line if you are using a list of symbols:
%  \end{tabbing}
  
%%%%%%%%%%%%%%%%%%%%%%%%%%%%%%%%%%%%%%%%%%%
%% End single spacing after last list
  \end{singlespace}     %% Do not remove this line
  
%%%%%%%%%%%%%%%%%%%%%%%%%%%%%%%%%%%%%%%%%%%
%%                                                                    %%
%% Main matter                                                        %%
%%                                                                    %%
%%%%%%%%%%%%%%%%%%%%%%%%%%%%%%%%%%%%%%%%%%%

  \mainmatter           %% Do not remove this line
    
%%%%%%%%%%%%%%%%%%%%%%%%%%%%%%%%%%%%%%%%%%%
%%
%% Chapters
%%%%%%%%%%%%%%%%%%%%%%%%%%%%%%%%%%%%%%%%%%%
  
  \chapter{Introduction}
From the stripes on a zebra to the spots on a leopard, nature provides a wide variety of beautiful patterns (Fig. \ref{fig:naturalPatterns1}). In 1952 Alan Turing proposed a system of partial differential equations (PDEs) aimed at explaining the formation of natural patterns. The patterns are created from chemicals which diffuse and react together through a spatial medium \citep{turing1952}. The chemicals are thought to cause various phenomena such as specialization of tissue in a process known as morphogenesis. Therefore, these chemicals are referred to as morphogens\footnote{The term \Quotes{morphogen} should not be confused with Wolpert’s positional signals definition \citep{wolpert1996}.}. This system is named reaction-diffusion and has since become widely used in mathematical and computational modelling of natural pattern formation. In 1972 Hans Meinhardt and Alfred Gierer independently discovered and advanced reaction-diffusion by focusing on the roles of long-range activation and short-range inhibition \citep{gierer1972}. Since then, advanced reaction-diffusion models have been created and used to explain many different biological patterns \citep{garzon2011, fowler1992, lefevre2010, meinhardt2009}.

\begin{figure}[H]
  \centering
  \includegraphics[width=\columnwidth]{fig1}
  \caption{Examples of beautiful patterns found in nature. \textit{Photographs: pixabay.com}}
  \label{fig:naturalPatterns1}
\end{figure}

There has been a large body of work focused on simulating reaction-diffusion patterns. These simulations can be used to give insights into the inner workings of nature or provide textures for use in computer graphics. Regular grids represent the space in which most simulations compute the partial differential equations. Chemicals are found in the grid cells, which represent discrete areas. This arrangement offers many advantages, such as ease of evaluation and the use of specialized graphics hardware to accelerate computation.

\citet{witkin1991} used reaction-diffusion as a method of texture synthesis for computer graphics and extended the range of possible patterns from traditional reaction-diffusion by introducing anisotropic diffusion and varying diffusion rates in their simulations. The patterns were simulated on a grid, which was subsequently mapped onto a parametric surface as a texture. Grid boundaries were connected in the topology of a torus to avoid seams in the texture.

Growth of the spatial medium supporting pattern formation can affect the pattern's appearance\footnote{This is an observation that Turing identified but purposely ignored.}. Although grid domains can grow, they are not suitable for growth occurring just at a single location. Local growth requires an arbitrary surface not constrained to a rectangular shape.

\citet{lefevre2010} modelled the growth and folding of a brain using reaction-diffusion. In this work, they simulated a labyrinth pattern on a spherical mesh representing the brain's surface. Chemical concentrations determine the rate of mesh growth. This growth, in turn, provides more space for the pattern to develop. The result is labyrinth-like folds protruding from the surface of the mesh. \citet{harrison2002, holloway2007} modelled the growth of plants by coupling reaction-diffusion and surface deformation in the same way. \citet{fowler1992} modelled patterns found on seashells using a special case of growth. The shell domain starts as a 1D layer representing the initial conditions of the pattern. Layers are accrued over time, where each subsequent layer is a progression through time of the simulation.

\citet{turk1991} simulated reaction-diffusion on meshes by using a second Voronoi mesh to represent the original mesh surface. This Voronoi mesh was used as the spatial domain and chemicals were stored in its faces. The rate of diffusion across face edges depends on the edge lengths. Using a second mesh avoids modification of the original mesh and allows for the generation of detailed pattern textures. Also, there is no need to correct for pattern distortion that occurs when mapping a grid to an arbitrary surface. Unfortunately, there is no consideration for growth or interaction. % Finer details in the texture are captured by the second mesh containing small faces, smaller than the original mesh

Reaction-diffusion has also been solved directly on triangular meshes \citep{descombes2016}, avoiding the need for a Voronoi mesh. This work leveraged the GPU allowing for much faster simulation progression compared to CPU-based computation. This speed facilitates parameter space exploration and pattern creation. However, there is no mention of growth or anisotropic diffusion.

In this thesis, I present an environment named \ProgramName{} that allows for convenient simulation and exploration of reaction-diffusion patterns, including those formed on grids and arbitrary triangular meshes. These meshes have support for growth, anisotropic diffusion, and parameters that can change value in space and over time. Reaction-diffusion is computed directly on triangular meshes by combining the use of the CPU and GPU. Adaptive subdivision and user interaction, which do not benefit from parallelism due to their recursive or sequential nature,  use the CPU. Parallelizable algorithms, such as reaction-diffusion equations are calculated efficiently by using the GPU. This mixed approach makes it possible to create fast and interactive simulations. 

%I apply \ProgramName{} to explore several cases of patterns that occur in nature.

This thesis is organized as follows. First, I review the principles of reaction-diffusion patterning, and I explain the computational aspects of simulating reaction-diffusion. Next, I describe the implementation details of the \ProgramName{} system. On this basis, I present four case studies. The first study analyses ladybug patterns which occur on the ladybug shell. \citep{liaw2001} produced simulations of ladybug patterns on a grid located on a partial sphere. I improve on this study by using arbitrary triangular meshes to replace the partial sphere and represent ladybug shells. These meshes allow for an organic shell shape and pattern formation directly on the mesh surface. I also explore some alternative parameters and initial conditions to improve on the pattern stability and appearance. The second study produces reaction-diffusion patterns that resemble those found on snakes. Anisotropy is used to align these patterns to the snake mesh and growth is used to create more complicated patterns like rows of dual spots. The third study explores pigmentation patterns on flowers. I produced models of various flower species on meshes by using varying parameters and anisotropic diffusion. Some flowers considered are foxglove, monkeyflower, and orchids, which display vibrant patterns on attractive curved petals. Although petal pigmentation patterns are visually striking, it is an area that remains largely unexplored despite the large amount of work put into modelling other aspects of flower petals \citep{owens2016}. Next, I present a case study on modelling the autoimmune disease psoriasis based on contemporary research into cytokine interactions. A computational model of psoriasis has the potential to be a useful tool as it provides a mathematical representation of the disease and a fast, non-invasive way to test the disease's response to treatments. It is hoped this model can be extended to other autoimmune diseases as well as increasing the efficacy of treatments. Last, I conclude the main contributes of this thesis and discuss future work.

  \chapter{Principles of reaction-diffusion patterning} 
Alan Turing proposed a system of differential equations as a model for biological pattern formation \cite{Turing1952}. In this system a homogeneous distribution of chemicals is located in a spatial medium. The presence of small perturbations in the distribution instigate pattern formation by causing the system to become unstable. The system may respond in a few ways: over time, it can regain stability in a patterned state, oscillate between patterns, or revert to a non-patterned state. These chemical patterns are thought to cause various phenomena such as specialization of tissue in a process known as morphogenesis. Therefore, these chemicals are often referred to as morphogens \footnote{The term \Quotes{morphogen} should not be confused with Wolpert’s positional signals definition \cite{WolpertLewis1996}.}.

\section{Reaction-diffusion in a continuous domain}
Reaction-diffusion is formalized as a set of partial differential equations that represent the change in concentration of morphogens $a$ and $b$ over time:
	\begin{equation}
	\begin{aligned} \label{eq:basicRD}
		\frac{\partial a}{\partial t} &= F(a, b) + D_a \Lap a,\\
		\frac{\partial b}{\partial t} &= G(a, b) + D_b \Lap b.
	\end{aligned}
	\end{equation}
As the name suggests, reaction-diffusion is composed to two mechanisms, reaction and diffusion. Functions $F$ and $G$ describe the production and decay of $a$ and $b$ and together constitute the reactions of the system. $D_a$ and $D_b$ are coefficients that control how fast these morphogens diffuse through the domain. Physically, they depend on the morphogen particle size and permeability of the domain. $\Lap{}$ is the Laplacian operator and in conjunction with the diffusion coefficients determine the diffusion component of the system.

These equations apply at any given position in a domain in which reaction-diffusion is occurring. The size, shape, and growth of the domain also play a critical role in pattern formation. Some patterns require a minimum amount of space to form. As domain size increases, the same parameters can produce different patterns such as spots and rings. Also, the shape of the domain can affect pattern positioning and orientation.

\section{Reaction-diffusion models}
\subsection{The Turing model}
Th most common instance of Turing's model is:
\begin{equation}
\begin{aligned} \label{eq:turingRD}
		\frac{\partial a}{\partial t} &= s(\alpha-ab) + D_a \Lap a,\\
		\frac{\partial b}{\partial t} &= s(ab-b-\beta) + D_b \Lap b.
\end{aligned}
\end{equation}
The morphogen $a$ has a base production $\alpha$ and $ab$ describes the rate at which $a$ is converted into $b$. 

\subsection{The activator-inhibitor model}
The concept of reaction-diffusion was reinvented by Alfred Gierer and Hans Meinhardt \cite{Gierer1972}. In this work they considered two morphogens, an activator $a$ and an inhibitor $h$. The activator is autocatalytic; using itself to reproduce. It also spurs production of the inhibitor which slows autocatalysis. The interplay between these two actions is what allows for patterns to stabilize. It was thought that another key mechanism that pattern formation depended on is the activator diffusing much slower than the inhibitor. Although, in later models, such drastic differences in diffusion are not always needed. The activator-inhibitor model is defined as:
\begin{equation}
\begin{aligned} \label{eq:activatorInhibitorRD}
		\frac{\partial a}{\partial t} &= \rho \frac{a^2}{h} - \mu_a a + \rho_a + D_a \Lap a,\\
		\frac{\partial h}{\partial t} &= \rho a^2 - \mu_h h  + \rho_h + D_h \Lap h.
\end{aligned}
\end{equation}
$\rho$ is the reaction rate, $\mu_a$ and $\mu_h$ are the decay rates of $a$ and $h$. $\rho_a$ and $\rho_h$ represent the base production of $a$ and $h$. 

\subsection{The activator-depleted substrate model}
Another proposed model is called the activator-depleted substrate. In this model the inhibitor is replaced by a substrate that the activator uses to perform autocatalysis. The inhibition mechanism is then provided by the depletion through consumption of the substrate by the activator. This is represented by:
	\begin{equation}
	\begin{aligned} \label{eq:activatorSubstrateRD}
		\frac{\partial a}{\partial t} &= \rho sa^2 - \mu_a a + \rho_a + D_a \Lap a,\\
		\frac{\partial s}{\partial t} &= -\rho sa^2 - \mu_s s + \rho_s + D_s \Lap s.
	\end{aligned}
	\end{equation}
Here $\rho$ is the reaction rate, $\mu_a$ and $\mu_h$ are the decay rates, and $\rho_a$ and $\rho_s$ are the base production rates of $a$ and $s$.

\subsection{The Gray-Scott model}
Gray and Scott investigated the behaviour of a simple irreversible set of reactions and discovered they produce interesting patterns \cite{Gray1984}. These reactions occur in an isothermal continuous stirred tank reactor into which chemicals $U$ and $V$ are continuously fed. $V$ reacts with $U$ to perform autocatalysis and also decays into an inert product $P$. This system is known as the Gray-Scott model and the reactions are formalized as:
	\begin{equation}
	\begin{aligned}
	U + 2V &\to 3V, \\
	V &\to P.
	\end{aligned}
	\end{equation}
John Pearson extensively explored and visualized the Gray-Scott model in 2D and produced many diverse patterns \cite{pearson1993}. The range of patterns can be seen in Fig \ref{fig:grayscottParameterMap}. When represented as partial differential equations we see that this system is a subset of activator-depleted substrate:
	\begin{equation}
	\begin{aligned} \label{eq:grayscottRD}
	\frac{\partial v}{\partial t} &= uv^2 - (F+k)v + D_v \Lap v,\\
	\frac{\partial u}{\partial t} &= -uv^2 + F(1-u) + D_u \Lap u.
	\end{aligned}
	\end{equation}
Remarkably, the diffusivity coefficients use a ratio of 1:2 for the activator and substrate, which is a more equivalent ratio than considered by Gierer and Meinhardt. Also, pattern formation in this system is not instigated by auto-excitation. Instead of noise, a pre-pattern is required to start pattern formation. $F$ and $k$ are scalar parameters where $F$ controls how much of $u$ is fed into the system and the proportion that is removed. Being used for both purposes has the effect of trying to keep the concentration of $u$ near $1$. $k$ is added to $F$ to determine the decay rate of $v$. Implying $v$ is always removed from the system at a faster rate than $u$.

\begin{figure}[H]
	\centering
	\incfig{fig2_1}
	\caption{Gray-Scott parameter space for parameters $k$ on the x-axis ranging from $[0.019 - 0.78]$ and $F$ on the y-axis ranging from $[0 - 0.11]$. Visualized is the concentration of morphogen $a$. Low concentration values are light blue and high concentrations are dark blue.}
	\label{fig:grayscottParameterMap}
\end{figure}

\section{Extensions to basic reaction-diffusion}

The basic reaction-diffusion concept proposed by Turing and contextualized by Gierer and Meinhardt has provided the tools for reasoning about natural pattern formation. From that idealized system we consider extensions to more accurately represent nature and produce realistic patterns.

% Non-homogeneities 
One such extension is adding non-homogeneities to the underlying domain. One example is spatially varying diffusion rates along the domain. This can be used to cause patterns such as meandering stripes to exibit a defined orientation. This was hypothesized as a mechanism explaining stripes found on fish by \cite{zheng2009}. Witkin and Kass also varied diffusion rates to correct for pattern distortion on curved surfaces \cite{Witkin1991}. Zhou et al. modelled pigment patterns on flower petals by considering the influence of veins on diffusion rates \cite{Zhou2007}. The diffusion between adjacent cells is modified based on two cell types: vascular and tissue. Diffusion is highest between vascular cells, less between vascular and tissue cells, and lowest between tissue cells. This is expanded further by considering vein width. Creating darker pigment patterns along veins as seen in real flower petals.

% Anisotropic diffusion
Not only does diffusion vary, it is often anisotropic due to the heterogeneous structure of tissue. This property is used in diffusion tensor imaging to produce 3D visualizations for medical purposes \cite{Bihan2001}. Anisotropic diffusion has been used with reaction-diffusion to visualize vector fields \cite{Sanderson2004}. A uniform spot pattern has its diffusion driven the vector field which produces distorted ovals. In a study of snake pattern diversity, Allen et al. used anisotropic diffusion to model various snakeskin pigment patterns and associated them with snake behaviour \cite{Allen2013}.

% Growth controlled by pattern or pattern affected by growth
Morphogen patterns may develop when an animal is an infant and fix in place while to animal grows into adulthood. This will cause patterns to stretch and deform. Alternatively, they can develop in tandem with the growth of an animal, adjusting and migrating with available space. Consequently, growth is a common addition to reaction-diffusion simulations. Fowler et al. modelled pigment patterns formed on a growing seashell margin \cite{fowler1992modeling}. Each new layer in the shell is a progression through time of the pattern. This results in a 2D pattern as layers of the shell are accrued. The stripe patterns found on the marine angelfish Pomacanthus have been modelled with reaction-diffusion due to their dynamic movement \cite{Kondo1995}. The pattern is not affixed to the underlying skin allowing for development with growth. The pattern gains new stripes that insert between existing ones as the angelfish grows. Snakeskin pigment patterns have been modelled using growth as well \cite{MURRAY1991}.

% multi chemical or multi stage
Multi-stage models where parameter values change over time have been used to simulate the pigment patterns on leopards and jaguars \cite{liu2006two}. Malheiros and Walter simulated moray eel spots by varying diffusion rates and changing morphogen saturation limits on different areas of the body. This had the effect of creating irregular spots that change into a thicker labyrinth like pattern \cite{malheiros2017}. Having more than two morphogens allow for different patterns such as the 3-chemical localized moving structures known as quasi-particles \cite{schenk2000quasi}. Or even the 5-chemical model that creates zebra like stripe patterns found in \cite{Meinhardt1982}.

The accuracy and legitimacy of traditional reaction-diffusion in the context of biology is often criticized. This is due to a high sensitivity to small changes in parameters and the dependence on a difference in diffusion rates not seen in nature. Often when making a simulation of a biological phenomenon we abstract away immobile cell-autonomous factors and only focus on the obvious morphogens. These immobile factors represent cells which are affected by and produce other morphogens and can be thought of as non-diffusing morphogens as well. Previously, they were seen to be of lesser importance to pattern formation, but it was found that 70\% of three or four morphogen systems including them do not require differing diffusion rates. And the patterns are much less sensitive to parameter changes \cite{Marcon2016}. By including these factors, reaction-diffusion models can be more biologically accurate and realistic. This implies that some basic assumptions such as long-range inhibition and short-range activation should be used as a special case instead of the
standard when modelling patterns found in nature.

A method known as the kernel-based Turing model (KT model), was proposed to generalize the process of diffusion \cite{KONDO2017120}. In the KT model, a kernel is specified that corresponds to the activation and inhibition of the present morphogens with respect to the distance from a source. A normal gaussian curve recreates the effect of standard diffusion. Allowing for arbitrary kernels supports a wider range of biological functions and patterns.
  \newcommand{\Morph}{u}

\chapter{Computational aspects}
\section{Systems of chemical reactions}
Reaction-diffusion equations are an idealized and abstract representation of reality. They capture two phenomena, chemical reactions and diffusion. Mathematically, this is represented as a system of partial differential equations, whose numerical solution requires discretization of time and space. Computation of the chemical reactions at a specific point in space depends on the elapsed time and previous concentration. Computation of diffusion, however, depends additionally on the concentrations at adjacent points.

\section{Isotropic diffusion}
Diffusion is a process in which particles of a substance move from areas with high concentration to areas with low concentration. The change in concentration over space is referred to as a concentration gradient. Diffusion was formalized in 1855 by Fick's second law:
\begin{equation}
\label{eq:ficks2ndlaw0}
	\frac{\partial \Morph}{\partial t} = \Div (D \nabla \Morph).
\end{equation}
The symbol $\Div$ is the divergence operator, $\nabla$ is the gradient operator, and $\Morph$ is a scalar field. Given Eqn. \ref{eq:ficks2ndlaw0}, we see that the diffusion rate depends on the gradient of the concentration. At a given location, the divergence of the gradient of $\Morph$ measures the difference between the concentration at that point and the average of the neighbouring concentrations. Diffusion is proportional to this change in concentration. However, particle size and domain porosity also affect the diffusion rate. This is represented by the diffusivity coefficient $D$. If diffusivity is the same regardless of direction, this diffusion is said to be isotropic, and we can simplify Eqn. \ref{eq:ficks2ndlaw0} to:
\begin{equation}
\label{eq:ficks2ndlaw1}
	\frac{\partial \Morph}{\partial t} = D \Lap \Morph,
\end{equation}
where $\Lap$ is the divergence of the gradient and is called the Laplacian. Formally, it is the sum of the second spatial derivative in each basis direction $x_i$.
\begin{equation}
\label{eq:laplacian}
	\Lap = \sum_{i=0}^{n} \frac{\partial^2}{\partial x_i^2}.
\end{equation}
	
\section{Anisotropic diffusion}
In many biological scenarios, particles do not diffuse at an equal rate in all directions. In that case, the diffusion is called anisotropic, and the diffusivity coefficient changes based on the direction considered. For anisotropic diffusion, a single scalar $D$ no longer suffices. We can represent anisotropy by a tensor $\Lambda$. In the two-dimensional case, the tensor has the form:
\begin{equation}
\label{eq:diffTensor}
\Lambda =
\begin{bmatrix}
    \lambda_1 & 0 \\
    0 & \lambda_2 
\end{bmatrix}
\end{equation}
Each $\lambda_i$ represents the diffusivity in a basis of the space. In practice, $\lambda_1$ and $\lambda_2$ are used to represent a portion of the desired scalar diffusion rate. To represent isotropic diffusion, $\lambda_1=1$ and $\lambda_2=1$. $\Lambda$ is axis-aligned. Arbitrary orientations can be calculated with a rotation matrix $R$, allowing us to represent general anisotropic diffusion:
\begin{equation}
\label{eq:anisoDiffTensor}
	D = R^T \Lambda R.
\end{equation}
We can use $D$ to transform the gradient and thereby the diffusion rate, obtaining Eqn. \ref{eq:ficks2ndlaw0}.

\section{Discrete diffusion operators}
To simulate reaction-diffusion, the domain on which it is simulated must be discretized, which in turn means that we must use a discrete version of the Laplacian.

\subsection{Diffusion on grids}
Grids of square cells are a common representation of domains. This has many benefits as grids are easy to represent, and the required differential operators are simple to implement. Each cell in the grid has an area associated with concentrations of morphogens. Computationally, concentrations are represented by a scalar value assigned to each cell.

\subsubsection*{Diffusion in 1D} 
Diffusion on a 1D grid of cells can be computed by representing the Laplacian through finite differencing operations. Because the Laplacian involves second-order derivatives, we can approximate it by computing the difference of two first-order differences for a grid cell centred at $i$, where $i$ is the $i^{th}$ cell:
\begin{equation}
\begin{aligned} \label{eq:firstOrderDifference}
	T_0 &= \frac{(\Morph_{i} - \Morph_{i-1})}{h},\\
	T_1 &= \frac{(\Morph_{i+1} - \Morph_i)}{h}.
\end{aligned}
\end{equation}
Here $\Morph_{i-1}, \Morph_i,$ and $\Morph_{i+1}$ are morphogen concentrations at their respective grid cells, $T_0$ and $T_1$ are the first order differences, and $h$ is the distance between the centres of adjacent cells. The discrete Laplacian is then:
\begin{equation} 
\label{eq:secondOrderDifference}
	\Lap \Morph_i = \frac{T_1 - T_0}{h} = \frac{\Morph_{i-1} - 2\Morph_i + \Morph_{i+1}}{h^2}.
\end{equation}
This differencing can be represented as a convolution mask during computation, as shown below:
\begin{figure}[!ht]
\centering
\begin{tikzpicture}
\draw[step=1.0cm,color=gray] (0,0) grid (3,1);
\node at (-0.5,+0.5) {$\Lap =\textit{ }$};
\node at (+0.5,+0.5) {1}; %\node at (+0.5, -0.25) {$f_{i-1}$};
\node at (+1.5,+0.5) {-2}; %\node at (+1.5, -0.25) {$f_i$};
\node at (+2.5,+0.5) {1}; %\node at (+2.5, -0.25) {$f_{i+1}$};
\end{tikzpicture}
\end{figure}	
	%\[\frac{\partial f_i}{\partial t} = \frac{\Lap*f}{h^2} \]
\subsubsection*{Diffusion in 2D}
In this case we have two directions to consider. Recall that the Laplacian is the sum of the second derivatives in each principle direction. This allows us to use the summation of the 1D case in both $x$ and $y$ directions, which, for a cell $\Morph_{i,j}$, yields:
\begin{equation} 
\label{eq:gridLaplacian}
	\Lap \Morph_{i,j} = \frac{\Morph_{i,j−1} + \Morph_{i,j+1} + \Morph_{i−1,j} + \Morph_{i+1,j} − 4\Morph_{i,j}}{h^2} .
\end{equation} 
Again, the Laplacian operator can be represented as a convolution mask:

\begin{figure}[H]
\centering
\begin{tikzpicture}
\draw[step=1.0cm,color=gray] (0,0) grid (3,3);
\node at (-0.5,+1.5) {$\Lap =\textit{ }$};
\node at (+0.5,+2.5) {0};
\node at (+1.5,+2.5) {1};
\node at (+2.5,+2.5) {0};
\node at (+0.5,+1.5) {1};
\node at (+1.5,+1.5) {-4};
\node at (+2.5,+1.5) {1};
\node at (+0.5,+0.5) {0};
\node at (+1.5,+0.5) {1};
\node at (+2.5,+0.5) {0};
\end{tikzpicture}
\end{figure}

\subsection{Diffusion on arbitrary triangular meshes}
Triangular meshes are ubiquitous in computer graphics. Meshes allow for a discrete representation of arbitrary surfaces, and there are well-studied algorithms for growing and subdividing them. These properties make meshes a good candidate for reaction-diffusion simulations. As in the grid case, concentrations are associated with cells. These cells are the faces of the mesh dual graph. Concentrations are stored at the vertices because each vertex is located at the centre of its cell. To allow for easy neighbour identification and area calculation, I represent the mesh as a half-edge data structure \citep{marschner2015}.

Two half-edges replace each edge in the mesh to build this data structure. Each half-edge stores a pointer to the next half-edge in the same face, a pointer to the vertex that it originates from, and a pointer to the complementary half-edge (Fig. \ref{fig:halfEdgeMesh}). Each vertex stores a scalar representing the area of the dual cell. This area is the sum of one-third of each adjacent face's area. %Alternate area partitions are given in \citep{Meyer2003}.

\begin{figure}[H]
    \centering
    \incfig{halfEdgeMesh}
    \caption[Two triangles and their half-edge representation denoted by black arrows]{Two triangles and their half-edge representation denoted by black arrows.}
    \label{fig:halfEdgeMesh}
\end{figure}

\subsubsection*{Isotropic diffusion on meshes}
To compute diffusion on an arbitrary triangular mesh, we need a discrete Laplacian. This Laplacian can be a generalization of Eqn. \ref{eq:gridLaplacian}:
\begin{equation}
(\Lap{} \Morph)_i = \frac{1}{A} \sum_{i \sim j} w_{ij}(\Morph_i - \Morph_j).
\end{equation}
This generalized equation states that the Laplacian at vertex $i$ of the scalar field $\Morph$ is the sum of the weighted differences between, $\Morph_i$, the concentration at $i$, and each neighbouring concentrations $\Morph_j$. The weight is $w_{ij}$, the neighbouring edges of vertex $i$ are denoted $i \sim j$, and the area associated with $i$ is $A$. The choice of weight determines the behaviour of the discrete Laplacian. Unfortunately, \citet{wardetzky2007} have shown that no discretization maintains all the properties of the continuous Laplacian. The most common weighting used is the cotangent Laplacian,
\begin{equation}
	(\Lap{} \Morph)_i = \frac{1}{2A} \sum_{i \sim j} (\cot{\alpha} + \cot{\beta)} (\Morph_i - \Morph_j).
	\label{eq:cotanLaplacian}
\end{equation}

A vertex $i$ and a neighbouring vertex $j$ are connected by an edge coloured black in Fig. \ref{fig:dualMesh}. The cotangent weight for the black edge is computed from the angles $\alpha$ and $\beta$, which are opposite from the edge. This weight can be derived from the ratio of the edge length of the dual cell, shown in red, and the black edge's length. The black edge length is inversely proportional to the magnitude of the morphogen gradient between vertices $i$ and $j$. The length of the red edge represents how much of an interface between the area associated with $i$ and the area associated with $j$ exists. Consequently, the amount of diffusion increases when the length of the red edge increases. To find the cotangent Laplacian for the entire mesh, we evaluate Eqn. \ref{eq:cotanLaplacian} at each vertex. A rigorous derivation is given in \citep{crane2013, herholz2013}. The drawback of this Laplacian compared to the continuous version is that the cotangent weights can be negative, which occurs when angles are greater than $90^\circ$. Consequently, care must be taken when meshing the domain.

\begin{figure}[H]
	\centering
	\incfig{dualMesh}
	\caption[A vertex $i$ and its dual area $A$]{A vertex $i$ and its dual area $A$. The weighting for the cotangent Laplacian between vertex $i$ and $j$ is dependant on the length of the red and black edges.}
	\label{fig:dualMesh}
\end{figure}

\subsubsection*{Anisotropic diffusion on meshes}

% 1. Pose the problem first in the context of a figure. 
%		Figure needs diffusion dir
% 2. Be as explanatory as possible
% 3. Introduce significant elements that come into play

Computing anisotropic diffusion on a mesh requires Laplacian weights that reflect the change in diffusivity for a given direction. This change is computed by applying a diffusion tensor, $D$, to the gradient of the morphogen field (Eqn. \ref{eq:ficks2ndlaw0}). The diffusion tensor at a vertex can be visualized as an ellipse (Fig. \ref{fig:anisoMesh}a). The length of the ellipse's axes correspond to the eigenvalues in the diffusivity matrix $\Lambda$ (Eqn. \ref{eq:diffTensor}). To properly orient $\Lambda$, we need a vector denoting the dominant direction of diffusion. For a given vertex, $i$, a vector, $\boldsymbol{\vec{d_i}}$ is specified which provides a notion of direction on the mesh. However, morphogen gradients are defined on triangle faces. To calculate a diffusion direction for a given face, the angle weighted average of its vertex directions, $\boldsymbol{\vec{d_i}}$, $\boldsymbol{\vec{d_j}}$, and $\boldsymbol{\vec{d_k}}$, is computed and projected onto the face to obtain $\boldsymbol{\vec{d}_{ijk}}$ (Fig. \ref{fig:anisoMesh}b). The face normal is used with $\boldsymbol{\vec{d}_{ijk}}$ to obtain a rotation matrix $R$ and subsequently, $D$ (Eqn. \ref{eq:anisoDiffTensor}). 
\begin{figure}[H]
	\centering
	\incfig{aniso_vec2}
	\caption[Visualization of a diffusion tensor at a cell and diffusion direction on a face]{Visualization of a diffusion tensor at a cell and diffusion direction on a face. \textbf{a:} the diffusion tensor at vertex $i$. The green arrow, $\boldsymbol{\vec{d_i}}$, is the primary direction of diffusion and its length corresponds to the first eigenvalue value of $\Lambda$. The red arrow is perpendicular to $\boldsymbol{\vec{d_i}}$ and its length corresponds to the second eigenvalue value of $\Lambda$. \textbf{b:} A triangle with diffusion vectors at each vertex and the resulting face vector.}
	\label{fig:anisoMesh}
\end{figure}
The morphogen gradient on a face is determined by the sum of the face's edge normals. An edge normal provides the gradient direction and is weighted by the morphogen concentration opposite to that edge. To find an edge normal, we apply a $90^\circ$ rotation matrix, $Q$, about the face normal, to the edge. $Q$ is multiplied with $D$ to include this rotation:
\begin{equation} 
	D^\perp = Q^T D Q.
\label{eq:perpTensor}
\end{equation}
Using $D^\perp$ we calculate the Laplacian weighting and obtain the discrete anisotropic Laplacian:
\begin{equation} 
   L_{ij} =
   \begin{cases} 
      -\frac{1}{2A}\sum\limits_{t} \frac{D^\perp e_i}{\norm{e_i}} \cdot \frac{e_i}{\norm{e_i}}(\cot{\alpha}+\cot{\theta})  & \text{if } i = j, \\
      \frac{1}{2A}(\Gamma \frac{\cos{\gamma}}{\sin{\alpha}} + \Xi\frac{\cos{\xi}}{\sin{\beta}})                                     & \text{if } i \sim j, \\
      0 & \text{otherwise.}
   \end{cases}
   \label{eq:anisoLaplacian}
\end{equation}
This matrix is used to compute diffusion when given a vector of morphogen concentrations, $\bm{u}$, by:
\begin{equation}
	L\bm{u} = \Lap \bm{u}.
\end{equation}
This Laplacian is more complicated than before, and a complete derivation is given by \citet{mathieu2014}. The parameters correspond to Fig. \ref{fig:meshLaplacian}. $i$ and $j$ are vertices, and $e_i$ and $e_j$ are edges, as shown in Fig. \ref{fig:meshLaplacian}. $t$ is a triangle that shares vertex $i$. $\Gamma = \frac{\norm{D^\perp e_j}}{\norm{e_j}}$ and $\gamma$ is the angle between $e_j$ and $D^\perp e_j$. If $D^{\perp} = I$ then $\gamma = \alpha$ and we obtain isotropic diffusion. $\Xi$ and $\xi$ are the same quantities measured on the adjacent triangle with that triangle's $D^\perp$. 

\begin{figure}[H]
	\centering
	\incfig{meshLaplacian}
	\caption[Two triangles with their angles and edges associated with the anisotropic cotangent Laplacian]{Two triangles with their angles and edges associated with the anisotropic cotangent Laplacian.}
	\label{fig:meshLaplacian}
\end{figure}

\section{Boundary conditions}
PDEs determine the state of the simulation inside the domain. Interaction between the simulation and the world outside the domain is based on boundary conditions. Dirichlet and Neumann are the two most common boundary conditions. Dirichlet enforces the morphogen values directly at the boundary, and Neumann specifies the rate of change of the morphogens across the boundary. \ProgramName{} uses Neumann with a value of zero by default or Dirichlet if specified.

\section{Systems with dynamic structure}
In nature, domains grow over time, which can have an appreciable effect on pattern formation. One of them is a diluting effect on chemical concentrations that make up a pattern. When simulating directly on meshes, the face's area represents a minimum size of pattern detail that can be represented. Thus, it is important to consider the increase in face size from growth. A subdivision algorithm is used to split large triangles into smaller ones to address this. However, subdividing every face in the mesh causes small triangles to be subdivided unnecessarily, and the total number of triangles increases quickly. This is a problem because the simulation's performance decreases as the number of vertices increases. 

\begin{figure}[H]
	\centering
	\incfig{recursiveSubdiv}
	\caption[Subdividing a face, shown in red, using longest-edge bisection]{Subdividing a face, shown in red, using longest-edge bisection. \textbf{a:} subdivision starts with the red triangle. \textbf{b-d:} the progression of algorithm.} 
	\label{fig:recursiveSubdiv}
\end{figure}

Adaptive subdivision is a technique used to only subdivide triangles with an area larger than a given threshold. This approach allows all triangles to maintain a similar area and limits the unnecessary creation of triangles. The choice of adaptive subdivision algorithm used determines the shape of the generated triangles. If the internal angles of a triangle exhibit large deviations from $60^{\circ}$, it can pose a problem for the simulation. The number and magnitude of deviations in a mesh give an informal notion of mesh quality and affects the simulation. Depending on the angle, the cotangent weights can be negative or widely varying due to the behaviour of cotan around 0 and 180 degrees. Negative cotangent weights cause diffusion to move morphogens from low to high concentrations incorrectly \citep{wardetzky2007}. 

To obtain optimal incremental triangulation that tends to produce good quality triangles, we use longest-edge bisection \citep{rivara1998}. In this method, faces are only ever subdivided with respect to their longest edge. The red triangle in Fig. \ref{fig:recursiveSubdiv}a is too large and must be subdivided. In Fig. \ref{fig:recursiveSubdiv}b, the red triangle has been subdivided, but the new edge does not connect to an edge in the adjacent face. To solve this, we also subdivide the adjacent face along its longest edge. This may fix the issue, or it may cause another triangle to be missing an edge. We proceed by subdividing every face missing an edge in the same way until all the edges are connected. This process is shown in Fig. \ref{fig:recursiveSubdiv} and the algorithm is detailed in Alg. \ref{alg:subdivisionAlgorithm}. 

Subdivision also requires the proper handling of concentrations. The concentration assigned to the new vertex is the average of the neighbouring concentration values that shared the edge.

\begin{algorithm}[!ht]
  \KwIn{Triangle t0}
  \KwResult{Triangle is subdivided along its longest edge. Adjacent triangles are recursively subdivided to share edges.}
  \SetKwFunction{subdivide}{subdivide}
  \SetKwProg{myalg}{}{}{}
  \myalg{\subdivide{Triangle t0}}
  {
   Edge \textit{e0 = getLongestEdge(t0)}\\
   bool \textit{subdividing = hasAdjacentFace(e0)}\\
   \textit{subdivideFace(t0, e0)}\\
   \While{subdividing}
   {
    \eIf{hasAdjacentFace(e0)}
    {
      Triangle \textit{t1 = getAdjacentFace(e0)}\\
      Edge \textit{e1 = getLongestEdge(t1)}\\
      \eIf{e1 == getPairEdge(e0)}
      {
       \textit{subdivideFace(t1, e1)}\\
       \textit{subdividing = false}\\
      }{
        \textit{subdivide(t1)}\\   
      }
    }{
     \textit{subdividing = false}\\
    }
   }
  }
  \textbf{end}
  \caption{An algorithm to recursively subdivide a triangle and its neighbours based on \citep{rivara1998}.}
  \label{alg:subdivisionAlgorithm}
\end{algorithm}

\section{Numerical methods}
The simulation is advanced by taking small steps through time from a given initial condition. I use the forward Euler method to perform this integration. A reaction-diffusion formula, similar to Eqn \ref{eq:basicRD}, integrated with forward Euler is:
\begin{equation}
	\boldsymbol{x_{i+1}} = \boldsymbol{x_{i}} + (D\Lap{}\boldsymbol{x_i} + F(\boldsymbol{x_i}))\Delta t. \\
\end{equation}
Here $\boldsymbol{x_i}$ is a vector of scalars representing morphogen concentrations at step $i$ and $\Delta t$ is the time-step. $D$ is diffusivity, $\Lap{}$ is the discrete Laplacian used to compute diffusion, and $F(\boldsymbol{x_i})$ encapsulates the reactions of the system. This method suffers from inaccuracy at larger time-steps because we are assuming $\Lap{}\boldsymbol{x_i} + F(\boldsymbol{x_i})$ is constant for the whole time-step. This inaccuracy can cause instability in stiff equations by accruing error with each simulation step. Semi-implicit integration schemes \citep{Nie2006} allowing for larger time-steps, but in practice, it is possible to use small enough time-steps to minimize inaccuracy with forward Euler. 
  \chapter{Software design and  implementation}
\section{General requirements} % Requirements
\ProgramName{} was designed to fulfil six main requirements.

\begin{enumerate}
	% 1 support for grids and arbitrary surfaces
	\item To simulate reaction-diffusion on grids and arbitrary triangulated surfaces. Grids provide a common and simple domain for simulating reaction-diffusion. Meshes afford a more realistic domain because they are not restricted to the plane and are better suited for growth. 
	
	% 2 interactivity
	%TODO look for more info on user attention span wrt simulation programs
	\item To allow for interaction with simulation parameters at runtime. Real-time interaction allows for rapid iteration during model creation and pattern exploration. Minimizing delays increases user enjoyment and productivity. %Studies have shown that when using websites, a delay greater than 1 second interrupts the user's flow of thought, and if a delay is greater than 10 seconds, the user will want to do something else \citep{nielsen1994}. 

	% 3 pattern development visualization
	\item To visualize pattern formation over time. Watching a pattern's development allows the user to gain an intuition about simulation behaviour. Also, patterns seen in nature are not necessarily those at a steady state.
	
	% 4 easy to use 
	\item For easy modification of parameters and the equations themselves. Allowing configurable PDEs dramatically increases the usefulness of \ProgramName{} as users can create custom models. Another benefit of configurable PDEs is that not all users may have access to the program's source code. 
		
	% 5 support for advanced features
	\item To support growth, spatially varying parameters, and anisotropic diffusion. These features are essential because their impact on pattern development can lead to more biologically relevant models. 
	
	% 6 tracking model progress
	\item For the ability to track the incremental changes made to models. A track record of a model's changes allows the user to see its developmental process. This history provides a list of the avenues examined during the model's creation and future areas of interest.
\end{enumerate}

\section{Program architecture} 
\ProgramName{} was implemented on the Windows 10 operating system using C++ and OpenGL. The graphical user interface was developed with the library \Quotes{Dear ImGui} \citep{cornut2019}.

\ProgramName{} contains two important abstract classes: \textbf{SimulationDomain} and \textbf{Simulation}. The former is a class which abstracts the concept of a domain. It contains an array of morphogen concentrations and all the domain-specific functionality that can be performed without knowledge of the spatial relationships within the domain. Pure virtual functions such as Laplacian and gradient are declared in \textbf{SimulationDomain}. However, these functions must be implemented by subclasses of \textbf{SimulationDomain}. An example subclass is the \textbf{HalfEdgeMesh} class, which implements the Laplacian function using Eqn. \ref{eq:anisoLaplacian} defined in Chapter 3. The \textbf{SimulationDomain} base class provides an extensible way to add support for other domain types without having to copy non-domain specific code. The reaction-diffusion models are represented by extending the \textbf{Simulation} class. This abstracts and provides all the functionality relevant to the simulation except the PDE formulas. Simulation also contains a \textbf{SimulationDomain} member variable, which is how a reaction-diffusion model is associated with a domain. \textbf{GPUSim} and \textbf{CustomSim} extend \textbf{Simulation} and are the two main classes which provide the link between \ProgramName{} and the Simulation Module. This architecture is shown in Fig. \ref{fig:umlDiagram}.

\begin{figure}[H]
	\centering
	\includegraphics[width=0.7\columnwidth]{ProgramUML.pdf}
	\caption[UML diagram representing \ProgramName{}'s architecture]{UML diagram representing \ProgramName{}'s architecture.}
	\label{fig:umlDiagram}
\end{figure}

% Implementation (Programmer's Perspective)
\section{Simulation creation}
Upon start-up, \ProgramName{} parses user-provided command-line arguments and reads in a parameter file. The possible command line arguments are described in \ref{appendix:CLargs}. The label-value pairs in the parameter file are used to create a symbol table. Then, the window is created along with the camera and scene objects responsible for rendering. Now, an instance of the class SimulationDomain is created. This domain instance is then used to create a Simulation object during which the symbol table is used to construct the Simulation Module as a DLL or compute shader. After creation, the initial conditions, and boundary conditions are applied from the parsed parameters to the simulation. Next, the simulation and domain instances are added to the scene so they can be updated and rendered. The remainder of the simulation executes a while loop that calls the scene's update and render functions. The update function invokes the Simulation Module, which computes the PDEs.

\subsection{GPU acceleration}
Due to the popularity of video games, GPUs have become affordable and included with most computers. They contain many more cores than the CPU, enabling them to perform highly parallelized processing of triangles and pixels at a much faster rate. With the advent of compute shaders in modern graphics APIs, the power of the GPU can be leveraged to perform general-purpose computation. 

In \ProgramName{}, these shaders are written in the OpenGL Shading Language (GLSL). Since the evaluation of reaction-diffusion equations relies on the previous simulation state solely, reaction-diffusion is easily parallelized. A single execution of the shader evaluates the equations once at a single location on the domain. For an entire step of the simulation, the compute shader is executed conceptually in parallel, across all the GPU cores, until all locations have been processed. Representing the half-edge data structure on the GPU can be done the same way as in RAM, by using structs and arrays. In practice, a simulation step will processes groups of cells in parallel. Each group of cells has access to a small amount of memory. This memory is fast to use but may not be large enough to hold our domain, especially if growth is involved. General-purpose GLSL arrays that are not limited in size are named Shader Storage Buffer Objects (SSBOs). SSBOs are slower to use but allow for arrays big enough to hold all the simulation data. To allow for growth, I allocate SSBOs larger than first needed, providing extra capacity for the simulation to grow. In my GPU half-edge data structure, pointers are replaced by index offsets into their respective SSBOs.

A drawback of compute shaders is the slow transfer rate between RAM and GPU memory. This problem arises with frequent or substantial data transfers. The data are usually the result of algorithms that cannot be parallelized and thus are computed with the CPU. The results are then transferred to the GPU for further processing. The reverse case can also occur; results are generated on the GPU and require further computation on the CPU. After the CPU is done, the data may then be transferred back to the GPU. Performance loss due to transferring occurs in \ProgramName{} due to the recursive nature of the subdivision algorithm used. The CPU is used to perform subdivision when a face becomes too large. Then the GPU is updated. Care has been taken to only transfer the changes subdivision has caused, avoiding performance drops. Nevertheless, if significant changes that affect many locations on the domain are made, users may experience a pause as data synchronizes between GPU memory and RAM. 

\section{Parameter file}
When starting the program, the user may specify a configuration file from the command line. If no file is specified, the program will look for \Quotes{SimConfig.txt}. The parameter text file is a customizable model specification (Fig. \ref{fig:paramFileExample}). Text prefixed by hash symbol denotes a comment. It has no effect on the simulation and is used for documentation. A parameter is defined by a label-value pair delimited by a colon. The user can define any number of parameters that do not contain reserved labels. A complete listing of reserved labels is provided in \ref{appendix:Reservedlabels}. The \textit{domain} parameter is how the user specifies the OBJ filename or grid domain. \textit{morphogens} allows the user to specify the names in uppercase, of the morphogens involved in the simulation. These uppercase names are used when defining the initial conditions and PDEs.

User-defined parameters can be declared under the \textit{params} label. Parameters are written inside a pair of curly brackets. The cell indices associated with the parameters are denoted by the \textit{indices} label. In grid-based domains, cells are stored in columns with index 0 representing the bottom left cell. Mesh cells are indexed by the order in which they are declared in the OBJ file. Valid indices are comma-separated integers, a range defined with a hyphen, \Quotes{all}, or \Quotes{boundary}, which only identifies the boundary indices. A random selection of $m$ indices can be requested using $rand(m)$. If the user wishes each index to represent a larger area, an integer radius can be specified with \textit{radius}, which corresponds to an n-ring region around the cell. For a given cell, a 1-ring region corresponds to the eight adjacent cells on a grid and the cells that can be reached within one edge on a mesh. Multiple entries can be made within the curly brackets to define non-homogeneous parameters. This allows for different sets of indices to be associated with different parameter values.

The \textit{initialConditions} parameter specifies the morphogens associated with each index at start-up. Like \textit{params}, the \textit{initialConditions} are declared in a pair of curly brackets. After the indices are defined, the morphogen concentrations are declared. This can be a float or a random selection from a range of values in $[n, m)$, by using $rand(n, m)$, where $n$ and $m$ are floating-point values. The \textit{simFile} label is used to start the model from a saved simulation state. This accepts a filename with a per-index entry for each morphogen concentration, anisotropic vectors, and principle diffusion rates.

The \textit{boundaryConditions} parameter precedes a pair of curly brackets that determine the PDE behaviour at the domain boundary. A set of cell indices is defined. Next, the behaviour for each morphogen is defined. These behaviours only affect the cells defined. The two options are Neumann set to 0 (the default setting) or Dirichlet. 

The \textit{rdModel} parameter can be assigned a value of CPU or GPU depending on the desired computation mode. PDEs are declared within curly brackets that follow the \textit{rdModel} parameter. These equations are declared in either GLSL or C++, depending on the computation mode. In the PDEs, user-defined parameter values are accessed by pre-pending \Quotes{params.} to their name. Current morphogen values are used by referencing their names in lower-case. Equation splitting is used to evaluate diffusion before the PDEs are calculated. An array L is predefined and contains the results of the Laplacian of the current morphogen field. When using the GPU, this array is big enough to hold a value for each morphogen at the current cell, and when using the CPU, it is big enough to hold values for every cell and their morphogens. When defining the PDEs, the user only needs to access L with the upper-case morphogen name. This name is then converted to the proper index when the Simulation Module is created. Similarly, morphogen values are updated by using upper-case morphogen names and writing to a predefined \Quotes{new} array. An example parameter file is shown in Fig. \ref{fig:paramFileExample}.

\begin{figure}[p]
\RestyleAlgo{boxed}
\LinesNotNumbered
\begin{algorithm}[H]
	\# The following parameters correspond to the Gray-Scott model \\	
	domain: icosphere.obj \\
	colorMap: color.map \\
	morphogens: A, S\\
	
	params:\\
	\{\\
\quad indices: all\\
\quad dt: 0.1\\
\quad Da: 0.0005\\
\quad Ds: 0.001\\
\quad f: 0.03\\
\quad k: 0.063\\
	\}\\
	initialConditions:\\
	\{\\
\quad indices: 0-641 \# Example of index range, could also be \Quotes{all}\\
\quad A: 0\\
\quad S: .5 + rand(0, .5)\\
\quad \\
\quad indices: rand(10)\\
\quad radius: 1\\
\quad A: 1\\
\quad S: 1\\
	\}\\
	boundaryConditions:\\
	\{\\
\quad indices: boundary \\
\quad A: dirichlet\\
\quad S: dirichlet\\
	\}\\
	rdModel: GPU\\
	\{\\
\quad float saa = s*a*a;\\
\quad new[A] = a + (params.Da * L[A] + saa - (params.k + params.f) * a) * params.dt;\\
\quad new[S] = s + (params.Ds * L[S] - saa + params.f * (1 - s)) * params.dt;\\
	\}
\end{algorithm}
	\caption[Example parameter file for the Gray-Scott model]{Example parameter file for the Gray-Scott model. The domain referenced is a unit icosahedron OBJ model. The OBJ model has no boundaries. However, boundary conditions are included for completeness.}
	\label{fig:paramFileExample}
\end{figure}

% User's Perspective
\section{User interface}
Exploring different parameter values at runtime is achieved by using graphical control panels shown in Fig. \ref{fig:GUIexample}. It has controls for growth, rendering, and interactive non-homogeneous parameter specification. 

Parameters are shown in textboxes generated from the symbol table. Initially, changes to these parameters affect all indices. For more control, subgroups of indices can be selected through painting. Groups of indices with the same set of parameters are given an ID number and can be selected by cycling through numbers with the \Quotes{Params} textbox. 

When specifying parameters locally on a mesh, the user selects a painting mode from the control panel and right clicks on the mesh to change values stored at vertices within a given radius. When the left mouse button is clicked, and a mode is selected, a sphere is projected onto the mesh using a raycast. The cell closest to the sphere centre is used to find the vertices that are also affected by the painting operation. From the closest cell, subsequent rings are checked to see if they reside in the sphere. This continues until no more vertices are found. There are three painting modes. The first painting mode is \Quotes{selection,} which allows the user to select groups of vertices and set parameters in bulk. The second painting mode is \Quotes{morphogen,} which allows the user to add to or set the morphogen value at the painted cells. The third painting mode is \Quotes{anisotropic diffusion,} which allows the user to paint the direction of diffusion and the principle diffusion rates. The previous raycast location is recorded and used with the current location to determine the direction the cursor is moving when painting anisotropic diffusion (Fig. \ref{fig:painting}). The principle diffusion rates are specified in the control panel. When painting morphogens or diffusion, the quantity at each cell is modified with a linear falloff from the centre of the sphere.

The sphere radius can be adjusted from the control panel or with the mouse scroll wheel. If not painting, the mouse scroll wheel controls the camera zoom. The left mouse button can be used to rotate the model, and right-click translates the model. Domain orientation and position can be reset by pressing the 1 key or through the control panel. The camera can be moved left, right, up, down, in, and out with the W, A, S, D, R, and F keys.

\begin{figure}[ht]
	\centering
	\resizebox{\columnwidth}{!}{
	\includegraphics{gui.png}	
	}
	\caption[The available control panels]{The available control panels. \textbf{GPU:} The control panel for modifying parameters as well as collapsible menus for controlling growth, painting, and rendering behaviour. \textbf{Info:} information about the model such as the PDEs used and selected vertex attributes. \textbf{Controls:} simulation controls allow for saving screenshots and textures. \textbf{Stats:} program statistics are shown such as the time per frame, cell count, and total area of the domain.} 
	\label{fig:GUIexample}
\end{figure}

\begin{figure}[ht]
	\centering
	\includegraphics[width=0.7\columnwidth]{painting.png}	
	\caption[Painting direction is determined by taking the difference in cursor positions from consecutive frames]{Painting direction is determined by taking the difference in cursor positions from consecutive frames. The \textbf{X} inside dashed circle is the initial cursor location, and the \textbf{X} inside the solid circle is the current cursor position. Vertices in the blue area are affected by the paint operation.} 
	\label{fig:painting}
\end{figure}

\section{Visualization}
The user has a choice of visualizing different morphogen concentrations by actual or normalized value. In the latter case, every morphogen concentration is divided by a user-provided value. The concentration is mapped onto a colormap to determine the colour the concentration represents. A separate colormap can be specified for each side of the domain. This feature can be used to hide the pattern on one side of the mesh. Concentration gradients and the vectors driving anisotropic diffusion can also be visualized as lines (representing vectors) extending from their corresponding vertices. These lines are coloured black at their vertex and transition to green for diffusion vectors and red for gradient vectors (Fig. \ref{fig:vector_fields}). Wireframe mode allows users to see the underlying triangular geometry of the domain. The mesh rendering is enhanced by diffuse lighting to highlight its shape. When selected vertices are visualized, the unselected ones will appear faded. Users can export the generated pattern as a texture for higher quality rendering of models through other software. Exporting requires the OBJ model to come with texture coordinates. A blank texture is then coloured using the meshes texture coordinates.

\begin{figure}[ht]
	\centering
	\includegraphics[width=\columnwidth]{vector_fields.png}	
	\caption[Visualization of vector fields]{Visualization of vector fields. \textbf{a}: Pattern gradient is visualized by lines fading from black to red. \textbf{b}: The vector field driving anisotropic diffusion is shown by lines fading from black to green.} 
	\label{fig:vector_fields}
\end{figure}

\section{Saving and loading}
After the simulation has loaded, the user can save the simulation at any point. Saving is used to preserve the state of the simulation after pattern development. Another use for saving is if a specific vector field or morphogen configuration is desired, which otherwise might be tedious to define directly with indices. Saving creates a set of files containing the program and simulation state. These files include the colour map, OBJ file, and the parameter file. Another created file is \Quotes{EditorSettings.txt}, which contains settings not integral to the simulation's behaviour such as background colour and cursor radius. The saved state of the program is in a \Quotes{.rd} textfile. This file contains a header that specifies the number of morphogens on the first line and the number of cells on the second line. After the header, the following n lines correspond to the n cells in the domain. The first line is the 0th indexed cell and contains a list of floating-point values separated by spaces. For a model with morphogens $A$ and $S$, a line would look like:
\[a\text{ }s\text{ }x_a\text{ }y_a\text{ }z_a\text{ }x_s\text{ }y_s\text{ }z_s\text{ }\lambda_{1a}\text{ }\lambda_{2a}\text{ }\lambda_{1s}\text{ }\lambda_{2s}\text{ }d_a\text{ }d_s.\]
Parameters $a$ and $s$ are the morphogen concentrations and entries with subscripts $a$ and $s$ belong to $A$ and $S$ morphogens respectively. $x,\text{ }y, \text{ and } z$ are the components of a cell's anisotropic direction vector. $\lambda_1$ and $\lambda_2$ are the diffusivities and $d$ is the diffusion scale. A user can load a simulation by specifying the individual files to be used in the parameter file. Using a \Quotes{.rd} file overrides the model's initial conditions.

\section{Model exploration}
Designing and exploring a model's pattern forming potential can be challenging. Tracking progress as incremental changes are made is a requirement of model creation with \ProgramName{}. To satisfy this, I used the GIT version control system. I commit the files associated with a model to a repository periodically during exploration. This allows for easy reproduction of previous patterns and avoids duplication of past efforts. The greatest benefit from this work-flow is that the progression through a multi-dimensional parameter space is tracked, providing a mapping of what has previously been explored. Fig. \ref{fig:version_control} shows a visualization of the version control history.

\begin{figure}[ht]
	\centering
	\includegraphics[width=0.7\columnwidth]{version_control.png}	
	\caption[Visualization of the GIT repository by the Sourcetree software]{Visualization of the GIT repository by the Sourcetree software \citep{atlassian2019}. Coloured vertical lines represent models. A dot on the corresponding line represents each model change, and a user-provided comment appears on the right. Models derived from others have their vertical line connected by a horizontal line leading to the parent.} 
	\label{fig:version_control}
\end{figure}
  \chapter{Case Study 1: Ladybug patterns}
\section{Literature review}
The ladybug (also known as ladybird beetle or lady beetle) is an insect in the family Coccinellidae. They are a small round beetles that range in length from 0.8 to 18mm \citep{King1996} and are often found in leaf piles and gardens. Ladybugs exist throughout the world and display a myriad of spot and stripe pigmentation patterns on their elytra: the two symmetric hard shells on the dorsal side of the insect (Fig. \ref{fig:realLadyBugPatterns}). The elytra's main purpose is to protect the fragile wings located underneath, and the pattern is thought to deter predators by indicating that the ladybug is bitter tasting \citep{King1996}. The pattern on one elytron is a mirror image of the pattern on the other. 

\begin{figure}[ht]
	\centering
	\includegraphics[width=\columnwidth]{realLadybeetles.png}
	\caption[A selection of \textit{Harmonia axyridis} ladybugs displaying various spot patterns]{A selection of \textit{H. axyridis} ladybugs displaying various spot patterns. \CpyMsg{Entomart}{2019}.}
	\label{fig:realLadyBugPatterns}
\end{figure}
%\footnotetext{\url{www.entomart.be/INS-0038.html}, Retrieved, September 10, 2019}

Understanding the ladybug life cycle can give insight into how and when their patterns form. The cycle starts with eggs laid on the underside of leaves. These eggs hatch into larvae, which then eat aphids and other food sources until they can pupate and metamorphose into adults. Immediately after pupation, the elytra appear patternless and are a pale-yellow colour. In a timespan of hours to days, dark spots emerge and become black. The yellow transitions to red, giving the ladybug its characteristic appearance. Although certain species of ladybugs are described by the number of spots on their elytra, there can be a variable number and shape of spots found on ladybugs of the same species.

\citet{Ando2018} explored the genetic mechanisms governing the formation of ladybug patterns. They found a gene, \textit{pannier}, which is responsible for much of the observed pigmentation patterns in \textit{H. axyridis} and \textit{Coccinella septempunctata}. Before any pigment is visible, \textit{pannier} is found in a pre-pattern on the elytra. It then promotes melanin (black) and inhibits carotenoids (red) pigmentation accumulation creating a visible pattern. Future work is needed to identify if other specific genes are involved in pigmentation expression.

Although the specific genetic mechanisms behind pattern formation are not fully known, \citet{liaw2001} used reaction-diffusion to simulate visually similar ladybug patterns. The equations used are activator-depleted substrate (Eqn. \ref{eq:activatorSubstrateRD}) with saturation. Their results were obtained numerically by using forward Euler integration on a grid located on the surface of a partial sphere. A Laplacian defined in spherical coordinates was used to compute diffusion on partial hemispheres of radius 1. It was found that the domain boundary and the curvature change the final position of spots and stripes. Five models were proposed. Three species considered display black spots on a red/orange background. In particular, \textit{Platynaspidius quinquepunctatus}, \textit{Coccinella septempunctata}, and \textit{Epilachna crassimala} display 5, 7, and 10 spots respectively. \textit{Macroilleis hauseri} has brown stripes aligned with the long axis of the shell on a yellow background. \textit{Bothrocalvia albolineata} displays elongated orange loops on a brown background.

\section{Model description} 
I used \ProgramName{} to improve these models. The equations remain the activator-depleted substrate formula \citep{meinhardt1982}:
\begin{equation}
	\begin{aligned} \label{eq:ladybugRD}
   \frac{\partial u}{\partial t} &= \rho_u \frac{u^2v}{1+\kappa u^2} + \sigma_u - \mu_u u + D_u \Lap u, \\
   \frac{\partial v}{\partial t} &= -\rho_v \frac{u^2v}{1+\kappa u^2} + \sigma_v + D_v \Lap v.
	\end{aligned}
\end{equation}

The activator is represented by $u$ and is displayed as the pigmentation on the elytra. The substrate is denoted by $v$. The rate of conversion of $v$ into $u$ is determined by the reaction rate $\rho_u$. Similarly, $\rho_v$ represents how much $v$ is used in the conversion to $u$. $\kappa$ controls saturation, $\sigma_u$ and $\sigma_v$ are the base production rates, and $\mu_u$ is the decay of $u$. $\Lap$ is the discrete Laplacian with the diffusion rates being $D_u$ and $D_v$ for $u$ and $v$. The boundary conditions are no-flux, except for the model of  \textit{B. albolineata}, which contains a sink along the middle of the domain where the two shell halves would meet.

\section{Results}
By simulating ladybug patterns on a mesh, I improved on the results in \citep{liaw2001}. Due to the mesh's flexible representation of 2D surfaces, my model provides a more faithful portrayal of ladybug elytra compared to a partial sphere (Fig. \ref{fig:ladyBugPatterns}). A couple of aspects of patterning contributed to the quality of the results. The pattern shape, such as spots or stripes, was essential, and the previous work provided this. Pattern colouration was important as I observed a substantial qualitative increase in patten realism by using images of real ladybug specimens to determine the pattern colours. The pattern's positioning on the mesh, especially for spots, was a necessary pattern feature. It was difficult to predict the final spot positions before the simulation reached a steady-state. Final spot locations often varied based on small changes to the specific initial distribution of morphogens. Being able to run the models quickly helped when picking a morphogen distribution. Fig. \ref{fig:ladyBugDev} shows the development of the simulated patterns. During the simulation of spot patterns, the initial pattern will disappear quickly and then reappear as large blotches. The blotches settle split into smaller ovals and then settle as spots. The original parameters provided by \citet{liaw2001} have been altered to be more mathematically sound, and to account for the differences when simulated on a mesh. A complete list of model parameters is shown in Table \ref{tab:ladyBugParameters}. As noted by \citet{liaw2001}, a starting band of morphogens across the top of the elytra is important for spot alignment. In C, I also used a stripe at the top but ignored the central stripe. This alternative starting pattern produced rows of spots more predictably as individual spots migrated less during pattern formation. I also produced patterns on elytra without the no-flux boundary in the middle of the elytra, and this produced the same patterns. This is probably due to the high amount of symmetry of the domain and starting pattern. A ladybug rendering is shown in Fig. \ref{fig:ladybugRender}.

\begin{figure}[ht]
	\centering
	\includegraphics[width=\columnwidth]{ladybugs.png}
	\caption[Simulation of ladybug patterns]{Simulation of ladybug patterns. \textbf{Row 1:} initial simulation state. \textbf{Row 2:} final simulation state. \textbf{Row 3:} collection of the real ladybug species. \textbf{A-E:} \textit{P. quinquepunctatus}, \textit{C. septempunctata}, \textit{E. crassimala}, \textit{M. hauseri}, and \textit{B. albolineata} respectively. 
	 \textit{Photographs: \citep{chen2008}}.}
	\label{fig:ladyBugPatterns}
\end{figure}

\begin{figure}[ht]
	\centering
	\includegraphics[width=0.75\columnwidth]{ladybug-dev.png}
	\caption[Progression of ladybug patterns over time]{Progression of ladybug patterns over time. \textbf{A-E:} \textit{P. quinquepunctatus}, \textit{C. septempunctata}, \textit{E. crassimala}, \textit{M. hauseri}, and \textit{B. albolineata} respectively.}
	\label{fig:ladyBugDev}
\end{figure}

\newcommand{\orig}[1]{\textcolor{gray}{(#1)}}
\begin{table}[ht]
	\resizebox{\columnwidth}{!}{
	\centering
	\begin{tabular}{llllll}
	\hline
	\textbf{Model} & \textbf{Species} & \bm{$D_u$} & \bm{$D_v$} & \bm{$\kappa$} & \bm{$\sigma_u$} \\ \hline 
	A & \textit{P. quinquepunctatus} & 0.0005 & 0.035 & 0 & 0 \\ 
	B & \textit{C. septempunctata} & 0.0005 & 0.025 & 0 & 0 \\ 
	C & \textit{E. crassimala} & 0.0003 & 0.024 \orig{0.04} & 0 & 0.01 \orig{0} \\ 
	D & \textit{M. hauseri} & 0.000028 & 0.00168 \orig{0.000168} & 0.5 \orig{0.35} & 0 \\ 
	E & \textit{B. albolineata} & 0.000026 & 0.00182 \orig{0.000182} & 0.45 & 0.0019 \\
	\hline
	\end{tabular}
	}
	\caption [Parameter values used for ladybug models on a mesh]{Parameter values used for ladybug models on a mesh. The following parameters remain constant for all models $dt = 0.001$, $\sigma_v= 0.1$, $\rho_u = 0.18$, $\rho_v = 0.36$, and $\mu_u = 0.08$. The total number of steps is $1,500,000$ for all models except E where it has been decreased to $500,000$. Parameters in parenthesis are the original values used by \citep{liaw2001}. Parameter values were changed due to small differences between the patterns formed on a partial sphere and a mesh. In my model of D and E, the parameter $D_v$ has been increased by an order of magnitude to obey the rule that $D_v / D_u \geq 7.8$ \citep{liaw2001}. The number of spots on C (\textit{E. crassimala}) was less than ten. To rectify this, I lowered $D_v$ from $0.04$ to $0.024$ and changed the initial morphogen distribution to a stripe at the top. $\sigma_u$ was then increased from $0$ to $0.01$, allowing for horizontal lines to form which turn into spots over time. Another discrepancy observed with the initial parameters was that the stripes in D and E turned into spots and irregular lines near the boundary. In D, the initial morphogens propagate as a wave, leaving stripes in its wake. On a mesh, the wave was observed to outpace itself in places, causing it to self-interact and destroy the vertical line pattern. I have increased $\kappa$ from $0.35$ to $0.5$, strengthening the tendency to form lines. I also changed the initial distribution from a vertical stripe of $u$ down the centre to include stripes along the boundary (excluding the top). This has the effect of aligning the pattern by reducing the distance the middle wave must travel and avoiding the self-interaction. The total simulation steps was reduced to 500,000 from the original model's 1,500,000 steps to account for lines becoming spots in E. This could also be addressed by decreasing $\sigma_u$, which produces a pattern like D.}
	\label{tab:ladyBugParameters}
\end{table}

\section{Discussion and future work}
Improvements to these models can be made once the biological chemical interactions are fully understood. The most relevant insights would be the real initial distribution of morphogens, the actual reaction behaviour between morphogens, and the rate at which they diffuse throughout the elytra. Some ladybug species can control their pigmentation patterns depending on the season or the surface the ladybug is on. This suggests a simple reaction-diffusion model cannot represent all types of ladybug patterns \citep{insects2009}.

Further investigation should be made to determine how much the patterns found on the head of the ladybug effect the elytra pattern. Another critical question is if the patterns displayed are at a biological steady-state or does pattern formation stop prematurely. Development of the ladybug patterns over time is less studied, and it would be interesting to see if there exists a real chemical pre-pattern that moves like the models.

%Both my model and \citep{liaw2001} use a morphogen sink along the centre of the elytra to create the loop pattern. Patterns observed in D seem to be a good fit for modelling E as well.

\begin{figure}[p]
	\centering
	\includegraphics[width=\columnwidth]{ladybug.png}
	\caption[A rendering of two ladybugs on a leaf]{A rendering of two ladybugs on a leaf.}
	\label{fig:ladybugRender}
\end{figure}
  \chapter{Case Study 2: Snakes}
\section{Biological background}
Snakes display many pigmentation patterns such as speckles, blotches, longitudinal stripes, and transverse stripes. Some of these pattern types are shown in Fig. \ref{fig:realSnakePatterns}. Snakes are entirely covered by protective scales whose top layer is made of translucent keratin. Under this layer is where pigmentation patterns are found. Although the keratin layer is translucent, it may contribute to the pattern's appearance by acting as prisms through which light refracts, producing an iridescent sheen. Scales contribute to pattern formation in other reptiles by limiting diffusion of pigments across the scale boundaries \citep{manukyan2017}.

Pigmentation patterns perform many useful functions, such as camouflaging the snake from prey and predators or acting as a brightly coloured warning signal that the snake may be venomous. An example warning pattern is seen on the North American coral snake, \textit{Micrurus fulvius}, with its distinctive red, yellow, and black coloured bands (Fig. \ref{fig:realSnakePatterns}a). Some non-venomous snakes employ mimicry, which serves as a defence mechanism by displaying a pattern similar to that of a venomous snake's. A well-known example of this is the scarlet king snake, \textit{Lampropeltis elapsoides}, which also displays red, black and yellow bands like \textit{M. fulvius} (Fig. \ref{fig:realSnakePatterns}b).

The regions of a snake that display pigmentation patterns can be roughly partitioned into the head, body, tail, and underbelly. Here I focus on the most striking patterns found on the body and tail. The similarity between patterns found in these regions and those found on the head or underbelly varies between species. It is not uncommon for the head and underbelly to display a different pattern than the body. A notable example is the ring-necked snake, shown in Fig. \ref{fig:realSnakePatterns}c, which presents a dull colour on its back, serving as camouflage. When the ring-necked snake is provoked, it will display its bright red and orange underside to deter predators.

\begin{figure}[hb]
	\centering
	\includegraphics[width=\columnwidth]{snakes.png}
	\caption{A collection of snakes with interesting patterns. \textbf{a:} An American coral snake, \textit{M. fulvius}. \textbf{b:} A scarlet kingsnake, \textit{L. elapsoides}. \textbf{c:} The tail of a ring-necked snake. \textbf{d:} A python. \textbf{e:} A corn snake. \textit{Photographs \textbf{a}: Peter Paplanus (CC BY-NC 2.0); \textbf{b}: courtesy of Trent Adamson; \textbf{c}: Glenn Bartolotti (CC BY-SA 3.0); \textbf{d,e}: courtesy of pixabay.com.}}
	\label{fig:realSnakePatterns}
\end{figure}

Snake patterns start development when the animal still in its egg. \citet{murakami2018} studied the early development of pigmentation patterns in the Japanese four-lined snake (\textit{Elaphe quadrivirgata}). The characteristic pattern \textit{E. quadrivirgata} displays is four black lateral lines on a light brown background (Fig. \ref{fig:Snake1}c). In juveniles, these stripes are dark brown and connected by transverse lines like a ladder. These transverse liens disappear in mature snakes.
 
To understand how the patterns of \textit{E. quadrivirgata} develop, snakes were observed at different points in time during embryonic development. Pigmentation is measured by observing chromatophore cells, which contain pigmentation molecules. A type of chromatophore, called a melanophore, is responsible for brown to black colouration and is first seen in deeper tissues of \textit{E. quadrivirgata} around 18-21 days after ovulation. These cells do not affect the later development of melanophores seen in the dermal and epidermal layers. Striped pigmentation patterns emerge behind the head after 26-32 days. Stripes then appear on the whole body, first thinly at 29-35 days and then clearly at 35-42 days. Data about how patterns emerge and develop are valuable because they contribute to an understanding of the patterning process and provide testable data points to guide the model's development.

\citet{allen2013} studied the role behaviour, and ecological factors played in snake pattern diversity by using reaction-diffusion. They classified these patterns by simulating reaction-diffusion with a custom-built program and having users compare the simulated patterns against real snake images. The match between a simulation and real snake image provided an association between the reaction-diffusion model and a real pattern. Pattern features such as size, complexity, and anisotropy were then represented as parameters.

To gain insight into how patterns are related the behaviour and environment of snakes, these parameters were associated with ecological and behavioural variables. Examples of ecological variables are those related to habitat like Desert or forest. Behavioural variables correspond to speed, aggression, and hunting strategy. Phylogenetic analysis then revealed to what extent the environment and or snake behaviour was responsible for the diversity of patterns. The results of this analysis suggested snake patterns are mainly correlated with behaviour rather than the environment the snakes inhabited. \citet{allen2013} found that plain longitudinally- striped snakes are usually smaller and prefer to flee from predators. The striped pattern makes the snake harder to track by sight while moving. Transverse striped and blotched snakes are often larger and more aggressive. They may also be more venomous and hunt by ambush.

\section{Previous modelling work}
Patterns on the ocellated lizard\footnote{Although lizards are of a different suborder than snakes, they both reside in the squamate reptile family because of their scaled bodies.} have been simulated by \citet{manukyan2017}. These lizards are covered in a quasi-hexagonal lattice of pigmented scales. A juvenile lizard's scales display white spots on a brown background. This pattern changes in adults whose scales are coloured individually, either a solid green or black. Pattern development continues over the lifetime of the adult, with scales switching between green and black. \citet{manukyan2017} modelled the adult pattern using a reaction-diffusion system on a grid where multiple cells represented one scale. They lowered the diffusion rates across the scale boundaries simulating thinner skin. Consequently, the diffusion between scales is much slower than inside an individual scale. This reaction-diffusion system behaves like a cellular automaton.

\citet{murray1991} modelled snake patterns by simulating the movement and interaction of chromatophores. Before chromatophores differentiate, they exist as chromatoblasts, which are found uniformly in the dermis. After some time, these cells may become chromatophores by producing pigments, resulting in a visible pattern. Movement of these cells is driven by diffusion and chemotaxis. The use of chemotaxis makes it possible to generate simple and more complex patterns when calculated on a growing domain. Although \citet{murray1991} proposed that standard reaction-diffusion models may also produce the same patterns. \citet{murray1991} produced more intricate patterns such as staggered and side-by-side spots as well as diamond-shaped patterns by growing the domain. Simulations are carried out on a grid, and only the patterns on the body of the snake are considered.

\citet{pinheiro2017} created a program for modelling snake patterns by using a combination of textures, cellular automata, and image manipulation. To simulate a transverse stripe pattern, a modeller can define up to four different coloured bands of various thicknesses can be combined. Similarly, for longitudinal stripes, the modeller defines the number and colour of the stripes can be customized. To simulate spots, circle textures are randomly distorted in size and position. For other simple patterns, cellular automata are used to generate blotch or zigzag patterns. Generated patterns are unnaturally uniform, so they are distorted using Perlin noise to look more organic. This results in a generated texture that is rendered on a snake mesh and enhanced by using a roughness and height map to provide a scaly appearance. This approach produces compelling results, although \citet{pinheiro2017} notes that more complex patterns need a phenomenon such as chemotaxis, which involves simulating chemicals moving up concentration gradients.

\section{Model description}
I have produced snake patterns using reaction-diffusion on a mesh representing the snake's skin. Most of my models use different parameters for the ventral scales, as there often is a different pattern found there. Snake meshes have been modelled and rendered using the 3D computer graphics software Blender. A normal map is used to add a scaly appearance, and two black spheres are used to represent snake eyes. Models A-E use Gray-Scott reaction-diffusion equations, Eqn. \ref{eq:grayscottRD}, and the parameter values are found in Table \ref{tab:gsSnakeParameters}. Model F uses the activator-depleted substrate model with saturation, Eqn. \ref{eq:ladybugRD}, and its parameters are found in Table \ref{tab:snakeParameters}. Models B, D, and F assume anisotropic diffusion. The vector field used runs parallel to the snake's longitudinal axis. The coefficient $\lambda_{1}$ is the diffusivity in the direction of the vector field, and $\lambda_{2}$ is the diffusivity orthogonal to it (Eq. \ref{eq:diffTensor}). 

\newpage
% first model
The first snake model displays brightly coloured transverse stripes like those of the Honduran milk snake (\textit{Lampropeltis triangulum hondurensis}). This species is non-venomous but appears similar to other venomous snakes. The initial distribution of morphogens and the final pattern are shown in Fig. \ref{fig:Snake1}.

\begin{figure}[ht]
	\centering
	\includegraphics[width=\columnwidth]{Snake1.png}
	\caption{Model of the Honduran milk snake. The activator, $a$, is visualized where concentration values from low to high are represented as dark orange, black, and bright orange. \textbf{a:} The initial state: $a=1$ on the nose and $0$ elsewhere, $s=1$ everywhere. \textbf{b:} The final pattern. \textbf{c:} An image of a real Honduran milk snake. \textit{Photograph: courtesy of Douglas Mong}}
	\label{fig:Snake1}
\end{figure}

% second model
The next snake model displays a pattern consisting of four black lateral stripes, similar to those of an adult \textit{E. quadrivirgata}. I have assumed anisotropic diffusion of the activator $a$, as described in Eqn. \ref{eq:ficks2ndlaw0}, of $a$. The vector field used by $a$ runs parallel to the snake's longitudinal axis. The diffusivity coefficients are $\lambda_{1}=1$ and $\lambda_{2}=0.502$. Anisotropic diffusion with $\lambda_{1} > \lambda_{2}$ was important because the lines should form parallel to the longitudinal axis of the snake. The ventral scale's parameters vary from those used on the dorsal side by setting $f=0$ so that extra lines do not form on the underside. The initial and final patterns are shown in Fig. \ref{fig:Snake2}.

\begin{figure}[ht]
	\centering
	\includegraphics[width=\columnwidth]{Snake2.png}
	\caption{Model of the \textit{E. quadrivirgata}. The activator, $a$, is visualized where concentration values from low to high are represented as brown to black. \textbf{a:} The initial morphogen distribution is $s=1$ and $a=0$ except a stripe of $a=1$ down the dorsal side of the snake. \textbf{b:} The final pattern. \textbf{c:} An image of a real \textit{E. quadrivirgata}. \textit{Photograph: courtesy of Anthony von Plettenberg Laing}}
	\label{fig:Snake2}
\end{figure}

\newpage
% third model - 11028.041 to 44303.008 area
The third model I made displays a spot or blotch pattern like that of the spotted rock snake (\textit{Lamprophis guttatus}). %TODO why was this interesting
The model includes growth and adaptive subdivision. There are two phases to pattern formation. In phase 1, the simulation is initialized from a random placement of morphogens. After initialization, a row of spots forms on the snake. Phase 1 lasts for 100,000 iterations, after which phase 2 begins. The snake is grown uniformly over 70,000 iterations increasing its surface area by four times. Faces of the mesh start with an average area of $0.52$ and are subdivided when they exceed the max face area of $1$. This growth allows the spot pattern to form into two rows of spots. Fig. \ref{fig:Snake3} shows this model.

\begin{figure}[ht]
	\centering
	\includegraphics[width=\columnwidth]{Snake3.png}
	\caption{Model of the spotted rock snake. \textbf{a:} Start of phase 1 showing the initial morphogen distribution of $s=1$ and $a=0$ except for 30 randomly placed spots where $a=1$. \textbf{b:} A pattern of spots has formed along the snake. This is the end of phase 1 and the start of phase 2. \textbf{c:} End of phase 2 where rows of spots have formed on an enlarged snake. \textbf{d:} An image of a real spotted rock snake. \textit{Photograph: courtesy of Tyrone Ping}}
	\label{fig:Snake3}
\end{figure}

\newpage
%Fourth model
The common European viper, \textit{Vipera berus}, displays an interesting zigzag pattern. This snake is venomous, and the pattern can serve as a warning signal or as camouflage when the snake is tightly coiled \citep{lillywhite2014}. Morphogen $a$ diffuses anisotropically using coefficients $\lambda_{1}=0.81$, $\lambda_{2}=1$. This model is shown in Fig. \ref{fig:Snake4}.

\begin{figure}[ht]
	\centering
	\includegraphics[width=\columnwidth]{Snake4.png}
	\caption{Model of the European viper. \textbf{a:} The initial distribution of $a$ is randomly chosen from $[.5, 1)$ for each vertex and $s=1$ everywhere. The concentration of $a$ seen is multiplied by $0.263$ to show the character of the start state pattern. The colour map expects values in the range $[0-1]$. \textbf{b:} The final pattern. In this case $a$ multiplied by $0.556$. \textbf{c:} An image of a real \textit{V. berus}. \textit{Photograph: Benny Trapp (CC BY 3.0)}}
	\label{fig:Snake4}
\end{figure}

\newpage
%Fifth model
Another snake model is based on the transverse stripes of a southern coral snake (\textit{Micrurus frontalis}). The stripes are brightly coloured and are a warning to others that this snake contains a potent venom. There are two phases: pattern establishment and mesh growth. The mesh grows uniformly over $23,000$ iterations causing the snake's surface area to double. Mesh faces start with an average area of $0.55$ and the faces are subdivided when they exceed an area of $1$. The snake mesh used is shown in Fig. \ref{fig:Snake5}.

\begin{figure}[ht]
	\centering
	\includegraphics[width=\columnwidth]{Snake5.png}
	\caption{Model of the \textit{M. frontalis}. \textbf{a:} Initially $s=1$ and $a=0$ everywhere except on the nose, where $a=1$. $a$ is visualized from red to black to white and is normalized by $0.520$. \textbf{b:} end of phase 1 where a basic stripe pattern has formed. \textbf{c:} The final pattern after growth. Black stripes have appeared in-between the previous stripes. \textbf{d:} A picture of a real \textit{M. frontalis}. \textit{Photograph: William Quatman (CC BY-SA 2.0)}}
	\label{fig:Snake5}
\end{figure}

\newpage 

The California kingsnake (\textit{Lampropeltis californiae}) contains white and black transverse stripes, which sometimes bifurcate. I have assumed anisotropic diffusion on the snake's body. Morphogen $u$ assumes diffusivity coefficients $\lambda_{1}=0.75$, $\lambda_{2}=1$, and $v$ assumes $\lambda_{1}=1$, $\lambda_{2}=0.75$. Anisotropy was needed to align the stripes perpendicular to the body strongly. The snake's head assumes standard isotropic diffusion as the real snake does not have stripes on its head. This model is shown in Fig. \ref{fig:Snake6}.

\begin{figure}[ht]
	\centering
	\includegraphics[width=\columnwidth]{Snake6.png}
	\caption{Model of the \textit{L. californiae}. Morphogen $u$ is visualized as white when its value is low and black when high. \textbf{a:} The initial distribution is $u=0$ and $v=1$ everywhere except a lateral stripe  of $u=1$. \textbf{b:} The final pattern. \textbf{c:} An image of a real \textit{L. californiae}. \textit{Photograph: courtesy of David Steen}}
	\label{fig:Snake6}
\end{figure}

\newpage

\section{Discussion and future work}
Snake patterns provide an interesting modelling challenge due to the broad diversity of patterns on a geometrically simple domain. I have produced a variety of patterns based on real snake species. By using the features of \ProgramName{}, such as the simulation of growth and anisotropic diffusion, I generated relatively complex pigmentation patterns. My previous models of ladybugs and research into the parameter space of the Gray-Scott model (Eqn. \ref{eq:grayscottRD}) helped me identify iconic features of reaction-diffusion patterns that were also seen on snakes. Most of the models produced the same type of pattern on a straight and curved domain. However, simulating model E on a curved domain caused the stripes to split, creating an incorrect forked pattern. Likewise, simulating model F on a straight domain did not allow the stripes to fork correctly. This behaviour begs the question as to what pose a snake assumes throughout its pattern's development. As stated by \citet{murray1991}, reaction-diffusion without the effect of chemotaxis is expressive enough to produce some snake patterns. As illustrated in Fig. \ref{fig:SnakeRendering}, reaction-diffusion models can be used to create convincing biological patterns for use in computer graphics. Future work may provide an understanding of the role of scales in pattern formation. Models that account for the effect of scale boundaries on diffusion may produce the more delicate details seen in nature, such as the pigmented tips of scales. 

One noticeable limitation of the models is seen on the ends of the tails. Stationary reaction-diffusion patterns are frequently standing waves with a fixed wavelength. Thus, the patterns can only fit on a domain if there is enough space for them. A  fixed wavelength poses a problem because the end of the tail runs out of space to support the same pattern seen on the body. In nature, the patterns tend to scale down to account for the tail tapering. However, gradually reducing the diffusion rate based on the proximity to the end of the tail might be a solution.

Snake pattern formation is of great interest to herpetologists, who study reptiles and amphibians. They can understand snake evolution through the myriad of patterns snakes display. Snake breeders profit off selling snakes that display striking and unique patterns. Consequently, there is also a financial incentive to predict the effect of breeding on pattern development.


% snake on ground with some iridescence
\begin{figure}[ht]
	\centering
	\includegraphics[width=\columnwidth]{ground_1.png}
	\caption{A rendering of \textit{E. quadrivirgata} with iridescence.}
	\label{fig:SnakeRendering}
\end{figure}

\begin{table}[ht]
	\centering
	\resizebox{\columnwidth}{!}{
	\begin{tabular}{llllllll}
	\hline
	\textbf{Model} & \textbf{Species} &\bm{$D_a$} &\bm{$D_s$} &\bm{$f$}    &\bm{$k$}   &\bm{$dt$} & \thead{\textbf{Total steps} \\ \textbf{(x1000)}}\\ \hline 
	A     &\textit{L. triangulum hondurensis} &1.000 &2.000 &0.026 &0.055 &0.030 &200 \\ 
	B     &\textit{E. quadrivirgata} &0.175 &0.350 &0.078 &0.061 &0.030 &28 \\ 
	C     &\textit{L. guttatus} &0.350 &0.700 &0.022 &0.022 &0.100 &100; 70 \\ 
	D     &\textit{V. berus} &0.150 &0.300 &0.109 &0.053 &0.300 &40 \\ 
	E     &\textit{M. frontalis} &1.000 &2.000 &0.034 &0.057 &0.030 &200; 23 \\ 
	\hline \end{tabular}
	}
	\caption {Parameters for models A-E using the Gray-Scott equations (Eqn. \ref{eq:grayscottRD}). In B and C the ventral scales have $f=0$. In D the ventral scales have $f=0$ and $k=0.08$. In the \Quotes{Total steps (x1000)} column, values separated by a semicolon denote a multi-phase model with the first and second values representing phase 1 and 2 respectively.}
	\label{tab:gsSnakeParameters}
\end{table}

\begin{table}[ht]
	\centering
	\resizebox{\columnwidth}{!}{
	\begin{tabular}{llllllllllll}
	\hline
	\textbf{Model} & \textbf{Species} & \bm{$D_u$} &\bm{$D_v$} &\bm{$\kappa$} &\bm{$\rho_u$} &\bm{$\rho_v$} &$\bm{\sigma_u$} &\bm{$\sigma_v$} &\bm{$\mu_u$} &\bm{$dt$} &\thead{\textbf{Total steps} \\ \textbf{(x1000)}}\\ \hline 
	F     &\textit{L. californiae} &0.056 &3.36 &0.5 &0.18 &0.36 &0.001 &0.1 &0.08 &0.01 &70                \\ \hline
	\end{tabular}
	}
	\caption {Parameter values for the California kingsnake model. This model uses the activator-depleted substrate equations (Eqn. \ref{eq:grayscottRD}).}
	\label{tab:snakeParameters}
\end{table}

  \chapter{Case Study 3: Flowers Petal Patterns}

\section{Literature Review}
To humans, flowers are a symbol of natural beauty and their spots and stripes serve to enhance this quality. A variety of these flower patterns are shown in Fig. \ref{fig:realFlowers}. For nature, flower pigment patterns have a more  utilitarian purpose. They are used to help flowers reproduce by attracting insects and likely evolved to exploit pollinator vision to further this goal. Patterns guide insects to nectar and can appear as paths or landing spots, sometimes visible only in the UV spectrum \citep{davies2012}. Patterns may also develop due to infections, age, and the environment \citep{davies2012, robinson2015}. Thus, understanding them also gives insight into flower reproduction, insect behaviour, and environmental factors. Some patterns mimic the appearance of female bugs, as in the case of the bee orchid \citep{vereecken7484}. This deceives bees looking for mates, into pollinating the flower. Surprisingly, given the importance of petal patterns to nature and the ubiquity of flowers in movies and video games, little attention has been given to simulating them. 

\begin{figure}[!ht]
	\centering
	\includegraphics[width=\columnwidth]{flowers}
	\caption{Examples of pigment patterns on real flowers. \textit{Photographs: courtesy of Przemyslaw Prusinkiewicz}}
	\label{fig:realFlowers}
\end{figure}
A clonal mosaic model was used to procedurally generate patterns found in the plant kingdom \citep{binsfeld2011}. Specifically, they produce spot patterns found on mushrooms, the stripes on watermelons, and the blotches on plants such as the shrub \textit{Codiaeum variegatum}. In this model, individual plant cells are represented discretely and are separated into two types, foreground and background. Each type contains its own set of parameters that consist of colour, division rate, cell mobility, adhesion between cell types, and chance to switch types. Cells are represented as randomly distributed points over a user defined mesh. The type of each cell may also be random, or user defined. During simulation, a given cell will produce a child cell based on the parent's division rate. This child cell may inherit its parent's type or change types based on its parent's probability to change types. These cells repel each other and spread out over the domain. For a given cell, the strength of this repulsion is based on the distance between cells, and the number of cells surrounding it, as well as the cell adhesion factor between cell types. Cell division and movement is repeatedly calculated, simulating plant aging and producing a uniformly covered domain. To render the result, the cells are triangulated to create a new mesh representation of the original domain. 

A simulation of procedurally generated two dimensional flowers is proposed by \citep{risi2012}. In this model a flower is represented by a specialized artificial neural network which encodes the flower shape and colour. This network is called a compositional pattern-producing network (CPPN) \citep{stanley2007}. It uses a wider range of activation functions compared to a standard neural network to produce symmetric and repeating patterns, which are features often seen in natural flower petals. To create a flower, the CPPN takes as input an angle, $\Theta$, a distance from the centre of a circle, $r$, and a layer number, $L$. It then returns a distance value and a pigment colour. The distance is used to define the perimeter of the flower. To determine the colour of the petals, the CPPN is queried for locations on the petal based on their angle and distance from the centre. This process is repeated for multiple layers and composited together to create more complex flowers. 

This simulation was implemented in the video game Petalz which is based on procedurally generating and sharing flowers of different shapes and colours. Users are given pre-made starting flowers which they can then breed, display, and sell. Breeding is accomplished through mutating a single flower or cross-pollinating different flowers together. Specifically, the nodes and connections of the CPPN are mixed with another network. This results in a combination of each flower's colours and shape. Selling flowers gives the user in-game currency which is used to buy flowers from other users which can then be cross-bred to create new and unique flowers. The social aspect of this game leads to a crowd sourced method of flower creation where the flower attributes are selected for based on their visual appeal.

Another mathematical approach to flower modelling is proposed in \citep{lu2014}. This work focuses on modelling flower petal patterns as well as geometric components such as the pistil, stamen, and receptacle in 3-dimensions. These flower components are modelled with parametric ellipsoids and cylinders and the petal geometry is modelled by deforming rectangular surfaces. Pigment intensity is determined from various combinations of sine functions. The distance from the centre of the flower is used as the argument of these functions and the resulting values correspond to pigment intensity. This approach provides visually good results, but it does not seem to support irregular or complex patterns.

Reaction-diffusion has been used to simulate petal patterns on a grid \citep{zhou2007}. In this model, petal shape, initial morphogen distribution, and a venation map are input as textures. The reaction-diffusion equations used consist of Schnakenberg kinetics with diffusion \citep{schnakenberg1979}. The diffusion rate at a given point is determined by the distance to a vein. This distance is found by querying the venation map. The reaction-diffusion equations are then simulated to generate a final pigment distribution. The colour is then determined by mapping the concentration to a user provided colormap.

\citep{Yuan2019} used a biologically motivated approach to analyse pattern formation on a flower \textit{Mimulus} (monkeyflower). This flower contains red spots located where the flower petals converge into semi-tubular furrows as shown in Fig. \ref{fig:monkeyflower_real}. The red pigmentation consists of anthocyanin, molecular compounds that appear red, blue or orange \citep{bayer1966}. Through the exploration of the genetic mechanisms controlling pigmentation accumulation and distribution, they identified the key proteins responsible for spot formation. The protein interactions that were also discovered display the key aspects of reaction-diffusion: diffusion, autocatalysis, and inhibition. The proteins are named nectar guide anthocyanin (NEGAN) and red tongue (RTO). NEGAN is a transcription factor responsible for production of red pigment, and its distribution over the flower can be thought of as a prepattern. Through experimentation, they found that RTO diffuses throughout the flower and NEGAN is localized in the furrows. The protein NEGAN is shown to be autocatalytic and in the presence of RTO this reaction is inhibited. Consequently, RTO is the inhibitor and NEGAN is the auto-catalytic activator.

\begin{figure}[ht]
	\centering
	\includegraphics[width=0.7\columnwidth]{monkeyflower_furrows.png}
	\caption{Image of a monkeyflower covered in raindrops. The black arrows denote regions where pattern formation occurs. \textit{Photograph: James Gaither (CC BY-NC-ND 2.0)}}
	\label{fig:monkeyflower_real}
\end{figure}

\section{Monkeyflower modelling}
I modelled monkeyflower spots on a triangular mesh using \ProgramName{}. The PDEs chosen are the activator-inhibitor model, Eqn. \eqref{eq:activatorInhibitorRD}, because of the inhibitory relationship between NEGAN and RTO. Here NEGAN is $a$ and RTO is $h$. Since in nature pattern formation is suppressed at the periphery, my model uses two sets of parameters. The first set is for the region containing NEGAN and the second is for the remaining periphery. The initial morphogen distribution has NEGAN $= 0$ and RTO $= 1$ everywhere. The boundary between the regions provides the required noise to instigate pattern formation. Consequently, spots first form on the boundary and continue to form towards the centre. This procession aligns the spots to the shape of the boundary as seen in Fig. \ref{fig:monkeyflower}. The parameters for the inner region are: $D_a=0.000025$, $D_h=0.001$, $dt=0.005$, $p=0.05$, $p_a=0.0125$, $p_h=0$, $u_a=0.05$, $u_h=0.08$. The periphery uses the same parameter values except the base production of NEGAN is $p_a=0$.

\subsection*{Results}
This model produces convincing results as compared to the real picture and has mimicked the spot positioning and general character. The spots seen on the region boundary look artificial because of how neatly they are arranged. This may be due to the sharp transition between regions and can be improved with noise or more care when they are specified. The yellow pigmentation may not directly represent NEGAN concentration and more accurate models could account for this with more morphogens.

\begin{figure}[ht]
	\centering
	\includegraphics[width=\columnwidth]{monkeyflower.png}
	\caption{Simulated and real monkeyflowers. \textbf{a:} NEGAN base production, $p_a$, is 0 in the dark region. \textbf{b:} initial appearance of the model. \textbf{c:} final appearance of the model. \textbf{d:}. a picture of a real monkeyflower.}
	\label{fig:monkeyflower}
\end{figure}

\section{Other flower models}
Orchids are bountiful in their production of beautiful and varied patterns. Foxglove and kohleria also display interesting spot and stripe patterns. I have used \ProgramName{} to model these flower patterns, shown in Fig. \ref{fig:miscFlowers}.

\textbf{Model a:} 
This model is of an orchid which displays varying sizes of purple spots on a white background. The model is produced using the Gray-Scott reaction-diffusion system, Eq \ref{eq:grayscottRD}. At the start, 400 random vertices are initialized with both activator, $a$ and substrate, $s$ values randomly chosen in $[0, 1)$. The remaining vertices have no activator and substrate concentration of 1. The simulation is then stopped after $150$ iterations. This is before the pattern can become fully stable and allows for varying sizes of spots in the resulting patterns. The parameters are: $D_a = 0.25$, $D_s = 0.5$, $dt = 0.3$, $f = 0.082$, $k = 0.063$.

\textbf{Model b:}
Some Kohleria flowers display two distinct patterns: A white background with red spots on the border of the petals, and red oriented lines that branch and radiate from the centre. The outside of the flower appears as a solid light pink. I have modelled the inner patterns using the Turing reaction-diffusion formula with anisotropy. This model uses two morphogens: $u$ and $v$. Initially they are set to $4 + [-3.0, 3.0)$ everywhere. The whole domain uses parameters for a spot pattern. The region containing line patterns has been modelled by increasing the diffusion rates with respect to the longest axis of the flower. This has the effect of spots turning into a series of branching connected lines. Both regions use the following parameters: $Du=0.3$, $Dv=0.0625$, $b=12$, $dt=0.1$, $p_v=0$, $s=0.035$, $uSat=7$, and $u_u=0$. The inner region parameters differ by $a=16$ and anisotropic diffusion is applied to $u$ using the coefficients $\lambda_{1}=0.5$, $\lambda_{2}=1$ and vectors radiating from the centre. To form spots in the outer region I use $a=16.5$ with isotropic diffusion. The number of steps used is $10000$.

\textbf{Model c:}
Foxglove has a scattered pattern of dark purple spots on the bottom inside of its flowers. These spots are surrounded by a white halo that merges with others nearby. Beyond these halos the rest of the flower appears pink or light purple. I have modelled them using the activator-depleted substrate formula, Eqn. \ref{eq:activatorSubstrateRD}. This model uses two morphogens $s$ and $a$. The morphogen $s$ is $1$ everywhere and $a$ is $0$ except for a few vertices at the bottom of the flower. These vertices have a value of $a=1$ and will grow to become the dark purple pigment spots. The parameters used are $Da=0.00004$, $Ds=0.0015$, $dt=0.005$, $p=0.05$, $p_a=0.0125$, $p_s=0$, $u_a=0.05$, and $u_s=0.08$. This simulation runs for $20700$ steps.

\textbf{Model d:}
This flower displays orange and yellow stripes across the flower petals. To model this, I have used the Turing reaction-diffusion formula. The domain is partitioned into three circular zones that increase the pattern scale as it moves from the centre to the ends of the petals. Initially the morphogens $u$ and $v$ are set to $4 + [-2.0, 2.0)$ everywhere. The whole flower uses parameters $Du=0.4$, $Dv=0.01$, $a=16$, $b=12$, $dt=0.05$, and $uSat=6.3$. Anisotropic diffusion is used to orient the pattern across the petals with both $u$ and $v$ using $\lambda_1=0.35$, $\lambda_2=1$ and directions radiating from the centre. The regions change from the centre having $s=0.005$, middle $s=0.002$, outer $s=0.001$. After $60000$ steps, this pattern has settled and there is a second phase. In this phase, the boundary conditions are changed in some sections to act as sources and sinks. In these sections $u$ and $v$ are frozen at $0$ and $30$ respectively. This simulation is then stopped after $4500$ steps, before the pattern fully settles.

\begin{figure}[p]
	\centering
	\includegraphics[width=\columnwidth]{misc_flowers.png}
	\caption{A developmental sequence of the simulated flower models and the corresponding real flowers on the right. \textbf{a-d:} orchid, koleria, foxglove, and orchid. \textit{Photographs \textbf{a,d}: courtesy of Przemyslaw Prusinkiewicz; \textbf{b,c}: courtesy of pixabay.com}}
	\label{fig:miscFlowers}
\end{figure}

\section{Discussion}

% What was accomplished?
I have implemented a biologically motivated model of monkeyflower pattern formation on a mesh. Through this simulation, a possible reason for the linear arrangement of spots is seen in the shape of a parameter boundary. I have also produced other visually similar flower models using various reaction-diffusion equations, highlighting the usefulness and flexibility of reaction-diffusion and \ProgramName{}.

Other procedural generation methods may also be useful for modelling petal patterns. Specifically, some orchid patterns appear to arise from cells passing on their colour to their descendants. This would be a good candidate for a clonal mosaic \citep{korn2007}, as there are apparent demarcations between a pigmented cell and an adjacent non-pigmented cell. Advances in imaging and gene modification may provide insight into how the pigmentation pre-patterns develop over time. Future works may investigate the role of growth and vasculature pattern formation.
  \chapter{Case Study 4: Psoriasis}

This chapter is an edited version of \Quotes{skin patterning in psoriasis by spatial interactions between pathogenic cytokines} \citep{ringham2019}. \footnote{P.P. and R.G. designed research, L.R. and P.P. created mathematical model and performed computer simulations, L.R., P.P. and R.G. wrote the paper.}

\section{Summary} 
Disorders of human skin manifest themselves with patterns of lesions ranging from simple scattered spots to complex rings and spirals. These patterns are an essential characteristic of the disease, yet the mechanisms through which they arise remain unknown. Here we show that all known patterns of psoriasis, a common inflammatory skin disease, can be explained in terms of reaction-diffusion. We constructed a computational model based on the known interactions between the main pathogenic cytokines, IL-17, IL-23 and TNF$\alpha$. Simulations revealed that the parameter space of the model contained all classes of psoriatic lesion patterns. They also faithfully reproduced the growth and evolution of the plaques, and the response to treatment by cytokine targeting. Thus, pathogenesis of inflammatory diseases, such as psoriasis, may readily be understood in the framework of the stimulatory and inhibitory interactions between a few diffusing mediators.

\section{Introduction} 
Most skin diseases manifest themselves with reproducible patterns of skin lesions, which are conventionally described in terms of lesion morphology (e.g. macules, papules, plaques, etc) and distribution on the skin surface \citep{nast2016}. The biological basis of pattern formation is only understood in a few special cases. For instance, the segmental pattern of herpes zoster reflects dermatomal viral reactivation through sensory nerves, and the linear pattern in Blaschko lines represents genetic mosaicism. In most cases, however, the mechanisms by which pathological processes in the skin generate reproducible patterns remain virtually unknown \citep{nast2016}.
The majority of skin diseases are inflammatory, which explains why the lesions are often red, elevated and scaly (resulting from, respectively: vasodilation and hyperemia, inflammatory infiltrate and edema, and pathologically increased epidermal keratinization secondary to inflammation). The skin has a large surface (average 1.5 m$^2$ - 2.0 m$^2$) compared to its thickness (0.5 mm-4 mm; the surface-to-volume ratio of approximately 650 m$^2$/m$^3$) \citep{leider1949}, and is therefore ideally suited to study the mechanisms of spatial propagation of inflammatory processes in a tissue. Psoriasis, a chronic, autoimmune inflammatory skin disease affecting $2\%-3\%$ of the population in Western countries \citep{parisi2013} provides a particularly useful model. The lesions are sharply demarcated, scaly, and distributed symmetrically on the body \citep{christophers2001, griffiths2007, nestle2009}. The plaques evolve from pinpoint papules by centrifugal growth, which explains an oval contour of mature lesions \citep{farber1985, soltani1972}. Individual plaques may merge producing polycyclic contours \citep{christophers2001, farber1985}. In some instances the plaques have the appearance of rings (referred to as annular, arciform or circinate patterns) \citep{christophers2001, nast2016}, which is the predominant morphological feature in approximately $5\%$ of patients \citep{morris2001}. The mechanisms responsible for these patterns are not readily explainable in terms of the lateral propagation of inflammation, in which one would expect gradual attenuation of inflammation due to the dilution of proinflammatory agents that diffuse in the skin. In contrast, in psoriatic lesions the intensity of inflammation is preserved throughout the whole plaque and sharply suppressed at its margin over the distance of a few millimeters. We show that the phenotypic features of psoriasis can be explained in terms of interactions between key pathogenic cytokines consistent with a reaction-diffusion model. This model captures all cardinal phenotypic features of psoriasis and may provide a wider framework to understand the patterning and maintenance of inflammation in other skin diseases. 

\section{Results}
\subsection{Classification of psoriasis plaque patterns}
The patterns repetitively identified in the literature are listed in Fig. \ref{fig:1}, see the full paper for additional details.

\begin{figure}[hb]
	\centering
	\includegraphics[width=1.0\columnwidth]{Realpso.png}
	\caption{Patterns of skin lesions psoriasis.}
	\label{fig:1}
\end{figure}

\subsection{Model of cytokine interactions in psoriasis}
Cytokines IL-23, IL-17 and TNF$\alpha$ are central mediators in the psoriatic plaque formation, as underscored by the fact that pharmacological blockade of either cytokine by monoclonal antibodies causes clinical remission in a large proportion of patients \citep{jabbar2017}. Interactions between the cytokines inferred from the available data are shown schematically in Fig. \ref{fig:2}A. The most important pathogenic cytokines are those of the IL-17 family being produced primarily by the T$_{\text{H}}17$ lymphocytes (interaction \textbf{0}) \citep{krueger2012}. These cells require IL-23 for expansion and activation \citep{cosmi2008, wilson2007, zheng2007}, and amplify the inflammatory process by inducing other proinflammatory cytokines, the most important of which is TNF$\alpha$ \citep{boehncke2015}. Psoriatic plaques contain both dendritic cells producing IL-23 and T$_{\text{H}}17$ cells expressing the IL-23 receptor \citep{cosmi2008, lee2004, tillack2014, wilson2007}. Treatment with guselkumab, a selective therapeutic monoclonal antibody inhibiting IL-23, attenuates IL-17s in psoriatic plaques and in serum in patients with psoriasis (interaction \textbf{1}) \citep{hawkes2018,sofen2014,tillack2014}. This attenuation is correlated with the clinical clearing of psoriasis lesions \citep{sofen2014}. IL-17 and TNF$\alpha$ synergize with each other \citep{alzabin2012, krueger2012, xu2017}: IL-17 increases the expression of TNF$\alpha$ \citep{jovanovic1998} (interaction \textbf{2}), whereas therapeutic TNF$\alpha$ inhibition blocks IL-17 in responding patients (interaction \textbf{3}) \citep{zaba2007, zaba2009}. The positive feedback of IL-17 cytokines on their own production (interactions \textbf{2} and \textbf{3}) is further demonstrated by the findings that IL-17A induces IL-17C \citep{xu2018}, and that the therapeutic inhibition of the IL-17 receptor with brodalumab reduces the expression of the IL-17 cytokine (IL-17A, C, F) \citep{russell2014}. TNF$\alpha$ downregulates IL-23 (interaction \textbf{4}) either directly \citep{notley2008, zakharova2005} or indirectly via inhibition of interferons \citep{palucka2005, tillack2014}. Disturbance of this negative interaction is probably responsible for paradoxical induction of psoriasis in patients with rheumatoid arthritis and inflammatory bowel disease treated with TNF$\alpha$ antibodies \citep{palucka2005, tillack2014}. That induction is readily reverted by therapeutic inhibition of the excess of IL-23 by ustekinumab, an antibody binding to the p40 chain of IL-23 \citep{tillack2014}. 

\begin{figure}[p]
	\centering
	\incfig{drawing}
	\caption{Modeling plaque formation in psoriasis. \textbf{A)} Interactions between key cytokines involved in psoriasis plaque formation. Labels 0-4 refer to the observations from which these interactions have been inferred (see Results). \textbf{B)} A simplified diagram of interactions, in which cytokines IL-17 and TNF$\alpha$ are considered jointly. \textbf{C)} Diagram B relabeled as an activator (A) - depleted substrate (S) system. \textbf{D)} Skin representation and simulation initialization. The skin surface is partitioned into square regions. A lesion is initiated by an activated T$_{\text{H}}17$ cell (red) which is either a resident memory T-cell activated by a dendritic cell (green, interaction a) or has migrated from circulation through a capillary wall (interaction b). The area of microinflammation around the activated T$_{\text{H}}17$ cell is considered as a “seed” region and its projection to the surface (arrow c) is colored in red.  Epidermis, the upper layer of the skin is shaded in grey, capillaries in the dermis are colored in red (arterioles) and blue (venules). Skin resident memory T-cells are marked in grey. \textbf{E)} Detail of skin surface representation. Each region is two-dimensional projection of the underlying activator-depleted substrate system of proinflammatory cytokines and represents a computational cell implementing reaction system (C). These computational cells are interconnected (double arrows), allowing for the diffusion of cytokines.}
	\label{fig:2}
\end{figure}

\subsection{Computational model construction}
To analyze whether the molecular-level interactions depicted in Fig. \ref{fig:2}A can account for the observed plaque patterns and the response of the disease to treatment, we constructed a mathematical model. We followed the standard method of simplifying the modeled system to focus on its essence and make the model more amenable to analysis \citep{bak1996, gaines1977, prusinkiewicz1998}. This simplification reduced the size of the parameter space and thus, to the extent possible, the use of parameters for which quantitative data are currently unavailable. It has also related the problem of plaque pattern formation to a known class of reaction-diffusion systems, which provided guidance for the exploration of the parameter space, and facilitated the analysis and interpretation of the results.
 
We have pursued the following train of thought. The mutual promotion of cytokines IL-17 and TNF$\alpha$, represented by interactions \textbf{2} and \textbf{3} in Fig. \ref{fig:2}A, suggests that their concentrations may change in concert. Assuming this is the case, we reduced the three-substance graph in Fig. \ref{fig:2}A by representing IL-17 and TNF$\alpha$ jointly. The resulting two-substance graph (Fig. \ref{fig:2}B) has the structure of an activator-depleted substrate reaction-diffusion model \citep{gierer1972, marcon2016}(Fig. \ref{fig:2}C). In this model, the substrate S with concentration s is locally converted into the activator A with concentration a according to the canonical equations \citep{gierer1972, meinhardt1982}:

\begin{equation}
	\begin{aligned} \label{eq:1}
	\frac{\partial a}{\partial t} &= ka^2s+\rho_{a0}-\mu_a a + D_a \Lap a \\
	\frac{\partial s}{\partial t} &= -ka^2s+\rho_{s0}-\mu_s s + D_s \Lap s
	\end{aligned}
\end{equation}

The term $ka^2s$ indicates that the conversion is autocatalytically promoted by the activator, with the rate controlled by parameter $k$. Its concentration increases at the expense of the substrate, thus the activator downregulates the substrate. Parameters $\rho_{a0}$ and $\rho_{s0}$ are the rates of the base production of the activator and the substrate, and $\mu_a$ and $\mu_s$ control their turnover. The remaining terms, $D_a \Lap a \text{ and } D_s \Lap s$, represent diffusion of the activator and substrate at the rates controlled by parameters $D_a$ and $D_s$, respectively (for simplicity, diffusion is not explicitly represented in Figs. \ref{fig:2}A-C). Consistent with figures \ref{fig:2}B and C, we identify variable $a$ with the concentration of cytokines TNF$\alpha$ and IL-17, and $s$ with the concentration of IL-23:
\[a=[TNF\alpha, IL17],\text{ }s=[IL23].\]

In the simulations, a patch of skin surface (Fig. \ref{fig:2}D) is represented by an array of interconnected computational “cells”, each of which performs local computation according to Equations (\ref{eq:1}) (Fig. \ref{fig:2}E). The initial state in all simulations is a uniform distribution of IL-23 in the whole array, except for randomly distributed small “seed” areas with a high concentration of IL-17 and TNF$\alpha$. These areas represent IL-17-secreting cells (such as the T$_{\text{H}}17$-cell) that either have been activated in situ (Fig. \ref{fig:2}D, interaction \textbf{a}) or have migrated from the circulation to the skin or (Fig. \ref{fig:2}D, interaction \textbf{b}) \citep{krueger2012}. 

\subsection{Exploration of the model parameter space}
Currently, it is not feasible to measure the diffusion of cytokines in human skin and consequently, there are no experimental data to provide suggestions for the parameter values of the model. Consequently, we adopted a reverse strategy where we explored the model parameter space by searching for values that would yield psoriasis patterns observed in patients (Fig. \ref{fig:1}). To guide this search, we referred to the Gray-Scott reaction-diffusion system \citep{gray1984}, for which the parameter space has been thoroughly explored:
\begin{equation}
	\begin{aligned} \label{eq:2}
	\frac{\partial a}{\partial t} &= a^2s-(f+c)a+ D_a \Lap a \\
	\frac{\partial s}{\partial t} &= -a^2s+(1-s)f + D_s \Lap s
	\end{aligned}
\end{equation}
We observe (see also \citep{yamamoto2010, yamamoto2011}) that Equations (\ref{eq:2}) are a special case of Equations (\ref{eq:1}), where
\[ k=1,\text{   }\rho_{a0}=0,\text{   }\mu_a=f+c,\text{   }\rho_{s0}=f,\text{ and }\mu_s=f.\]
The parameter space and details of six patterns obtained for specific parameter values are shown in Fig. \ref{fig:3}. These patterns correspond visually to the six types of psoriasis identified in patients (Fig. \ref{fig:1}). Note that, consistent with the common assumption of the Gray-Scott reaction-diffusion model, the ratio of the diffusion rates of substrate and activator was set to $D_s:D_a=2$ \citep{pearson1993}. This is a departure from the much larger ratios typically used in reaction-diffusion models \citep{diego2018, gierer1972, kondo2010, lengyel1991, marcon2016, vastano1987}. On biochemical grounds, this departure is justified by the commensurate, small size of the three cytokines, implying comparable diffusion rates (see Table \ref{tab:S1}). The small ratio of diffusion rates does not preclude Turing instability and spontaneous pattern emergence for carefully chosen values of the remaining parameters (see Fig. \ref{fig:S2}). Nevertheless, the  parameter values leading to the formation of plaque patterns are compatible with the ``filtering” operation mode, in which the patterns do not emerge spontaneously in a homogeneous medium and elaborate initial pre-patterns instead \citep{diego2018, lee1993, muratov2000, pearson1993}. This latter mode is more pertinent to the development of psoriasis plaques, which is initiated by an activated T$_{\text{H}}17$ cell in the skin (Fig. \ref{fig:2}D).

\begin{figure}[H]
	\centering
	\includegraphics[scale=0.65]{Map4.pdf}
	\caption{Parameter space of the model and selected patterns. Top left: A comprehensive representation of the range of patterns generated using Equations \ref{eq:2} for different values of the synthetic parameters c and f. \textbf{A-D}: magnified views of patterns generated using select parameter values. These labels and patterns correspond to the patterns of psoriatic skin lesions identified in Fig. \ref{fig:1}.}
	\label{fig:3}
\end{figure}

\subsection{The development of lesions and response to treatment}
The simulated development of psoriasis lesions and the response to treatment are shown in Fig. \ref{fig:4}. The development was simulated by using the forward Euler method to advance the state of the reaction-diffusion model over time, given an initial random distribution of small papules. The parameter values and the initial conditions for each of these simulations are listed in supplementary Table \ref{tab:S2}, with additional information characterizing the sensitivity of simulations to the variation of (individual) parameter values collected in Table \ref{tab:S3}. Minimum values of the activator A, representing cytokines IL-17 and TNF$\alpha$, needed to initiate pattern formation are collected in Table \ref{tab:S4}. The simulated patterns shown in Fig. \ref{fig:4} A-D3 have striking resemblance to the actual patterns of psoriatic skin lesions shown in Fig. \ref{fig:1}. Next, we simulated the effect of therapy by increasing the decay rate of cytokines IL-17 and TNF$\alpha$ (activator A), which mimics real-life treatment with an anti-cytokine antibody. Interestingly, the simulated lesion clearing was not simply a time-reversal of the processes of plaque formation: the interior of the plaques cleared first, producing annular lesions (Fig. \ref{fig:4}, row 5). The residual lesions dispersed slowly, eventually disappearing entirely or leaving residual spots (Fig. \ref{fig:4}, row 6). %These results closely resemble clinical situations, in which residual annular or papular lesions are often observed (Fig. \ref{fig:S1}C). 

\begin{figure}[p]
	\centering
	\resizebox{\columnwidth}{!}{
	\includegraphics[scale=1]{development.pdf}
	}
	\caption{The simulated progression of different types of psoriatic lesions. Rows 1-3: Development of the lesions. The earliest stage of a papule (Row 1) consists of randomly distributed small seed areas. Later forms of the disease (Rows 2 and 3) correspond to patterns identified in Figs. 1 and 3. Rows 4-6: The effect of treatment simulated by increasing the decay rate of IL-17 and TNF$\alpha$. Note that the treatment does not result in a simple reversal of the original pattern development, but produces residual lesions with more activity at the margin of the plaques (Row 5). In some instances, residual papules persist (Row 6).}
	\label{fig:4}
\end{figure}

Finally, to verify that the modeling results do not critically depend on the reduction of the three-substance system in Fig. \ref{fig:2}A to the two-substance system in Fig. \ref{fig:2}B, we have constructed a simulation model corresponding directly to Fig. \ref{fig:2}A (see Supplementary Text). Guided in part by parameter values found for the two-substance model (Supplementary Tables \ref{tab:S2} and \ref{tab:S3}), we found values for which the three-substance model produces qualitatively the same plaque patterns (Table \ref{tab:S5}). This result validates the simplification underlying the two-substance model.

\begin{figure}[!htb]
  \centering
  \includegraphics[width=0.7\columnwidth]{S2}
  \caption{(Example of a pattern generated de novo using the Gray-Scott model (Equation \ref{eq:2}) after 6000 iterations. The concentration is visualized from blue to orange. Parameter values: $f=0.042, c=0.06, D_a=0.25, D_s=0.5, dt=1$. The initial conditions are a homogeneous distribution everywhere, with the addition of a small amount of noise: $a=0.22557, s=0.45219±0.000001$.}
  \label{fig:S2}
\end{figure}

\newpage

\section{Discussion}
Since the foundation of dermatology as a medical specialty in the beginning of the 19th century, morphological patterns provided a useful and robust criterion for the diagnosis and classification of skin diseases. However, the mechanisms through which skin diseases produce diverse patterns remained unknown. We have shown that all major morphological types of the common skin disease psoriasis (papular, small plaque, large plaque, and different forms of circinate patterns) can be generated by a reaction-diffusion model with different parameter values. The model is based on the currently known up- and down-regulating interactions between three proinflammatory cytokines: TNF$\alpha$, IL-23 and IL-17. These interactions are not direct chemical reactions, but are mediated by the immunologically active cells stimulating or inhibiting the release and proliferation of intermediary cytokines. The model has a spatio-temporal character, explaining the emergence of patterns during disease development and their disappearance during subsequent treatment. Reaction-diffusion thus provides a promising framework for studying mechanisms underlying the progress and treatment of psoriasis. As detailed data regarding the interaction and diffusion of cytokines involved in psoriasis become available, more elaborate models may be constructed to recreate the actual biological processes in the skin with an increased accuracy. Recent advances in the theoretical understanding of reaction-diffusion \citep{diego2018} suggest that the resulting models may also become more robust to parameter changes, currently limited to narrow ranges. Inflammatory patterns related to psoriasis are found in other diseases as well. For example, annular lesions are seen in erythema multiforme, dermatophytosis and erythema annulare centrifugum; reniform patterns in erythema gyratum repens, urticaria and lupus erythematosus; and rosettes in granuloma annulare. We thus hypothesize that reaction-diffusion models can be applied further to explain the patterns of other inflammatory skin diseases, and suggest their treatment by selective cytokine inhibition. Eventually, reaction-diffusion models could provide a framework for understanding the pathogenesis and pharmacologic intervention of a broad spectrum of skin diseases. 

\section{Limitations of the study}
The main limitation of this study is that the validity of the proposed model cannot be confirmed by direct measurements of cytokine concentration gradients in the skin. Although the diffusion rates of the cytokines are expected to be similar to each other (Table \ref{tab:S1}), which is consistent with the Gray-Scott-type model, the range of diffusion is likely to be much larger than the predicted millimeter scale due to the accrual of cytokine-secreting cells to the inflammatory infiltrate and centrifugal cell movement. Currently, the large-scale measurements of cytokine gradients in human skin are not technically feasible. 

\section{Transparent methods}
\subsection{Approach to computer modelling}
The textures used in all simulations had a resolution of 500 x 500 texels, with each texel representing a sample point of a discretized patch of the skin. Parameters of individual simulations are collected in Table \ref{tab:S1}. We assumed Neumann boundary conditions set to 0, i.e., no diffusion of activator A and substrate S across the boundary. The initial activator concentration a was set to 0 in each texel except for 50 seed spots, placed randomly across the domain.  Each spot was represented by a 3x3 array of texels with $a$ concentration of 0.5 (See Table \ref{tab:S3} for the minimum values). The initial concentration of the substrate $s$ was 1.0 everywhere.  All concentrations were represented with 32-bit floating point accuracy. 

\section{Supplementary tables and figures}
\begin{table}[!htb]
	\centering
	\begin{tabular}{|l|l|l|}
	\hline
	\textbf{Molecule}  & \textbf{MW [kDa]} & \bm{$D_{tiss}[\mu m^2 / s]$} \\ \hline 
	TNF$\alpha$ & 26   & 154.4 \\ \hline
	IL17        & 35   & 123.6 \\ \hline
	IL23        & 54.1 &  89.1 \\ \hline
	\end{tabular}
	\caption {Diffusion coefficients for the three cytokines involved in our model using the empirical formula $D_{tiss}=1.778\text{x}10^{-4}\text{xMW}^{-0.75}$ \citep{swabb1974}(Equation F in their paper). The actual rates of macromolecule transport in a tissue may differ from these estimates, as other factors may also play a role.  These include  convection, which may run in the direction opposite to the concentration-gradient-driven diffusion \citep{swabb1974}, and cell proliferation, which may be relevant to the transport of cytokines otherwise mostly confined to their mother cells.}
	\label{tab:S1}
\end{table}

\begin{table}[!htb]
	\centering
	\begin{tabular}{|l|l|l|l|l|l|l|}
	\hline
	\textbf{Name}      & \textbf{Papular} & \textbf{\thead{Small \\ Plaque}} & \textbf{\thead{Large \\ Plaque}} & \textbf{Annular} & \textbf{Rosette} & \textbf{Reniform} \\ \hline
$\mu_{[IL23]}=\rho_{[IL23]0}$& 0.046            & 0.084                 & 0.091                 & 0.001            & 0.009            & 0.011             \\ \hline
\thead{$\mu_{[TNF\alpha]}$ \\ (\textit{before treatment})} & 0.116            & 0.141                 & 0.148                 & 0.028            & 0.056            & 0.057             \\ \hline
\thead{$\mu_{[TNF\alpha]}$ \\ (\textit{during treatment})} & 0.120            & 0.1467                & 0.153                 & 0.04             & 0.0625           & 0.065             \\ \hline
maxSteps           & 12,620           & 14,000                & 143,000               & 4,500            & 3,900            & 15,500            \\ \hline
treatSteps         & 12,000           & 12,000                & 140,000               & 1,700            & 2,700            & 13,000            \\ \hline
	\end{tabular}
	\caption{Parameter values used to generate the six classes of psoriasis plaque patterns shown in Fig. \ref{fig:4}. In all simulations $k=1, \rho_{a0}=0, D_a=0.25, \text{ and } D_s=0.5$.  Simulations were carried out using forward Euler methods with time-step $dt=0.4$ for $maxSteps$ iterations, with the treatment starting after $treatSteps$ iterations.}
	\label{tab:S2}
\end{table}

\begin{table}[!htb]
\centering
\begin{tabular}{|l|l|l|l|l|l|l|}
\hline
\textbf{Name} & \textbf{Papular}       & \textbf{\thead{Small \\ Plaque}} & \textbf{\thead{Large \\ Plaque}}  & \textbf{Annular}       & \textbf{Rosette}       & \textbf{Reniform}      \\ \hline
$\rho_{s0}$			& \thead{{[}0.04510, \\ 0.04705{]}} & \thead{{[}0.08375, \\ 0.15000{]}} & \thead{{[}0.09060, \\ 0.09110{]}} & \thead{{[}0.00075, \\ 0.00285{]}} & \thead{{[}0.00875, \\ 0.00910{]}} & \thead{{[}0.01085, \\ 0.01112{]}} \\ \hline
$\rho_{a0}$			& \thead{{[}0.00000, \\ 0.00075{]}} & \thead{{[}0.00000, \\ 0.00525{]}} & \thead{{[}0.00000, \\ 0.00002{]}} & \thead{{[}0.00000, \\ 0.00022{]}} & \thead{{[}0.00000, \\ 0.00015{]}} & \thead{{[}0.00000, \\ 0.00008{]}} \\ \hline
$\mu_{s}$			& \thead{{[}0.04435, \\ 0.04725{]}} & \thead{{[}0.03475, \\ 0.08475{]}} & \thead{{[}0.09085, \\ 0.09175{]}} & \thead{{[}0.00000, \\ 0.00125{]}} & \thead{{[}0.00875, \\ 0.00910{]}} & \thead{{[}0.01085, \\ 0.01120{]}} \\ \hline
$\mu_{a}$			& \thead{{[}0.11265, \\ 0.11899{]}} & \thead{{[}0.10000, \\ 0.14180{]}} & \thead{{[}0.14785, \\ 0.14870{]}} & \thead{{[}0.01500, \\ 0.03500{]}} & \thead{{[}0.05360, \\ 0.05650{]}} & \thead{{[}0.05620, \\ 0.05750{]}} \\ \hline
$D_{s}$				& \thead{{[}0.46000, \\ 0.57500{]}} & \thead{{[}0.42500, \\ 0.80000{]}} & \thead{{[}0.46100, \\ 0.50500{]}} & \thead{{[}0.00000, \\ 1.05000{]}} & \thead{{[}0.47500, \\ 0.52500{]}} & \thead{{[}0.42500, \\ 0.61500{]}} \\ \hline
$D_{a}$				& \thead{{[}0.22000, \\ 0.27000{]}} & \thead{{[}0.17500, \\ 0.29000{]}} & \thead{{[}0.24500, \\ 0.27000{]}} & \thead{{[}0.15000, \\ 0.75000{]}} & \thead{{[}0.23500, \\ 0.27500{]}} & \thead{{[}0.22500, \\ 0.28500{]}} \\ \hline
$k$					& \thead{{[}0.95900, \\ 1.05000{]}} & \thead{{[}0.98800, \\ 1.60000{]}} & \thead{{[}0.99000, \\ 1.00200{]}} & \thead{{[}0.75000, \\ 1.35000{]}} & \thead{{[}0.98500, \\ 1.06500{]}} & \thead{{[}0.97500, \\ 1.01500{]}} \\ \hline
\end{tabular}
\caption{Ranges of parameter values resulting in patterns visually similar to those shown in Fig. \ref{fig:4}. For each varied parameter all remaining values are as in Table \ref{tab:S2}.}
\label{tab:S3}
\end{table}

\begin{table}[!htb]
\centering
\begin{tabular}{|l|l|l|l|l|l|l|}
\hline
\textbf{Pattern}                                            & \textbf{A} & \textbf{B} & \textbf{C} & \textbf{D1} & \textbf{D2} & \textbf{D3} \\ \hline
\thead{Minimum initial concentration of \\ the activator at the spots} & 0.208      & 0.244      & 0.256      & 0.088       & 0.126       & 0.127       \\ \hline
\end{tabular}
\caption{Minimum values of the activator A needed to initiate the formation of patterns shown in Fig. \ref{fig:4}.}
\label{tab:S4}
\end{table}

\begin{table}[!htb]
\centering
\begin{tabular}{|l|l|l|l|l|l|l|}
\hline
\textbf{Name}           & \textbf{Papular} & \textbf{\thead{Small \\ Plaque}} & \textbf{\thead{Large \\ Plaque}}  & \textbf{Annular} & \textbf{Rosette} & \textbf{Reniform} \\ \hline
{[}IL23{]}={[}IL23{]}0       & 0.04             & 0.045                 & 0.055                 & 0.001            & 0.009            & 0.011             \\ \hline
\thead{$\mu_{[TNF\alpha]}$ \\ (before treatment)} & 0.103            & 0.103                 & 0.115                 & 0.028            & 0.055            & 0.054             \\ \hline
\thead{$\mu_{[TNF\alpha]}$ \\ (during treatment)} & 0.107            & 0.1087                & 0.12                  & 0.04             & 0.0615           & 0.062             \\ \hline
maxSteps                     & 13,000           & 14,000                & 143,000               & 4,500            & 3,900            & 15,500            \\ \hline
treatSteps                   & 12,000           & 12,000                & 140,000               & 1,700            & 2,700            & 13,000            \\ \hline
\end{tabular}
\caption{Parameter values for generating the six classes of psoriasis plaque patterns shown in Fig. \ref{fig:4} using the three-substance model. In all simulations: $k = 1, \eta = 2, \rho_{[TNF\alpha]0} = \rho_{[IL17]0} = 0, \mu_{[IL17] = 1, D_{[IL23]} = 0.5, D_{[TNF\alpha]} = D_{[IL17]} = 0.25}$. Simulations were carried out using forward Euler methods with time-step $dt = 0.4$ for $maxSteps$ iterations, with the treatment starting after $treatSteps$ iterations.}
\label{tab:S5}
\end{table}
  \chapter{Software Performance}
A critical aspect in the design of \ProgramName{} was real-time performance. In order to support pattern formation at interactive rates, the PDEs and extensions to reaction-diffusion were required to be calculated quickly. Efficient simulation was achieved by leveraging the GPU to perform computation in a highly parallelized fashion. As shown in Table \ref{tab:perf}, the high degree of parallelization provided by the GPU was integral for increasing simulation speed in medium to large domains.   And performance of \ProgramName{} can be improved greatly by upgrading the GPU used for computation.

This speed facilitated interaction by decreasing the time between user provided input and \ProgramName{}' response. GPU integration allows a user to directly manipulate their model and explore how parameter changes effect pattern details. Another benefit was the added ability to observe pattern formation in real-time.

User productivity is also affected by software performance. Studies have shown that, when using websites, a delay greater than 1 second interrupts the user's flow of thought, and if a delay is greater than 10 seconds, the user will want to do something else \citep{nielsen1994}. Thus, it was important to minimize delays after user's actions in order to increase user productivity.

\begin{table}[p]
	\centering
	\begin{tabular}{lll} \hline
	Cell Count & GPU - NVIDIA GTX 850M (ms)    & CPU - INTEL i7-4810MQ (ms)\\ \hline
	642      & 21,117 & 22,628    \\
	10,242   & 21,964 & 91,805    \\
	21,728   & 24,387 & 194,066   \\
	40,962   & 33,250 & 378,590   \\
	163,842  & 94,017 & 2,104,099 \\ \hline 
	\end{tabular}
	\caption[Analysis of \ProgramName{} performance]{Analysis of \ProgramName{} performance. Shown is the time taken to perform 10,000 iterations of a reaction-diffusion simulation. For small domains consisting of less than 1000 cells, the CPU and GPU exhibit similar performance. When simulating on medium to large domains, the GPU outperforms the CPU by an order of magnitude or more as the number of cells increases. This benefit is still seen, even with the modest graphics card used for this test. Timings of the specific models presented in this thesis are found in Table. \ref{tab:software-performance}.}
	\label{tab:perf}
\end{table}

  \chapter{Conclusions}
\section{Contributions}
In this thesis, I have introduced \ProgramName{}, an environment for quick and efficient authoring and simulation of reaction-diffusion models on grids and triangulated manifolds. To create a model, users easily specify their equations and parameters in a text file. These models are then simulated on the CPU or by using parallelized computation on the GPU. \ProgramName{} provides advanced features like anisotropic diffusion, non-homogeneous parameters, and domain growth with adaptive subdivision to give users flexibility when creating patterns. Integration of these features and the speed of the GPU has made LRDS an powerful tool for interactively exploring a wide array of reaction-diffusion models. Furthermore, these patterns can then be used to create textures or animations of pattern development for use in computer graphics, video games, and films. \ProgramName{} also has applications in a scientific setting, allowing users to model their observations and test their biological hypotheses quickly. These models can then be used to gain a better understanding of nature.

Using \ProgramName{}, I have produced models of natural patterns on grids and triangular meshes in a series of case studies. The first study concerns simulating ladybug elytra patterns. These patterns are composed of spots, stripes, and loops and are coloured in vivid red, black, brown, and yellow. To represent the elytra, I used a triangular mesh. Using a mesh provides a flexible and natural representation of the elytra's curvature and shape compared to grid-based simulations. Pattern formation occurs directly on the mesh, avoiding any mapping distortion. The colour of the simulated patterns was chosen by referencing real ladybug images.

The next study concerns pigmentation patterns seen on snakes. These patterns can be spots and stripes or more complex patterns composed of zigzags and blotches. Growth and anisotropic diffusion augment simple patterns to create more complex ones. Pattern simulation occurs on a snake-shaped mesh.

The third study deals with simulating flower petal patterns. Although the colourful patterns seen on flower petals are a defining aspect of their appearance, few studies research simulating them. I created novel models of orchids by referencing parameters from other works as well as independently searching the parameter space. The models were expanded by varying parameters spatially and by using anisotropic diffusion. I also modelled the spotted purple flower \textit{Digitalis}, among others. Using existing protein interactions that determine pigment formation, I modelled the monkeyflower species. The monkeyflower is a rare case where the real world morphogens responsible for pattern formation are known.

In the final case study, I presented a biologically-motivated model of the autoimmune disease psoriasis. This study took a departure from modelling pigmentation patterns and ventured into the domain of medicine. A hallmark of psoriasis is the red lesions that appear on the skin with a variety of geometric patterns. These patterns are an essential characteristic of the disease, yet the mechanisms through which they arise remain unknown. We modelled the interactions between the main pathogenic cytokines, TNF$\alpha$, IL17, and IL23, to produce all known patterns of psoriasis. From this, we simulated the treatment of the disease through cytokine targeting. This computational model offers an exciting approach to understanding psoriasis better because of the rapid rate in which psoriasis can be simulated compared to the actual disease. Modelling also provides an avenue for testing treatments and possibly a future cure.

\section{Future Work}
From this thesis, two main avenues can be explored as future work. First, \ProgramName{} can be extended with more features. Coupling growth and patterning has been shown to create compelling structures in 3D \cite{harrison2002, holloway2007}. This feature would be easy to add as the essential components are already implemented. Support for volumetric domains would allow for the exploration of 3D patterning. An area of research that would benefit from this feature is the simulation of vein formation. Supporting arbitrary values for Neumann boundary conditions and more boundary conditions in general would be useful as well. Active-transport is an alternative to diffusion that could also be a useful feature in \ProgramName{} for modelling biological systems. 

Concerning future reaction-diffusion work, the effect of more sophisticated methods for computing reaction-diffusion, like higher order differential operators and time-stepping schemes, should be researched to see what effect they have on pattern formation. Another future work is to simulate the same patterns on meshes of varying resolution and quality to determine how triangle shape and size effects patterning.

The second area of future work concerns the technical aspects of \ProgramName{}. Support for CPU multi-threading, SIMD, or distributed computing would increase the software's performance and productivity of users. Being able to edit all aspects of the parameter file directly inside \ProgramName{} would increase usability. Currently, the only supported operating system is Windows, adding cross-platform support would allow \ProgramName{} to reach a more extensive user base. Finally, I intend to open-source \ProgramName{} to allow users to explore and build off my results.

%%%%%%%%%%%%%%%%%%%%%%%%%%%%%%%%%%%%%%%%%%%%%%%%%%%%%%%%%%%%%%%%%%%%%%%%
%                                                                     %%
% Bibliography                                                        %%
%                                                                     %%
%%%%%%%%%%%%%%%%%%%%%%%%%%%%%%%%%%%%%%%%%%%%%%%%%%%%%%%%%%%%%%%%%%%%%%%%

%% As required by the 2014 thesis guidelines, the bibliography must
%% appear in the table of contents. Do not remove the following lines. 
  \cleardoublepage\phantomsection
  %\nocite{*} used to find uncited references
  \addcontentsline{toc}{chapter}{Bibliography}
  \bibliographystyle{apalike}
  \bibliography{References/collection} 

%%%%%%%%%%%%%%%%%%%%%%%%%%%%%%%%%%%%%%%%%%%%%%%%%%%%%%%%%%%%%%%%%%%%%%%%
%%                                                                    %%
%% Appendicies                                                        %%
%%                                                                    %%
%%%%%%%%%%%%%%%%%%%%%%%%%%%%%%%%%%%%%%%%%%%%%%%%%%%%%%%%%%%%%%%%%%%%%%%%

  \appendix             %% Do not remove this line

  \chapter{Inputs to program}

\section{Command line arguments}
\label{appendix:CLargs}
\begin{description}[itemsep=0cm]
    \item[ModelsPath=] Path to where OBJ models are located.
    \item[ShadersPath=] Path to where shaders are located.
    \item[ColorMapsPath=] Path to where colormaps are located.
    \item[SavePath=] Path to save output to.
    \item[ConfigFile=] SimConfig filename (eg. SimConfig.txt).
    \item[SimFile=] Filename containing the starting morphogen concentrations.
    \item[Steps=] Integer number of simulation steps until program exits.
    \item[SaveOnExit] Enable saving the model when program exits.
    \item[Run] Start the simulation running.
\end{description}

% List of reserved parameter file labels and their descriptions
\section{Reserved labels used in a parameter file}
\label{appendix:Reservedlabels}
\begin{description}[itemsep=0cm]
    \item[ModelsPath:] Filepath to folder containing OBJ models.
    \item[ColorMapsPath:] Filepath to folder containing colormaps.
    \item[ShadersPath:] Filepath to folder containing shaders.
    \item[camera:] Six comma separated floating point values representing the position and look at points used to orient the camera.
    \item[model:] Nine comma separated floating point values representing the X, Y, Z vectors used to orient the domain. 
    \item[domain:] The domain as either an OBJ filename (eg. model.obj) or grid. Only manifold OBJ meshes are supported. Grid domains look for two other parameters \textit{width} and \textit{height} which denote grid resolution.
    \item[xRes:] Integer representing width in squares of a grid domain.
    \item[yRes:] Integer representing height in squares of a grid domain.
    \item[cellSize:] A float representing spatial width of a single square in a grid domain.    
    \item[simFile:] A filename of a text file containing all per vertex values such as morphogen concentrations, vector directions and principle diffusivities. (eg. simfile.rd)
    \item[colorMap:] A filename of binary file containing a 256 RGB colormap. This is used for both inside and outside the mesh (eg. color.map).
    \item[colorMapOutside:] The colormap for the outside of the mesh.
    \item[colorMapInside:] The colormap for the inside of the mesh.
    \item[growthTickLimit:] An integer representing the number of simulation steps before the domain is grown.
    \item[growing:] \textit{true} or \textit{false} to turn growth on or off.
    \item[growthX:] A float percentage representing growth percentage on global X axis.
    \item[growthY:] A float percentage representing growth percentage on global Y axis.
    \item[growthZ:] A float percentage representing growth percentage on global Z axis.
    \item[maxFaceArea:] The face area threshold as a float for adaptive subdivision.
    \item[pauseAt:] Integer number of simulation steps until program pauses.
    \item[exitAt:] Integer number of simulation steps until program exits.
    \item[morphogens:] A comma separated list of morphogen names in uppercase (eg. A, S, U, V).
    \item[initialConditions:] The start of initial condition specification.
    \item[rdModel:] Either GPU or CPU depending on desired computation mode. Also denotes the
    \item[indices:] Specifies integer indices used to define initial conditions and parameters. Valid values are: all or 1,2,3 or 1-3.
\end{description}

\section{Three-substance model of psoriasis}
To show that the results obtained for the two substance system in Fig. \ref{fig:2}B also hold for the three-substance system in Fig. \ref{fig:2}A, we have constructed a simulation model corresponding directly to Fig. \ref{fig:2}A. The equations have the form: 
\begin{equation}
	\begin{aligned}
		\frac{\partial[TNF\alpha]}{\partial t}&=\rho_{[TNF\alpha]0}-\mu_{[TNF\alpha]}[TNF\alpha]+\eta[IL17]-k[TNF\alpha]^2[IL23] \\ & \quad{} + D_{[TNF\alpha]}\Lap[TNF\alpha]\\
		\frac{\partial[IL17]}{\partial t}     &=\rho_{[IL17]0}     -\mu_{[IL17]}[IL17]                     +k[TNF\alpha]^2[IL23]+D_{[IL17]}\Lap[IL17]\\
		\frac{\partial[IL23]}{\partial t}     &=\rho_{[IL23]0}     -\mu_{[IL23]}[IL23]                     -k[TNF\alpha]^2[IL23]+D_{[IL23]}\Lap[IL23]
	\end{aligned}
\end{equation}
Parameter values resulting in the different pattern classes shown in Fig. \ref{fig:4} are collected in Table \ref{tab:S3}.

\begin{table}[ht]
\centering
\begin{tabular}{|l|l|l|l|l|l|l|}
\hline
\textbf{Name}           & \textbf{Papular} & \textbf{\thead{Small \\ Plaque}} & \textbf{\thead{Large \\ Plaque}}  & \textbf{Annular} & \textbf{Rosette} & \textbf{Reniform} \\ \hline
{[}IL23{]}={[}IL23{]}0       & 0.04             & 0.045                 & 0.055                 & 0.001            & 0.009            & 0.011             \\ \hline
\thead{$\mu_{[TNF\alpha]}$ \\ (before treatment)} & 0.103            & 0.103                 & 0.115                 & 0.028            & 0.055            & 0.054             \\ \hline
\thead{$\mu_{[TNF\alpha]}$ \\ (during treatment)} & 0.107            & 0.1087                & 0.12                  & 0.04             & 0.0615           & 0.062             \\ \hline
maxSteps                     & 13,000           & 14,000                & 143,000               & 4,500            & 3,900            & 15,500            \\ \hline
treatSteps                   & 12,000           & 12,000                & 140,000               & 1,700            & 2,700            & 13,000            \\ \hline
\end{tabular}
\caption[Parameter values for generating the six classes of psoriasis plaque patterns]{Parameter values for generating the six classes of psoriasis plaque patterns shown in Fig. \ref{fig:4} using the three-substance model. In all simulations: $k = 1, \eta = 2, \rho_{[TNF\alpha]0} = \rho_{[IL17]0} = 0, \mu_{[IL17] = 1, D_{[IL23]} = 0.5, D_{[TNF\alpha]} = D_{[IL17]} = 0.25}$. Simulations were carried out using forward Euler methods with time-step $dt = 0.4$ for $maxSteps$ iterations, with the treatment starting after $treatSteps$ iterations. The textures used in all simulations had a resolution of 500 x 500 texels, with each texel representing a sample point of a discretized patch of the skin. Parameters of individual simulations are collected in Table \ref{tab:S1}. We assumed Neumann boundary conditions set to 0, i.e., no diffusion of activator A and substrate S across the boundary. The initial activator concentration a was set to 0 in each texel except for 50 seed spots, placed randomly across the domain.  Each spot was represented by a 3x3 array of texels with $a$ concentration of 0.5 (See Table \ref{tab:S3} for the minimum values). The initial concentration of the substrate $s$ was 1.0 everywhere.  All concentrations were represented with 32-bit floating point accuracy.}
\label{tab:S5}
\end{table}

%%%%%%%%%%%%%%%%%%%%%%%%%%%%%%%%%%%%%%%%%%%%%%%%%%%%%%%%%%%%%%%%%%%%%%%%
%%                                                                    %%
%% End of document                                                    %%
%%                                                                    %%
%%%%%%%%%%%%%%%%%%%%%%%%%%%%%%%%%%%%%%%%%%%%%%%%%%%%%%%%%%%%%%%%%%%%%%%%
\end{document}          %% Do not remove this line

%%
%% End of file `thesis.tex'.
%%%%%%%%%%%%%%%%%%%%%%%%%%%%%%%%%%%%%%%%%%%%%%%%%%%%%%%%%%%%%%%%%%%%%%%